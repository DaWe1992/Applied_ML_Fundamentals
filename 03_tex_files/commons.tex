% ====================================================
% ====================================================
% USEPACKAGES
% ====================================================
% ====================================================

\usepackage[T1]{fontenc}
\usepackage[utf8]{inputenc}
\usepackage[english]{babel}

% tables
\usepackage{tabularx}
\usepackage{colortbl}
\usepackage{multirow}
\usepackage{makecell}

% tikz and colors
\usepackage{tikz}
\usepackage{xcolor}
\usepackage{pgf-pie}
\usepackage{pgfplots}
\usepackage{pgfplotstable}
\usepackage{tikzsymbols}

\usetikzlibrary{calc}
\usetikzlibrary{trees}
\usetikzlibrary{patterns}
\usetikzlibrary{shadings}
\usetikzlibrary{positioning}
\usetikzlibrary{intersections}
\usetikzlibrary{decorations.pathreplacing}

\usetikzlibrary{arrows}
\usetikzlibrary{arrows.meta}

\usetikzlibrary{shapes}
\usetikzlibrary{shapes.arrows}
\usetikzlibrary{shapes.callouts}
\usetikzlibrary{shapes.symbols}
\usetikzlibrary{shapes.geometric}

\usepgfplotslibrary{patchplots}
\usepgfplotslibrary{fillbetween}

% boxes
\usepackage[many]{tcolorbox}

% math packages and fonts
\usepackage{bm}
\usepackage{ccfonts}
\usepackage{eulervm}
\usepackage{amsmath}
\usepackage{amsfonts}
\usepackage{amssymb}
\usepackage{amsthm}
\usepackage{mathtools}
\usepackage{nicefrac}
\usepackage{slashed}
\usepackage{bbold}
\usepackage{array}
\usepackage{cancel}

% algorithms and listings
\usepackage[ruled,vlined,linesnumbered]{algorithm2e}
\usepackage{listings}
\usepackage{setspace}

\tcbuselibrary{listings}
\tcbuselibrary{breakable}
\tcbuselibrary{skins}

% misc
\usepackage{soul}
\usepackage{pifont}
\usepackage{skull}
\usepackage{multicol}
\usepackage{animate}
\usepackage{cleveref}
\usepackage{hyperref}
\usepackage{wasysym}
\usepackage[absolute,overlay]{textpos}
\usepackage[hang,flushmargin]{footmisc}
\usepackage[framemethod=tikz]{mdframed} 				% provides frame for framed figure and framed table in theme_2

% ====================================================
% ====================================================
% COLOR DEFINITIONS
% ====================================================
% ====================================================

\definecolor{myblue1}{RGB}{35,119,189}
\definecolor{myblue2}{RGB}{95,179,238}
\definecolor{myblue3}{RGB}{129,168,207}
\definecolor{myblue4}{RGB}{26,89,142}

\definecolor{myred1}{RGB}{247,12,12}

% ====================================================
% ====================================================
% COMMON COMMANDS AND DEFINITIONS
% ====================================================
% ====================================================

% math definitions
% ====================================================
% argmin, argmax
\DeclareMathOperator*{\argmax}{arg\,max}
\DeclareMathOperator*{\argmin}{arg\,min}

% integration d
\newcommand*\diff{\mathop{}\!\mathrm{d}}

% independent sign
\newcommand{\indep}{\rotatebox[origin=c]{90}{$\models$}}

% vertical and horizontal bar
\newcommand*{\vertbar}{\rule[-1ex]{0.5pt}{2.5ex}}
\newcommand*{\horzbar}{\rule[.5ex]{2.5ex}{0.5pt}}

% math cancel sign
\newcommand\hcancel[2][black]{\setbox0=\hbox{$#2$}%
	\rlap{\raisebox{.45\ht0}{\textcolor{#1}{\rule{\wd0}{1pt}}}}#2}

% column type for matrices
\newcolumntype{C}[1]{>{\centering\arraybackslash}p{#1}}

% table definitions
% ====================================================
% centered X column in tabularx
\newcolumntype{Y}{>{\centering\arraybackslash}X}

% font commands
% ====================================================
% style of hyperlinks
\newcommand{\linkstyle}[1]{\underline{\smash{\texttt{#1}}}}

% slide architecture
% ====================================================
% divide frame into two parts
\newcommand{\divideTwo}[4]{
	\begin{minipage}{#1\textwidth}
		#2
	\end{minipage}
	\hfill
	\begin{minipage}{#3\textwidth}
		#4
	\end{minipage}
}

% divide frame into two parts (start on top)
\newcommand{\divideTwoTop}[4]{
	\begin{minipage}[t]{#1\textwidth}
		#2
	\end{minipage}
	\hfill
	\begin{minipage}[t]{#3\textwidth}
		#4
	\end{minipage}
}

% listings
% ====================================================
% listings
\lstset{
	backgroundcolor=\color{lightgray!50}, 			% choose the background color; you must add \usepackage{color} or \usepackage{xcolor}
	basicstyle=\ttfamily, 						% the size of the fonts that are used for the code
	breakatwhitespace=false,        	 			% sets if automatic breaks should only happen at whitespace
	breaklines=true,                 				% sets automatic line breaking
	captionpos=b,                    				% sets the caption-position to bottom
	columns=fixed,                  			 		% Using fixed column width (for e.g. nice alignment)
 	commentstyle=\color{green!30!black}\textit,    	% comment style
 	deletekeywords={...},            				% if you want to delete keywords from the given language
 	escapeinside={\%*}{*)},          				% if you want to add LaTeX within your code
 	extendedchars=true,              				% lets you use non-ASCII characters; for 8-bits encodings only, does not work with UTF-8
 	fillcolor=\color{lightgray!70}, 				% background color of number column
 	frame=l,	                   	   				% adds a frame around the code
 	framesep=4.5mm,
 	framexleftmargin=0.9mm,
 	keepspaces=true,                 				% keeps spaces in text, useful for keeping indentation of code (possibly needs columns=flexible)
 	keywordstyle=\color{blue}\bfseries,       		% keyword style
 	language=Python,                 				% the language of the code (can be overrided per snippet)
 	otherkeywords={*,...},           				% if you want to add more keywords to the set
 	numbers=left,                    				% where to put the line-numbers; possible values are (none, left, right)
 	numbersep=6pt,                   				% how far the line-numbers are from the code
	numberstyle=\normalfont\tiny\color{black},		% line number style
 	rulecolor=\color{black},         				% if not set, the frame-color may be changed on line-breaks within not-black text
 	showspaces=false,                				% show spaces everywhere adding particular underscores; it overrides 'showstringspaces'
 	showstringspaces=false,          				% underline spaces within strings only
 	showtabs=false,                  				% show tabs within strings adding particular underscores
 	stepnumber=1,                    				% the step between two line-numbers. If it's 1, each line will be numbered
 	stringstyle=\color{red}, 					% string literal style
 	tabsize=2,	                   					% sets default tab size to 2 spaces
 	title=\lstname,                  					% show the filename of files included with \lstinputlisting; also try caption instead of title
	xleftmargin=2.1em
}
