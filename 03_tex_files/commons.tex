% ====================================================
% ====================================================
% USEPACKAGES
% ====================================================
% ====================================================

\usepackage[T1]{fontenc}
\usepackage[utf8]{inputenc}
\usepackage[english]{babel}

% tables
\usepackage{tabularx}
\usepackage{colortbl}
\usepackage{multirow}
\usepackage{makecell}

% tikz and colors
\usepackage{tikz}
\usepackage{xcolor}
\usepackage{pgf-pie}
\usepackage{pgfplots}
\usepackage{pgfplotstable}
\usepackage{tikzsymbols}

\usetikzlibrary{calc}
\usetikzlibrary{trees}
\usetikzlibrary{patterns}
\usetikzlibrary{shadings}
\usetikzlibrary{positioning}
\usetikzlibrary{intersections}
\usetikzlibrary{decorations.pathreplacing}

\usetikzlibrary{arrows}
\usetikzlibrary{arrows.meta}

\usetikzlibrary{shapes}
\usetikzlibrary{shapes.arrows}
\usetikzlibrary{shapes.callouts}
\usetikzlibrary{shapes.symbols}
\usetikzlibrary{shapes.geometric}

\usepgfplotslibrary{patchplots}
\usepgfplotslibrary{fillbetween}

% boxes
\usepackage[many]{tcolorbox}

% math packages and fonts
\usepackage{bm}
\usepackage{ccfonts}
\usepackage{eulervm}
\usepackage{amsmath}
\usepackage{amsfonts}
\usepackage{amssymb}
\usepackage{amsthm}
\usepackage{mathtools}
\usepackage{nicefrac}
\usepackage{slashed}
\usepackage{bbold}
\usepackage{array}
\usepackage{cancel}

% algorithms and listings
\usepackage[ruled,vlined,linesnumbered]{algorithm2e}
\usepackage{listings}
\usepackage{setspace}

\tcbuselibrary{listings}
\tcbuselibrary{breakable}
\tcbuselibrary{skins}

% misc
\usepackage{soul}
\usepackage{pifont}
\usepackage{skull}
\usepackage{multicol}
\usepackage{animate}
\usepackage{cleveref}
\usepackage{hyperref}
\usepackage{wasysym}
\usepackage[absolute,overlay]{textpos}
\usepackage[hang,flushmargin]{footmisc}
\usepackage[framemethod=tikz]{mdframed} 				% provides frame for framed figure and framed table in theme_2

% ====================================================
% ====================================================
% COMMON COMMANDS AND DEFINITIONS
% ====================================================
% ====================================================

% centered X column in tabularx
\newcolumntype{Y}{>{\centering\arraybackslash}X}

% slide architecture
% =============================================
% divide frame into two parts
\newcommand{\divideTwo}[4]{
	\begin{minipage}{#1\textwidth}
		#2
	\end{minipage}
	\hfill
	\begin{minipage}{#3\textwidth}
		#4
	\end{minipage}
}

% divide frame into two parts (start on top)
\newcommand{\divideTwoTop}[4]{
	\begin{minipage}[t]{#1\textwidth}
		#2
	\end{minipage}
	\hfill
	\begin{minipage}[t]{#3\textwidth}
		#4
	\end{minipage}
}
