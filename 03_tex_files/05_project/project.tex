\documentclass[t,8pt,aspectratio=169]{beamer}

% ====================================================
% ====================================================
% USEPACKAGES
% ====================================================
% ====================================================

\usepackage[T1]{fontenc}
\usepackage[utf8]{inputenc}
\usepackage[english]{babel}

% tables
\usepackage{tabularx}
\usepackage{colortbl}
\usepackage{multirow}
\usepackage{makecell}

% tikz and colors
\usepackage{tikz}
\usepackage{xcolor}
\usepackage{pgf-pie}
\usepackage{pgfplots}
\usepackage{pgfplotstable}
\usepackage{tikzsymbols}

\usetikzlibrary{calc}
\usetikzlibrary{trees}
\usetikzlibrary{patterns}
\usetikzlibrary{shadings}
\usetikzlibrary{positioning}
\usetikzlibrary{intersections}
\usetikzlibrary{decorations.pathreplacing}

\usetikzlibrary{arrows}
\usetikzlibrary{arrows.meta}

\usetikzlibrary{shapes}
\usetikzlibrary{shapes.arrows}
\usetikzlibrary{shapes.callouts}
\usetikzlibrary{shapes.symbols}
\usetikzlibrary{shapes.geometric}

\usepgfplotslibrary{patchplots}
\usepgfplotslibrary{fillbetween}

% boxes
\usepackage[many]{tcolorbox}

% math packages and fonts
\usepackage{bm}
\usepackage{ccfonts}
\usepackage{eulervm}
\usepackage{amsmath}
\usepackage{amsfonts}
\usepackage{amssymb}
\usepackage{amsthm}
\usepackage{mathtools}
\usepackage{nicefrac}
\usepackage{slashed}
\usepackage{bbold}
\usepackage{array}
\usepackage{cancel}

% algorithms and listings
\usepackage[ruled,vlined,linesnumbered]{algorithm2e}
\usepackage{listings}
\usepackage{setspace}

\tcbuselibrary{listings}
\tcbuselibrary{breakable}
\tcbuselibrary{skins}

% misc
\usepackage{soul}
\usepackage{pifont}
\usepackage{skull}
\usepackage{multicol}
\usepackage{animate}
\usepackage{cleveref}
\usepackage{hyperref}
\usepackage{wasysym}
\usepackage[absolute,overlay]{textpos}
\usepackage[hang,flushmargin]{footmisc}
\usepackage[framemethod=tikz]{mdframed} 				% provides frame for framed figure and framed table in theme_2

% ====================================================
% ====================================================
% COMMON COMMANDS AND DEFINITIONS
% ====================================================
% ====================================================

% define colors
% ====================================================
\definecolor{myblue1}{RGB}{35,119,189}
\definecolor{myblue2}{RGB}{95,179,238}
\definecolor{myblue3}{RGB}{129,168,207}
\definecolor{myblue4}{RGB}{26,89,142}

\definecolor{myred1}{RGB}{247,12,12}

% math definitions
% ====================================================
\DeclareMathOperator*{\argmax}{arg\,max}
\DeclareMathOperator*{\argmin}{arg\,min}
\newcommand*\diff{\mathop{}\!\mathrm{d}}
\newcommand{\indep}{\rotatebox[origin=c]{90}{$\models$}}

\newcommand*{\vertbar}{\rule[-1ex]{0.5pt}{2.5ex}}
\newcommand*{\horzbar}{\rule[.5ex]{2.5ex}{0.5pt}}

% math cancel sign
\newcommand\hcancel[2][black]{\setbox0=\hbox{$#2$}%
	\rlap{\raisebox{.45\ht0}{\textcolor{#1}{\rule{\wd0}{1pt}}}}#2}

% column type for matrices
\newcolumntype{C}[1]{>{\centering\arraybackslash}p{#1}}

% table definitions
% ====================================================
% centered X column in tabularx
\newcolumntype{Y}{>{\centering\arraybackslash}X}

% font commands
% ====================================================
% style of hyperlinks
\newcommand{\linkstyle}[1]{\underline{\smash{\texttt{#1}}}}

% slide architecture
% ====================================================
% divide frame into two parts
\newcommand{\divideTwo}[4]{
	\begin{minipage}{#1\textwidth}
		#2
	\end{minipage}
	\hfill
	\begin{minipage}{#3\textwidth}
		#4
	\end{minipage}
}

% divide frame into two parts (start on top)
\newcommand{\divideTwoTop}[4]{
	\begin{minipage}[t]{#1\textwidth}
		#2
	\end{minipage}
	\hfill
	\begin{minipage}[t]{#3\textwidth}
		#4
	\end{minipage}
}


% ====================================================
% ====================================================
% LAYOUT AND THEME
% ====================================================
% ====================================================

% adjust margin left and right
\setbeamersize{text margin left=25pt,text margin right=25pt}

% define colors
% ====================================================
\setbeamercolor{frametitle}{fg=black}
\setbeamercolor{itemize item}{fg=black}
\setbeamercolor{itemize subitem}{fg=black}
\setbeamercolor{caption name}{fg=black!80!white}
\setbeamercolor{section in toc}{fg=black}
\setbeamercolor{subsection in toc}{fg=gray!20!black}

% define fonts and sizes
% ====================================================
\setbeamerfont{frametitle}{series=\bf,size=\footnotesize}
\setbeamerfont{caption name}{series=\bf}
\setbeamertemplate{itemize/enumerate subbody begin}{\normalsize}
\setbeamertemplate{itemize/enumerate subsubbody begin}{\small}
\usefonttheme[onlymath]{serif}

% table of contents
% ====================================================
\makeatletter
\def\beamer@endinputifotherversion#1{}
\def\beamer@sectionintoc#1#2#3#4#5{{\large \vspace*{3mm} \textbf{#2} \hfill \textbf{#3} \par}}
\def\beamer@subsectionintoc#1#2#3#4#5#6{{\normalsize \hspace*{3mm} \textbf{#3} \hfill \textbf{#4} \par}}
\def\beamer@subsubsectionintoc#1#2#3#4#5#6#7{{\normalsize \hspace*{6mm} \textbf{#4} \hfill #5 \par}}
\makeatother

% define bullet points
% ====================================================
\setbeamertemplate{itemize item}[circle]
\setbeamertemplate{itemize subitem}{--}
\setbeamertemplate{itemize subsubitem}{\textcolor{black}{$\triangleright$}}
\setlength{\leftmargini}{3.5mm}
\setlength{\leftmarginii}{3.5mm}
\setlength{\leftmarginiii}{3.5mm}

% frametitle
% ====================================================
\setbeamertemplate{frametitle}{%
	\vspace*{1mm}
	\ifthenelse{\boolean{deeptoc}}{
		\ifnum\insertframenumber=\insertsectionstartpage%
			\vspace*{0.1mm}
    			\begin{tcolorbox}[
				skin=enhanced,
      				boxrule=0.6mm, boxsep=0mm,
     	   			lowerbox=ignored,
        			colback=orange!60!red, colframe=black,
        			borderline={0.5pt}{3pt}{black}, borderline={1pt}{2pt}{red}
    			]
        			\centering
        			\Huge\textbf\insertsectionhead\par
    			\end{tcolorbox}
		\fi%
	}{}
	\ifnum\insertframenumber=\insertsubsectionstartpage%
    		\vspace*{0.1mm}
    		\begin{tcolorbox}[
     		   	boxrule=0.4mm, boxsep=-0.5mm,
     	   		lowerbox=ignored,
     	   		colback=yellow!60!orange, colframe=black
    		]
        		\centering
        		\huge\textbf\insertsubsectionhead\par
   	 	\end{tcolorbox}
	\fi%
	\begin{tcolorbox}[
    		boxrule=0.2mm,
    		boxsep=0mm,
    		lowerbox=ignored,
    		colback=yellow, colframe=black
	]
    		\centering
    		\large\insertframetitle
	\end{tcolorbox}
	\vspace*{-2mm}
}

% header and Footer
% ====================================================

% header
\setbeamertemplate{headline}{
	% skip header line on first frame of section
	\ifnum\insertsectionstartframe=\insertframenumber%
     		\vskip-\headheight%
	\else
		\begin{beamercolorbox}[wd=\textwidth,ht=4mm,dp=1mm]{}
			\hspace*{25pt} \Fheadline \hspace{25pt}
		\end{beamercolorbox}
		\centerline{\rule{\linewidth}{.2pt}}
	\fi
}

\setbeamertemplate{footline}{%
	\centerline{\rule{\linewidth}{.2pt}}
	\begin{beamercolorbox}[wd=\textwidth,ht=2mm,dp=3mm]{}
		\hspace*{25pt} \Ffootline \hspace{25pt}
	\end{beamercolorbox}
}

% definition of header
\newcommand{\Fheadline}{
	\ifx\insertshorttitle\undefined
		\inserttitle
	\else
		\insertshorttitle
	\fi
	\hfill
	\insertsection
}

% definition of footer
\newcommand{\Ffootline}{
	\textbf{\insertauthor~(\insertshortinstitute),~\insertdate}
	\hfill
	\insertframenumber
}

% customize captions
% ====================================================
\setbeamertemplate{caption}{%
	\begin{tcolorbox}[
		colback=lightgray, colframe=black,
		width=.17\linewidth,
		height=17pt,
		boxrule=0.5pt, boxsep=-1pt
	]
		\textbf{\strut\insertcaptionname~\insertcaptionnumber%
		\usebeamertemplate{caption label separator}}%
	\end{tcolorbox}%
	\space%
	\begin{tcolorbox}[
		colback=lightgray, colframe=black,
		width=.82\linewidth,
		height=17pt,
		boxrule=0.5pt, boxsep=-1pt
	]
		\strut\insertcaption%
	\end{tcolorbox}%
}

% remove navigation symbols
\setbeamertemplate{navigation symbols}{}

% remove header line on first frame of section
% ====================================================
\makeatletter
\newcount\beamer@sectionstartframe
\beamer@sectionstartframe=1
\apptocmd{\beamer@section}{\addtocontents{nav}{\protect\headcommand{%
            \protect\beamer@sectionframes{\the\beamer@sectionstartframe}{\the\c@framenumber}}}
}{}{}
\apptocmd{\beamer@section}%
{\beamer@sectionstartframe=\c@framenumber\advance\beamer@sectionstartframe by1\relax}{}{}
\AtEndDocument{
	\immediate\write\@auxout{\string\@writefile{nav}%
        	{\noexpand\headcommand{\noexpand\beamer@sectionframes{\the\beamer@sectionstartframe}%
	{\the\c@framenumber}}}}
}{}{}

\def\beamer@startframeofsection{1}
\def\beamer@endframeofsection{1}
\def\beamer@sectionframes#1#2{%
    		\ifnum\c@framenumber<#1%
    		\else%
    			\ifnum\c@framenumber>#2%
    			\else%
    				\gdef\beamer@startframeofsection{#1}%
    				\gdef\beamer@endframeofsection{#2}%
    			\fi%
    		\fi%
}

\newcommand\insertsectionstartframe{\beamer@startframeofsection}
\newcommand\insertsectionendframe{\beamer@endframeofsection}
\makeatother

% ====================================================
% ====================================================
% COMMANDS AND GENERAL DEFINITIONS
% ====================================================
% ====================================================

% variable definitions
% ====================================================
% variable deeptoc (deep or flat table of contents)
\newcommand{\dwDeepToc}[1]{
	\newboolean{deeptoc}
	\setboolean{deeptoc}{#1}
}

% adjust font
% ====================================================
\renewcommand*{\familydefault}{\sfdefault}

% framed floatings
% ====================================================
\newmdenv[
	innerlinewidth=0.05pt,
	roundcorner=4pt,
	linecolor=black,
	innerleftmargin=6pt,
	innerrightmargin=6pt,
	innertopmargin=6pt,
	innerbottommargin=6pt
]{mybox}

\newmdenv[
	innerlinewidth=0.05pt,
	roundcorner=4pt,
	linecolor=black,
	innerleftmargin=0pt,
	innerrightmargin=0pt,
	innertopmargin=0pt,
	innerbottommargin=-1pt
]{mytablebox}

% framed figure
\newcommand{\dwFigure}[3]{
	\begin{figure}
		\begin{mybox}
			\centering #1
  		\end{mybox}
  		\vspace{-4mm}
  		\caption{#2}
  		\label{#3}
	\end{figure}
	\vspace{-3mm}
}

% framed table
\newcommand{\dwTable}[4]{
	\begin{table}
		\begin{mytablebox}
			\renewcommand{\arraystretch}{#4} #1
		\end{mytablebox}
		\vspace{-2.5mm}
		\caption{#2}
		\label{#3}
	\end{table}
}

% sections and subsections
% ====================================================
% section command
\newcommand{\dwSection}[1]{
	\ifthenelse{\boolean{deeptoc}}{
		\section{#1}
	}{
		\section{#1}
		\subsection*{#1}
	}
}

% subsection command
\newcommand{\dwSubsection}[1]{\subsection{#1}}

% header
% ====================================================
% green header
\newcommand{\dwHeader}[1]{%
	\begin{tcolorbox}[
		boxrule=0.2mm, boxsep=-1mm,
		lowerbox=ignored,
		colback=green, colframe=black,
		hbox
	]
		\textbf{#1}
	\end{tcolorbox}
}

% alert box / info box
% ====================================================
\newcommand{\dwAlertBox}[1]{
	\vspace*{2mm}\hspace*{0.25mm}
	\begin{minipage}[c]{0.05\textwidth}
		\begin{tikzpicture}[rotate=180, transform shape]\thicki\end{tikzpicture}
	\end{minipage}
	\hfill
	\begin{minipage}[c]{0.92\textwidth}
		\begin{mybox}
			\textcolor{red}{\textbf{#1}}
		\end{mybox}
	\end{minipage}
}

\newcommand{\dwInfoBox}[1]{
	\vspace*{2mm}\hspace*{0.25mm}
	\begin{minipage}[c]{0.05\textwidth}
		\begin{tikzpicture}\thicki\end{tikzpicture}
	\end{minipage}
	\hfill
	\begin{minipage}[c]{0.92\textwidth}
		\begin{mybox}
			\textcolor{blue}{\textbf{#1}}
		\end{mybox}
	\end{minipage}
}

\newcommand{\thicki}{
	\draw[thick,fill=lightgray] (4.2,0) -- (4.5,0.5) -- (4.8,0) -- (4.2,0) -- cycle;
	\fill (4.45,0.05) rectangle (4.55,0.25);
	\fill (4.50,0.34) circle (1.5pt);
}

% custom itemize environment
% ====================================================
\let\tempone\itemize
\let\temptwo\enditemize
\renewenvironment{itemize}{\vspace*{1.5mm}\tempone\addtolength{\itemsep}{0.5\baselineskip}}{\temptwo}
\let\tempthree\enumerate
\let\tempfour\endenumerate
\renewenvironment{enumerate}{\vspace*{1.5mm}\tempthree\addtolength{\itemsep}{0.5\baselineskip}}{\tempfour}

% frames
% ====================================================
\newenvironment{dwHeaderFrame}[1]{
	\subsubsection{#1}
	\begin{frame}{#1}
}{
	\end{frame}
}

% special pages
% ====================================================
% title page
\newcommand{\dwPrintTitle}{
	{\usebackgroundtemplate{%
		\tikz[overlay,remember picture] \node[opacity=0.9, at=(current page.center)] {
  			\includegraphics[height=\paperheight,width=\paperwidth]{../03_img/processor_red.jpg}
		};
	}
	\begin{frame}[plain]
		\begin{center}
			\begin{tcolorbox}[
				skin=enhanced,
	      			boxrule=0.6mm, boxsep=0mm,
				lowerbox=ignored,
				colback=orange!60!red, colframe=black,
				borderline={0.5pt}{3pt}{black}, borderline={1pt}{2pt}{red},
				width=\textwidth
			]
				\centering
				\Huge\textbf{\inserttitle}
			\end{tcolorbox}
			\vspace*{1.4cm}
			\begin{tcolorbox}[width=0.5\textwidth]
				\centering
				\textbf{\insertauthor} \\[2mm]
				\insertinstitute \\[2mm]
				\insertdate
			\end{tcolorbox}
			
			\begin{textblock}{1}(1,13.5)
				\includegraphics[scale=0.04]{../03_img/logo_dhbw}
			\end{textblock}
		\end{center}
	\end{frame}}
}

% table of contents
\newcommand{\dwPrintToc}[1]{
	{\makeatletter
   		\setbeamertemplate{headline}[default]
   		\def\beamer@entrycode{\vspace*{-\headheight}}
	\makeatother

	\begin{frame}[allowframebreaks]
		\begin{tcolorbox}[
			skin=enhanced,
      			boxrule=0.6mm, boxsep=0mm,
			lowerbox=ignored,
			colback=orange!60!red, colframe=black,
			borderline={0.5pt}{3pt}{black}, borderline={1pt}{2pt}{red},
			width=\textwidth
		]
			\centering
			\huge\textbf{Agenda for this Unit}
		\end{tcolorbox}
		\vspace{2mm}
		
		{\renewcommand{\baselinestretch}{1.4}
		\tableofcontents}
	\end{frame}}
}

% thank you page
\newcommand{\makethanks}{
	{\beamertemplatenavigationsymbolsempty
	\begin{frame}[plain]
		\vfill
		\begin{tcolorbox}[
			skin=enhanced,
      			boxrule=0.6mm, boxsep=0mm,
			lowerbox=ignored,
			colback=orange!60!red, colframe=black,
			borderline={0.5pt}{3pt}{black}, borderline={1pt}{2pt}{red},
			width=\textwidth
		]
			\centering
			\Huge \textbf{Thank you very much for the attention!}
		\end{tcolorbox}
		
		\vfill
		\begin{tabbing}
			\hspace*{1.5cm}\= \kill
			\textbf{Topic:} 	\> \inserttitle \\
			\textbf{Term:} 	\> \insertdate
		\end{tabbing}
		
		\vfill
		\textbf{Contact:} \\
		\insertauthor \\
		\insertinstitute \\
		\href{mailto:daniel.wehner@sap.com}{\linkstyle{daniel.wehner@sap.com}}
		
		\vfill
		\begin{center}
			\Large\textbf{Do you have any questions?}
		\end{center}
		\vfill
	\end{frame}}
}

% hyperlinks
% ====================================================
% redefine cref (add a hyperlink)
\let\chyperref\cref % save original command under a new name
\renewcommand{\cref}[1]{\hyperlink{#1}{\textcolor{blue}{$\Rightarrow$ \chyperref{#1}}}}

\newcommand{\externalurl}[2]{\href{#1}{\textcolor{blue}{$\Rightarrow$ #2}}}


% ====================================================
% ====================================================
% OPTIONS
% ====================================================
% ====================================================

% number of levels in toc
\dwDeepToc{false}

\newcommand{\modulkatalog}{\externalurl{https://www.dhbw.de/fileadmin/user/public/SP/MA/Wirtschaftsinformatik/Data_Science.pdf}{Modulkatalog}}
\newcommand{\moodle}{\externalurl{https://moodle.dhbw-mannheim.de/course/view.php?id=5811}{Moodle}}
\newcommand{\template}{\externalurl{https://de.overleaf.com/latex/templates/astronomy-and-astrophysics-template/ngdddtchkbcg}{\LaTeX{} Template}}
\newcommand{\curemannheim}{\externalurl{https://curemannheim.de/wp/}{Cure Mannheim e.\,V.}}
\newcommand{\uci}{\externalurl{https://archive.ics.uci.edu/ml/datasets.php}{UCI Machine Learning Repository}}
\newcommand{\airsim}{\externalurl{https://github.com/microsoft/AirSim}{AirSim Simulation}}

% ====================================================
% ====================================================
% PRESENTATION DATA
% ====================================================
% ====================================================

\title[Data Exploration Project]{Data Exploration Project}
\author{Daniel Wehner M.Sc., René Penkert}
\date{Sommersemester 2020}
\institute{SAP\,SE / DHBW Mannheim}

% ====================================================
% ====================================================
% BEGIN OF DOCUMENT
% ====================================================
% ====================================================

\begin{document}

% Title frame
%______________________________________________________________________
\dwPrintTitle

% Agenda
%______________________________________________________________________
\dwPrintToc

% Section: Planung und Organisatorisches
%______________________________________________________________________
\dwSection{Planung und Organisatorisches}

% Grundsätzliches zum Projekt
\begin{dwHeaderFrame}{Grundsätzliches zum Projekt}
	\begin{itemize}
		\item Name der Veranstaltung: \textit{Data Exploration Project}
		\item Laut \modulkatalog{} beträgt der Workload des Projekts \textbf{pro Person}:
		\begin{itemize}
			\item Präsenzzeit: 27 Stunden
			\item Selbststudium: 47 Stunden
		\end{itemize}
		\item Definition des Projekts aus dem \modulkatalog:
		\begin{quote}
			\glqq{}Anwendung von Methoden und Verfahren des maschinellen Lernens auf eine vorgegebene Datenbasis unter Laborbedingungen. Verwendung von üblichen Repositorien wie Hadoop/Spark/Flink/Mahout, Python-RASBT, R, etc.
			\textbf{Ein besonderer Fokus soll auf einer ganzheitlichen wirtschaftsinformatischen Betrachtung liegen.} Es soll dabei neben der informatischen Betrachtung auch der betriebswirtschaftliche Nutzen, z.\,B. anhand eines Use Cases, betrachtet werden.\grqq{}
		\end{quote}
	\end{itemize}
\end{dwHeaderFrame}


% Bearbeitung
\begin{dwHeaderFrame}{Bearbeitung}
	\dwHeader{Grundsätzliches}
	\begin{itemize}
		\item Das Projekt ist in Gruppen von \textbf{drei bis vier Studierenden} (nicht mehr und nicht weniger) zu bearbeiten.
		\item Die Organisation der Gruppen erfolgt selbständig durch die Studierenden.
			Bitte melden Sie sich, falls es Probleme bei der Gruppenfindung geben sollte.
		\item Geben Sie Ihrer Gruppe einen Namen!
		\item Jede Gruppe bearbeitet \textbf{ein anderes Thema}. Die Gruppen dürfen eigene Themen vorschlagen.
			\textit{Siehe mögliche Themenvorschläge am Ende dieser Präsentation.}
		\item Bezüglich der zu verwendenden Technologien werden keine Einschränkungen gemacht, da je nach Projektthema
			andere Technologien sinnvoll sind.
	\end{itemize}
	
	\dwAlertBox{Bei der Themenwahl ist darauf zu achten, den Umfang weder zu gering noch zu groß zu wählen!
		Bitte stimmen Sie daher das Thema mit uns ab!}
\end{dwHeaderFrame}


\begin{frame}
	\dwHeader{Zwischenpräsentation}
	\begin{itemize}
		\item Die Zwischenpräsentation beträgt \textbf{maximal 10 Minuten} und dient dem Zweck, die grobe Konzeption des Projektvorhabens darzulegen
			und zu präsentieren.
		\item Die Studierenden sind dazu angehalten, den anderen Gruppen Feedback zu geben, beziehungsweise der präsentierenden Gruppe Anregungen
			und Ideen mitzuteilen. 
		\item Die Zwischenpräsentation geht \textbf{nicht} mit in die Endwertung ein.
		\item Von Seiten der Projektgruppen sind zu diesem Termin keine Arbeitsergebnisse oder Dokumente einzureichen.
	\end{itemize}
\end{frame}


% Abgabe
\begin{dwHeaderFrame}{Abgabe}
	\begin{itemize}
		\item Neben dem erstellten \ding{182}\,\textbf{Quellcode} sind am Ende des Semesters eine \ding{183}\,\textbf{Abschlusspräsentation} sowie ein
			\ding{184}\,\textbf{Projektreport} anzufertigen.
		\item Die formalen Kriterien für die Abgabebestandteile werden auf den nachfolgenden Folien beschrieben.
		\item Sämtliche Dokumente sind \textbf{gezippt} als \texttt{*.pdf} Datei in \moodle{} einzureichen.
		\item Bitte beachten Sie folgende Namenskonventionen:
	\end{itemize}
	
	\vspace*{1mm}
	{
	\begin{tabbing}
		\hspace*{4mm}\=\hspace*{4cm}\=\kill
		\> \textbf{Abschlusspräsentation} 	\> \texttt{project\_presentation\_<group>.pdf} 	\\[1mm]
		\> \textbf{Projektreport} 			\> \texttt{project\_report\_<group>.pdf} 		\\[1mm]
		\> \textbf{Zip-Datei}				\> \texttt{project\_submission\_<group>.zip}
	\end{tabbing}}
	
	\vspace*{1mm}
	\dwInfoBox{Der Abgabetermin und weitere wichtige Termine können der \cref{tab:ablauf-projekt} entnommen werden.}
\end{dwHeaderFrame}


\begin{frame}
	\dwHeader{\ding{182} Quellcode}
	\begin{itemize}
		\item Der Quellcode ist auf einem \textbf{öffentlichen GitHub Repository} abzulegen.
		\item Dem Repository ist eine aussagekräftige \texttt{README.md} hinzuzufügen, welche mindestens Folgendes enthält:
		\begin{itemize}
			\item Zielsetzung des Projekts
			\item Eine Liste der Gruppenmitglieder
			\item Eine kurze Beschreibung, wie der Quellcode ausgeführt werden muss (Installation/Dependencies/Packages/...)
		\end{itemize}
		\item Auch die Qualität des Quellcodes geht mit in die Bewertung ein.
	\end{itemize}
	
	\dwAlertBox{Kommentieren Sie den Quellcode ausreichend! Je leichter der Code für uns zu verstehen ist, desto besser ist es für Sie.}
\end{frame}


\begin{frame}
	\dwHeader{\ding{183} Abschlusspräsentation}
	\begin{itemize}
		\item Die Abschlusspräsentation weist einen zeitlichen Umfang von \textbf{20 Minuten} (+ max. 10 Minuten für Fragen) auf.
		\item Es sind \textbf{zwei gedruckte Exemplare} der Präsentation abzugeben (zwei Folien pro Seite und einseitig bedruckt).
		\item Beachten Sie folgende Dinge für eine gute Präsentation:
		\begin{itemize}
			\item Gehen Sie auf alle wichtigen Bestandteile Ihres Themas ein.
			\item \textbf{Halten Sie unbedingt den zeitlichen Rahmen ein.} Der Spielraum betragt $\pm$ 3 Minuten.
			\item Lesen Sie nicht ab, bauen Sie stattdessen Blickkontakt zum Publikum auf.
			\item Gestalten Sie Ihren Vortrag lebendig (Modulation der Stimme, interessante Hinführung zum Thema, Fragen ins Publikum, ...).
			\item Lassen Sie Raum für Fragen und Anmerkungen.
			\item Die Folien sollten übersichtlich gestaltet sein, d.\,h. nicht zu viel Text und nutzen Sie Visualisierungen!
		\end{itemize}
	\end{itemize}
\end{frame}


\begin{frame}
	\dwHeader{\ding{184} Projektreport}
	\begin{itemize}
		\item Der Projektreport ist gemäß den Regeln des wissenschaftlichen Arbeitens anzufertigen und weist einen Umfang von
			\textbf{minimal 3 und maximal 4 Seiten} (ohne Abbildungen und ohne Anhang) auf.
		\item Zum Zwecke der Vergleichbarkeit der Abgaben nutzen Sie bitte folgendes \template{} für die Erstellung Ihres Projektreports.
		\item Der Projektreport deckt mindestens folgende Bestandteile ab:
		\begin{itemize}
			\item Thema und Motivation
			\item Related Work (\textit{welche wissenschaftlichen Publikationen gibt es zu diesem Thema bereits?})
			\item Verwendete Technologien und Bibliotheken (z.\,B. \texttt{scikit-learn}, \texttt{tensorflow}, ...)
			\item Präsentation der Ergebnisse
			\item Kritische Bewertung der Ergebnisse (\textit{\glqq{}lessons learned\grqq{}: Was hat (nicht) funktioniert und warum?})
			\item Anmerkungen zum Quellcode im Anhang (\textit{wie ist der Code auszuführen und was gibt es zu beachten?})
		\end{itemize}
	\end{itemize}
\end{frame}


% Bewertung
\begin{dwHeaderFrame}{Bewertung des Projekts}
	\begin{itemize}
		\item Die komplette Abgabe besteht aus dem Projektreport (\texttt{*.pdf}), der Abschlusspräsentation (\texttt{*.pdf}),
			sowie dem im Rahmen des Projekts erstellten Quellcode.
		\item \textbf{Nur rechtzeitig eingereichte Dokumente können bewertet werden!}
		\item Die einzelnen Bestandteile werden folgendermaßen gewichtet:
		\begin{itemize}
			\item Quellcode und Ergebnisse (50\,\%)
			\item Projektreport (30\,\%)
			\item Abschlusspräsentation (20\,\%)
		\end{itemize}
		\item Jeder Bestandteil wird mit \textbf{maximal 60 Punkten} bewertet (die Bewertungskriterien befinden sich auf den nachfolgenden Seiten).
	\end{itemize}

	\dwAlertBox{Wichtig: Das Fehlen einer Teilabgabe führt zu erheblichem Punkteabzug (unter Umständen auch zum Nichtbestehen der gesamten Veranstaltung)!}
\end{dwHeaderFrame}


\begin{frame}
	\dwHeader{Bewertungskriterien Quellcode}
	\dwTable{
		\begin{tabularx}{\textwidth}{ c | X | c }
			\textbf{Nr.}		&
			\textbf{Kriterium} 	&
			\textbf{Punktzahl}	\\ \hline\hline
			\ding{182} 		&	Das Projekt ist auf GitHub veröffentlicht und enthält
								eine aussagekräftige \texttt{README.md}.						& 	3 	\\ \hline
			\ding{183}			& 	Der Quellcode ist ausreichend kommentiert. 						& 	6 	\\ \hline
			\ding{184} 		&	Der Quellcode ist übersichtlich und ordentlich formatiert,
								intuitiv verständlich und folgt generell dem
								\textit{\glqq{}Clean Code\grqq{}} Ansatz.						&	9 	\\ \hline
			\ding{185} 		& 	Das gesamte Entwicklungsprojekt ist gut strukturiert und modular aufgebaut
								(z.\,B. Datenvorverarbeitung ist getrennt von Datenanalyse und Evaluation, etc.).
																					& 	12 	\\ \hline
			\ding{186} 		& 	Die Grundregeln des maschinellen Lernens werden berücksichtigt
								(z.\,B. korrektes Splitting, X-Val., Occam's razor, Hyper-Parameter Suche, etc.)
								und spiegeln sich im Code wider. 							&	15 	\\ \hline
			\ding{187} 		& 	Es wird eine geeignete Datenbasis ausgewählt und entsprechend
								dem vorliegenden Problem vorverarbeitet.
																					& 	15 	\\ \hline\hline
			\multicolumn{2}{ l |}{\textbf{Summe}} 											& 	\textbf{60}
		\end{tabularx}
	}{Bewertungskriterien für den Quellcode}{tab:kriterien-quellcode}{1.3}
\end{frame}


\begin{frame}
	\dwHeader{Bewertungskriterien Abschlusspräsentation}
	\dwTable{
		\begin{tabularx}{\textwidth}{ c | X | c }
			\textbf{Nr.}		&
			\textbf{Kriterium} 	&
			\textbf{Punktzahl}	\\ \hline\hline
			\multicolumn{3}{ l }{\cellcolor{lightgray!40}\textbf{Mündlicher Vortrag}} 							\\ \hline
			\ding{182} 		&	Der vorgegebene zeitliche Rahmen wird eingehalten (20 min $\pm$ 3 min).	
																					&	6	\\ \hline
			\ding{183}			& 	Die Vortragenden lesen nicht ab und
								sind imstande, frei zu sprechen (Blickkontakt). 					& 	6 	\\ \hline
			\ding{184}			&	Die Geschwindigkeit des Vortrags ist dem Inhalt angemessen.		& 	6	\\ \hline
			\ding{185}			&	Die Gruppenmitglieder können vom Publikum gestellte Fragen
								sicher beantworten.										&	12	\\ \hline
			\ding{186}			&	Der Vortrag geht auf alle relevanten Punkte ein. 					& 	9	\\ \hline
			\multicolumn{3}{ l }{\cellcolor{lightgray!40}\textbf{Vortragsfolien}} 								\\ \hline
			\ding{187}			& 	Die Folien liegen im \texttt{*.pdf}-Format vor.					&	3	\\ \hline
			\ding{188}			& 	Die Gestaltung der Folien ist übersichtlich. Inhalte werden überwiegend
								visuell dargestellt. 										& 	9	\\ \hline
			\ding{189}			& 	Alle wichtigen Aussagen sind in den Folien festgehalten.			& 	9	\\ \hline\hline
			\multicolumn{2}{ l |}{\textbf{Summe}} 											& 	\textbf{60}
		\end{tabularx}
	}{Bewertungskriterien für die Abschlusspräsentation}{tab:kriterien-praesentation}{1.3}
\end{frame}


\begin{frame}
	\dwHeader{Bewertungskriterien Projektreport}
	\dwTable{
		\begin{tabularx}{\textwidth}{ c | X | c }
			\textbf{Nr.}		&
			\textbf{Kriterium} 	&
			\textbf{Punktzahl}	\\ \hline\hline
			\ding{182} 		& 	Der Projektreport liegt im richtigen Format vor
								(\LaTeX-Vorlage und \texttt{*.pdf}-Format).						& 	3 	\\ \hline
			\ding{183} 		& 	Der Projektreport bedient sich einer sachgerechten Sprache
								und ist verständlich verfasst.								& 	3 	\\ \hline
			\ding{184} 		&	Der Projektreport ist klar und übersichtlich strukturiert.				&	3	\\ \hline
			\ding{185} 		& 	Die Vorgaben bezüglich des Seitenumfangs werden eingehalten.		& 	3	\\ \hline
			\ding{186} 		& 	Die Regeln des wissenschaftlichen Arbeitens werden berücksichtigt
								(z.\,B. Zitation). 											& 	15 	\\ \hline
			\ding{187} 		&	Inhaltlich werden alle geforderten Bereiche abgedeckt.				& 	9 	\\ \hline
			\ding{188} 		& 	Der wirtschaftliche Kontext wird ausreichend herausgestellt.			&	9	\\ \hline
			\ding{189} 		& 	Die erzielten Ergebnisse werden kritisch reflektiert und bewertet.		& 	15 	\\ \hline\hline
			\multicolumn{2}{ l |}{\textbf{Summe}} 											& 	\textbf{60}
		\end{tabularx}
	}{Bewertungskriterien für den Projektreport}{tab:kriterien-report}{1.3}
\end{frame}


\begin{frame}
	\begin{itemize}
		\item Die finale Punktzahl für das Projekt berechnet sich somit folgendermaßen:
		\begin{equation}
			\text{Punkte} = \left\lceil \frac{5}{10} \cdot \Sigma_{\text{Code}} + \frac{3}{10} \cdot \Sigma_{\text{Report}} +
				\frac{2}{10} \cdot \Sigma_{\text{Präsentation}} \right\rceil
		\end{equation}
		\item Maximal zu erreichen sind \textbf{60 Punkte}.
		\item Die Modulnote ergibt sich aus der in der Klausur erreichten Punktzahl (\textit{Applied Machine Learning Fundamentals}, max. 60 Punkte)
			und der Bewertung des Projekts. Für das Modul können somit maximal 120 Punkte erreicht werden.
		\item Es gilt der offizielle Notenschlüssel der DHBW Mannheim.
	\end{itemize}
	
	\dwAlertBox{Das Ziel des Projekts ist es weniger, Ergebnisse zu erzielen, die dem \glqq{}State of the Art\grqq{} entsprechen.
		Vielmehr steht eine korrekte (wissenschaftliche) Vorgehensweise im Vordergrund. Falls Sie keine guten Ergebnisse erzielen,
		sollten Sie jedoch darlegen, woran es gelegen haben könnte und diesbezüglich kritisch reflektieren!}
\end{frame}


% Anwesenheitspflicht
\begin{dwHeaderFrame}{Anwesenheitspflicht}
	\begin{itemize}
		\item \textbf{Aufgrund der momentanen Situation wird die gesamte Veranstaltung ausschließlich virtuell stattfinden.}
			Siehe Link \textit{\glqq{}Remote Lecture\grqq{}} in \moodle.
		\item Mit wenigen Ausnahmen basiert die (virtuelle) Anwesenheit der einzelnen Projektgruppen an den Terminen \textbf{auf freiwilliger Basis}.
		\begin{itemize}
			\item Es sollte jedoch regelmäßig Rücksprache bezüglich des Zwischenstands gehalten werden.
			\item Bitte melden Sie sich \textbf{rechtzeitig} und \textbf{eigenverantwortlich},
				falls von Ihrer Seite aus Diskussionsbedarf besteht. Nutzen Sie hierfür die Veranstaltungstermine, das \moodle-Forum,
				oder schreiben Sie uns eine E-Mail.
		\end{itemize}
		\item \textbf{Anwesenheitspflicht} besteht an folgenden Terminen (siehe \cref{tab:ablauf-projekt}):
		\begin{itemize}
			\item Einführung
			\item Zwischenpräsentation
			\item Finale Präsentation und Abgabe
		\end{itemize}
	\end{itemize}
\end{dwHeaderFrame}


% Zeitlicher Ablauf des Projekts
\begin{dwHeaderFrame}{Zeitlicher Ablauf des Projekts}
	\dwTable{
		\begin{tabularx}{\textwidth}{ c | r | r | Y | c }
			\textbf{Datum} 		&
			\textbf{von}		&
			\textbf{bis}		&
			\textbf{Bemerkung} 	&								
			\textbf{Anwesenheitspflicht}															\\ \hline\hline
			08.05.2020	&	\textbf{3:30\,pm}
									& 	4:30\,pm		& 	\textbf{Einführung}				&	ja	\\ \hline
			15.05.2020	&	3:00\,pm	& 	4:00\,pm		&	\textit{Abstimmungstermin}			& 	nein	\\ \hline
			22.05.2020	& 	3:00\,pm 	& 	4:00\,pm		&	\textit{Abstimmungstermin}			&	nein	\\ \hline
			29.05.2020	&	\textbf{4:00\,pm} 
									& 	5:00\,pm		& 	\textit{Abstimmungstermin}			&	nein	\\ \hline
			05.06.2020	&	3:00\,pm 	& 	6:00\,pm		& 	\textbf{Zwischenpräsentation}		&	ja	\\ \hline
			10.06.2020	&	3:00\,pm 	& 	4:00\,pm		& 	\textit{Abstimmungstermin}			&	nein	\\ \hline
			19.06.2020	&	\textbf{4:30\,pm}
									& 	5:30\,pm		& 	\textit{Abstimmungstermin}			&	nein	\\ \hline
			26.06.2020	& 	3:00\,pm 	& 	4:00\,pm		& 	\textit{Abstimmungstermin}			& 	nein 	\\ \hline
			03.07.2020	& 	3:00\,pm	& 	4:00\,pm		& 	\textit{Abstimmungstermin}			& 	nein 	\\ \hline
			10.07.2020	& 	\textbf{3:30\,pm} 
									& 	4:30\,pm		& 	\textit{Abstimmungstermin}			& 	nein 	\\ \hline
			17.07.2020	& 	3:00\,pm 	&	4:00\,pm		& 	\textbf{Abgabe Moodle}			& 	nein 	\\ \hline
			20.07.2020	& 	\textbf{9:00\,am}
									& 	12:00\,pm		& 	\textbf{Finale Präsentation I}		& 	ja 	\\ \hline
			21.07.2020	& 	\textbf{9:00\,am} 
									& 	12:00\,pm		& 	\textbf{Finale Präsentation II}		& 	ja
		\end{tabularx}
	}{Alle wichtigen Termine der Veranstaltung}{tab:ablauf-projekt}{1.1}
\end{dwHeaderFrame}


\begin{frame}
	\begin{itemize}
		\item Es können sich kurzfristig Änderungen an den Terminen geben. Falls es zu Abweichungen kommt, wird dies rechtzeitig vorher bekannt gegeben.
		\item Aus zeitlichen Gründen wird es zwei getrennte Termine für die Abschlusspräsentation geben:
		\begin{itemize}
			\item 20.07.2020, 09:00\,am -- 12:00\,pm
			\item 21.07.2020, 09:00\,am -- 12:00\,pm
		\end{itemize}
		\item Um keine Gruppe zu benachteiligen, müssen \textbf{alle Gruppen} ihre Abgabe (inklusive der Abschlusspräsentation)
			\textbf{bereits am 17.07.2020 hochladen}. Das gilt unabhängig davon, wann die Präsentation gehalten wird.
	\end{itemize}
\end{frame}


% Section: Themen
%______________________________________________________________________
\dwSection{Themen}

% Eigene Themen
\begin{dwHeaderFrame}{Eigene Themen}
	\begin{itemize}
		\item Laut \modulkatalog{} soll der Fokus auf einer \textit{\glqq{}ganzheitlichen wirtschaftsinformatischen Betrachtung\grqq{}} liegen,
			und auch dem betriebswirtschaftlichen Aspekt Rechnung getragen werden.
		\item Es ist grundsätzlich erlaubt und auch erwünscht, \textbf{eigene Themenvorschläge} einzubringen.
		\item Eigene Themen müssen natürlich vorher genehmigt werden.
		\item Auf der nächsten Folie sind einige Projektvorschläge aufgelistet, falls einzelne Gruppen kein eigenes Thema finden sollten.
	\end{itemize}
\end{dwHeaderFrame}

% Themenvorschläge
\begin{dwHeaderFrame}{Themenvorschläge}
	\begin{itemize} 
		\item \texttt{Sentiment Analyse von Kundenrezensionen} (kommt das Produkt beim Kunden gut oder schlecht an?)
		\item \texttt{Vorhersage von Aktienkursen} (falls Sie reich werden möchten, ist das ein guter Anfang)
		\item \texttt{Recommender Systems} (z.\,B. \textit{Collaborative Filtering}, siehe Netflix)
		\item \texttt{Baue deinen eigenen Chatbot} (z.\,B. zur automatischen Beantwortung von Kundenfragen)
		\item \texttt{Spracherkennung}
		\item \texttt{Analyse medizinischer Scans zur Krankheitsdiagnose}
		\item \texttt{CureMannheim} (autonomes Fahren, \curemannheim)
		\item \texttt{Automatische Steuerung von Drohnen} (virtuell: \airsim)
		\item \texttt{Erkennung von \glqq{}Fake News\grqq{}} (Falschmeldungen können wirtschaftliche Entscheidungen stark beeinflussen)
	\end{itemize}
\end{dwHeaderFrame}


\begin{frame}
	\begin{itemize}
		\item \texttt{Vorhersage des Bitcoin Preises}
		\item \texttt{Aufdeckung von Kreditkartenbetrug}
		\item \texttt{Segmentierung von Kunden} (Unterteilen Sie Kunden bezüglich ihres Kaufverhaltens, Alters, Geschlechts, ...)
		\item \texttt{Identifikation von Emotionen in Texten / Audio}
	\end{itemize}
	
	\vspace*{4mm}
	Für zusätzliche Inspiration, siehe das \uci!
\end{frame}


% Thank you
%______________________________________________________________________
\makethanks

\end{document}