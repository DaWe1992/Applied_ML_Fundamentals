\documentclass[11pt]{article}

% ====================================================
% ====================================================
% USEPACKAGES AND IMPORTS
% ====================================================
% ====================================================

\usepackage[T1]{fontenc}
\usepackage[utf8]{inputenc}
\usepackage[english]{babel}

\usepackage{fancyhdr}

% definitions
% ====================================================
\let\titleoriginal\title
\renewcommand{\title}[1]{
	\titleoriginal{#1}
	\newcommand{\thetitle}{#1}
}

\setlength{\parskip}{\baselineskip}%
\setlength{\parindent}{0pt}%

% header and footer
\pagestyle{fancy}
\fancyhf{}
\lhead{Applied Machine Learning Fundamentals}
\rhead{\thetitle}
\cfoot{\thepage}

% ====================================================
% ====================================================
% BEGIN OF DOCUMENT
% ====================================================
% ====================================================
\begin{document}

\mytitle{Exercise 1 - Math Refresher}
\myauthor{student1, student2, student3}
\firstpage{\insertmytitle}{Winter term 2019/2020}{\insertmyauthor}

% Linear Algebra Refresher
%______________________________________________________________________
\section{Linear Algebra Refresher}

\begin{enumerate}[label=\alph*)]

% Task 1
\exercise{Matrix Operations (1 point)}{
A fellow student suggests that matrix addition and multiplication are very similar to scalar addition and multiplication, i.\,e. commutative, associative and distributive.
Is this a correct statement? Prove it mathematically or disprove it by providing at least one counter example per property (commutativity, associativity, distributivity).
}{
% >>>> your answer here <<<<
\vspace*{7cm}
}




% Task 2
\exercise{Matrix Inverse (1 point)}{
What is a matrix inverse? How can you build the inverse of a non-square matrix?
You would like to invert a matrix $\bm{M} \in \mathbb{R}^{2 \times 3}$, write down the equation for computing it and specify the dimensionality of the matrices after each single operation (e.\,g. multiplication, inverse).
}{
% >>>> your answer here <<<<
}

\newpage


% Task 3
\exercise{Eigenvectors and Eigenvalues (1 point)}{
Explain what eigenvectors and eigenvalues of a matrix $\bm{A}$ are. Why are they relevant in machine learning?
}{
% >>>> your answer here <<<<
}

\end{enumerate}

\newpage


% Statistics Refresher
%______________________________________________________________________
\section{Statistics Refresher}

\begin{enumerate}[label=\alph*)]

% Task 4
\exercise{Terminology (1 point)}{
What is a random variable? What is a probability density function (PDF)? What is a probability mass function (PMF)? What do a PDF and a PMF tell us about a random variable?
}{
% >>>> your answer here <<<<
\vspace*{7cm}
}




% Task 5
\exercise{Expectation and Variance (1 point)}{
State the general definition of expectation and variance for the probability density $f : \Omega \rightarrow \mathbb{R}$ of a discrete random variable. What do expectation and variance express?
}{
% >>>> your answer here <<<<
}

\end{enumerate}

\newpage


% Optimization
%______________________________________________________________________
\section{Optimization}

\begin{enumerate}[label=\alph*)]

% Task 6
\exercise{Numerical Optimization - Gradient Descent (5 points)}{
Implement a simple gradient descent algorithm for finding a minimum of the Rosenbrock function with $n = 2$ using Python and NumPy:
\begin{equation*}
	f(\bm{x}) = \sum_{i=1}^{n - 1} \left[ 100 (x_{i+1} - x_i^2)^2 + (x_i - 1)^2 \right]
\end{equation*}

Submit your code and a plot of the learning curve for the best run of your gradient descent implementation. Which learning rate worked best? (Hint: You need to find the first derivative(s) of $f(\bm{x})$ for $n = 2$ and iteratively evaluate them during gradient descent.
Automatic differentiation tools are not allowed for this exercise. Choose a random starting point for the parameters, for example $\bm{x} \in [-2, 2]$.)
}{
% >>>> your answer here <<<<
}

\end{enumerate}

\end{document}