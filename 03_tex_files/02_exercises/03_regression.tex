\documentclass[11pt]{article}

% ====================================================
% ====================================================
% USEPACKAGES AND IMPORTS
% ====================================================
% ====================================================

\usepackage[T1]{fontenc}
\usepackage[utf8]{inputenc}
\usepackage[english]{babel}

\usepackage{fancyhdr}

% definitions
% ====================================================
\let\titleoriginal\title
\renewcommand{\title}[1]{
	\titleoriginal{#1}
	\newcommand{\thetitle}{#1}
}

\setlength{\parskip}{\baselineskip}%
\setlength{\parindent}{0pt}%

% header and footer
\pagestyle{fancy}
\fancyhf{}
\lhead{Applied Machine Learning Fundamentals}
\rhead{\thetitle}
\cfoot{\thepage}

% ====================================================
% ====================================================
% BEGIN OF DOCUMENT
% ====================================================
% ====================================================
\begin{document}

\mytitle{Exercise 3 - Linear Regression}
\myauthor{name1, name2, name3}
\firstpage{\insertmytitle}{Winter term 2019/2020}{\insertmyauthor}

% Linear Regression
%______________________________________________________________________
\section{Linear Regression}

\begin{enumerate}[label=\alph*)]

% Task 1
\exercise{Ordinary Least Squares (5 points)}{
Implement an ordinary least squares regression model optimized with gradient descent to predict the value of a house in Boston. Please use the data set stored in \cb{\texttt{/data/bostonhousingdataset.csv}}.
The last column contains the class label which is called \cb{\texttt{medv}} (median value of owner-occupied homes in \$\,1000).
You can find an explanation of each attribute on Kaggle.\footnote{\url{https://www.kaggle.com/c/boston-housing}}
Split the data into train and test sets, evaluate and report the mean squared error (MSE) of your model on the test data set.
}{
% >>>> your answer here <<<<
\vspace*{7cm}
}




% Task 2
\exercise{Basis Function Features (3 points)}{
Compute polynomial or radial basis function features from the raw features in the data set.
Optimize the basis functions for the task (i.\,e. tune the degree of the polynomials or the means and scale of the radial basis functions). Which basis functions worked best?
}{
% >>>> your answer here <<<<
}

\newpage




% Task 3
\exercise{Regularization (2 points)}{
Explain in your own words what \textit{regularization} is, why it is beneficial and which kinds of regularization you could apply to a linear regression model. Finally, explain what \textit{ridge regression} is.
}{
% >>>> your answer here <<<<
\vspace*{7cm}
}




% Task 4
\exercise{Bonus Question 1 (1 point)}{
What are the three most important features for the prediction according to your linear regression model (without basis functions)? Explain your answer.
}{
% >>>> your answer here <<<<
}

\newpage


% Task 5
\exercise{Bonus Question 2 (1 point)}{
Plot the residuals for your regression model ($y$-axis) and the predicted values ($x$-axis). What can the residuals tell you about the performance of your model?
}{
% >>>> your answer here <<<<
}

\end{enumerate}

\end{document}