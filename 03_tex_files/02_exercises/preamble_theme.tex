\documentclass[11pt]{article}

% ====================================================
% ====================================================
% USEPACKAGES AND IMPORTS
% ====================================================
% ====================================================

\usepackage[T1]{fontenc}
\usepackage[utf8]{inputenc}
\usepackage[english]{babel}

\usepackage[
	left=2.5cm,
  	right=2cm,
  	top=2.5cm,
  	bottom=4cm,
]{geometry}

\usepackage{listings}

\usepackage{tikz}
\usepackage{fancyhdr}
\usepackage{etoolbox}
\usepackage{enumitem}

\usepackage{bm}
\usepackage{amssymb}
\usepackage{amsmath}
\usepackage{pifont}

\usepackage{hyperref}
\usepackage{tcolorbox}

\usepackage{pifont}

\renewcommand{\familydefault}{\sfdefault}

% definitions
% ====================================================

% title
\newcommand\insertmytitle{}  % empty by default
\newcommand\mytitle[1]{\renewcommand\insertmytitle{#1}}
% author
\newcommand\insertmyauthor{}  % empty by default
\newcommand\myauthor[1]{\renewcommand\insertmyauthor{#1}}

\setlength{\parskip}{\baselineskip}%
\setlength{\parindent}{0pt}%

% header and footer
\pagestyle{fancy}
\fancyhf{}
\lhead{\insertmytitle}
\rhead{\includegraphics[width=2cm]{../03_img/logo_dhbw.png}}
\cfoot{\thepage\\\vspace{0.6cm} \small Applied Machine Learning Fundamentals}
\setlength{\headsep}{1.5cm}

% listings
\lstset{
	backgroundcolor=\color{white},   				% choose the background color; you must add \usepackage{color} or \usepackage{xcolor}
	basicstyle=\ttfamily\footnotesize, 				% the size of the fonts that are used for the code
	breakatwhitespace=false,        	 			% sets if automatic breaks should only happen at whitespace
	breaklines=true,                 					% sets automatic line breaking
	captionpos=b,                    					% sets the caption-position to bottom
 	commentstyle=\color{green!30!black}\textit,    	% comment style
 	deletekeywords={...},            				% if you want to delete keywords from the given language
 	escapeinside={\%*}{*)},          				% if you want to add LaTeX within your code
 	extendedchars=true,              				% lets you use non-ASCII characters; for 8-bits encodings only, does not work with UTF-8
 	frame=tb,	                   	   				% adds a frame around the code
 	keepspaces=true,                 				% keeps spaces in text, useful for keeping indentation of code (possibly needs columns=flexible)
 	keywordstyle=\color{blue}\bfseries,       		% keyword style
 	language=Python,                 				% the language of the code (can be overrided per snippet)
 	otherkeywords={*,...},           				% if you want to add more keywords to the set
 	numbers=left,                    					% where to put the line-numbers; possible values are (none, left, right)
 	numbersep=5pt,                   				% how far the line-numbers are from the code
	numberstyle=\tiny, 						% line number style
 	rulecolor=\color{black},         				% if not set, the frame-color may be changed on line-breaks within not-black text (e.g. comments (green here))
 	showspaces=false,                				% show spaces everywhere adding particular underscores; it overrides 'showstringspaces'
 	showstringspaces=false,          				% underline spaces within strings only
 	showtabs=false,                  				% show tabs within strings adding particular underscores
 	stepnumber=1,                    				% the step between two line-numbers. If it's 1, each line will be numbered
 	stringstyle=\color{red}, 					% string literal style
 	tabsize=2,	                   					% sets default tab size to 2 spaces
 	title=\lstname,                  					% show the filename of files included with \lstinputlisting; also try caption instead of title
 	columns=fixed                   			 		% Using fixed column width (for e.g. nice alignment)
}

% commands
% ====================================================

% colored text
\newcommand{\cb}[1]{\colorbox{lightgray}{#1}}

% first page
\newcommand{\firstpage}[3]{
	\thispagestyle{empty}
	\begin{center}
		{\huge\textbf{#1}} \\
		\vspace*{10mm}
		{\large #2} \\
		\vspace*{2mm}
		#3
	\end{center}
	\vspace*{3mm}
	\hrule
	\vfill
	\begin{figure}[h]
		\centering
		\includegraphics[scale=0.05]{../03_img/logo_dhbw.png}
	\end{figure}
	\vfill
	\begin{tcolorbox}[colback=red!5,colframe=red!75!black,title=\textbf{General information}]
		The assignments are voluntary. All students who choose to participate have to form groups comprising three to four students (not more and not less).
		The groups do not have to be static, you may form new groups for each assignment.
		You have \textbf{two weeks} to answer the questions and to submit your work. The solutions are going to be presented and discussed after the
		submission deadline. Sample solutions will \textbf{not} be uploaded. However, you are free to share correct solutions with your colleagues
		\textbf{after they have been graded.}
	\end{tcolorbox}

	\begin{tcolorbox}[colback=red!5,colframe=red!75!black,title=\textbf{Formal requirements for submissions}]
		Please submit your solutions via Moodle (as a .zip file) as well as in printed form. The .zip file must contain one .pdf file for the pen-and-paper tasks
		as well as one .py file per programming task. Only pen-and-paper tasks have to be printed, you do not have to print the source code.
		Only one member of the group has to submit the solutions. Please make sure to specify the matriculation numbers (\textbf{not the names!})
		of all group members so that all participants receive the points they deserve! \\

		Please refrain from submitting hand-written solutions or images of solutions (\textit{.png} / \textit{.jpg} files). Rather use proper type-setting software like
		\LaTeX{} or other comparable programs. If you choose to use \LaTeX, you may want to use the template files provided. \\
		
		Code assignments have to be done in Python. Please submit \textit{.py} files (\textbf{no jupyter notebooks}). The following packages are allowed for code
		submissions: \texttt{numpy}, \texttt{pandas} and \texttt{scipy}. Please ask \textbf{beforehand}, if you want to use a specific package not mentioned here.
		Finally, do not use already implemented models (e.g. from \texttt{scikit-learn}).
	\end{tcolorbox}

	\begin{tcolorbox}[colback=red!5,colframe=red!75!black,title=\textbf{Grading details}]
		Your homework is going to be corrected and given back to you. Correct solutions are rewarded with a bonus for the exam which amounts to at most ten
 		percent of the exam, if all solutions submitted by you are correct (this corresponds to at most six points in the exam). It is still possible to achieve full points
		in the exam, even if you choose not to participate in the assignments (it is additional). The function which is used to compute the bonus is given by:
		
		\vspace*{3mm}
		\begin{equation}
			b(a) = \text{min}\left(B, \left\lceil \frac{B}{A^2} \cdot a^2 \right\rceil \right)
		\end{equation}
		\vspace*{1mm}

		\begin{itemize}
			\item $b$ denotes the number of bonus points you get for the exam (this is up to you)
			\item $B$ refers to the maximum attainable bonus points for the exam (six points)
			\item $A$ denotes the maximum attainable points in the assignments (40 points)
			\item $a$ is the score you achieved in the assignments (this is up to you)
		\end{itemize}
	
		\textbf{Please note:} You have to pass the exam \textbf{without the bonus points!} This means that it is not possible to turn a failing grade ($= 5.0$) into
		a passing grade ($\le 4.0$). The bonus points will be taken into account in case you have to repeat the exam (i.\,e. they do not expire if you fail the first
		attempt).
	\end{tcolorbox}

	\begin{tcolorbox}[colback=red!5,colframe=red!75!black,title=\textbf{Important!}]
		\textcolor{red}{\textbf{The solutions have to be your own work. If you plagiarize, you will lose all bonus points!}}
	\end{tcolorbox}
	\newpage
}

% exercise command
\newcommand\exercise[4][]{
	\ifx\hfuzz#1\hfuzz
  		\item #2\vspace{0.2cm}\newline #3 \\[5mm] \textbf{Solution:} \\[3mm] #4%
	\else
  		\item #2\vspace{0.2cm}\newline #3\vspace{0.5cm}\newline Solution:\newline#1  \\[5mm] \textbf{Solution:} \\ #4%
	\fi
}
