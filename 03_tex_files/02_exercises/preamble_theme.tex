\documentclass[11pt]{article}

% ====================================================
% ====================================================
% USEPACKAGES AND IMPORTS
% ====================================================
% ====================================================

\usepackage[T1]{fontenc}
\usepackage[utf8]{inputenc}
\usepackage[english]{babel}

\usepackage[
	left=2.5cm,
  	right=2cm,
  	top=2.5cm,
  	bottom=4cm,
]{geometry}

\usepackage{listings}

\usepackage{tikz}
\usepackage{fancyhdr}
\usepackage{etoolbox}
\usepackage{enumitem}

\usepackage{bm}
\usepackage{amssymb}
\usepackage{amsmath}
\usepackage{pifont}

\usepackage{hyperref}
\usepackage{tcolorbox}

\usepackage{pifont}

\renewcommand{\familydefault}{\sfdefault}

% definitions
% ====================================================

% title
\newcommand\insertmytitle{}  % empty by default
\newcommand\mytitle[1]{\renewcommand\insertmytitle{#1}}
% author
\newcommand\insertmyauthor{}  % empty by default
\newcommand\myauthor[1]{\renewcommand\insertmyauthor{#1}}

\setlength{\parskip}{\baselineskip}%
\setlength{\parindent}{0pt}%

% header and footer
\pagestyle{fancy}
\fancyhf{}
\lhead{\insertmytitle}
\rhead{\includegraphics[width=2cm]{../03_img/logo_dhbw.png}}
\cfoot{\thepage\\\vspace{0.6cm} \small Applied Machine Learning Fundamentals}
\setlength{\headsep}{1.5cm}

% listings
\lstset{
	backgroundcolor=\color{white},   				% choose the background color; you must add \usepackage{color} or \usepackage{xcolor}
	basicstyle=\ttfamily\footnotesize, 				% the size of the fonts that are used for the code
	breakatwhitespace=false,        	 			% sets if automatic breaks should only happen at whitespace
	breaklines=true,                 					% sets automatic line breaking
	captionpos=b,                    					% sets the caption-position to bottom
 	commentstyle=\color{green!30!black}\textit,    	% comment style
 	deletekeywords={...},            				% if you want to delete keywords from the given language
 	escapeinside={\%*}{*)},          				% if you want to add LaTeX within your code
 	extendedchars=true,              				% lets you use non-ASCII characters; for 8-bits encodings only, does not work with UTF-8
 	frame=tb,	                   	   				% adds a frame around the code
 	keepspaces=true,                 				% keeps spaces in text, useful for keeping indentation of code (possibly needs columns=flexible)
 	keywordstyle=\color{blue}\bfseries,       		% keyword style
 	language=Python,                 				% the language of the code (can be overrided per snippet)
 	otherkeywords={*,...},           				% if you want to add more keywords to the set
 	numbers=left,                    					% where to put the line-numbers; possible values are (none, left, right)
 	numbersep=5pt,                   				% how far the line-numbers are from the code
	numberstyle=\tiny, 						% line number style
 	rulecolor=\color{black},         				% if not set, the frame-color may be changed on line-breaks within not-black text (e.g. comments (green here))
 	showspaces=false,                				% show spaces everywhere adding particular underscores; it overrides 'showstringspaces'
 	showstringspaces=false,          				% underline spaces within strings only
 	showtabs=false,                  				% show tabs within strings adding particular underscores
 	stepnumber=1,                    				% the step between two line-numbers. If it's 1, each line will be numbered
 	stringstyle=\color{red}, 					% string literal style
 	tabsize=2,	                   					% sets default tab size to 2 spaces
 	title=\lstname,                  					% show the filename of files included with \lstinputlisting; also try caption instead of title
 	columns=fixed                   			 		% Using fixed column width (for e.g. nice alignment)
}

% commands
% ====================================================

% colored text
\newcommand{\cb}[1]{\colorbox{lightgray}{#1}}

% first page
\newcommand{\firstpage}[3]{
	\thispagestyle{empty}
	\begin{center}
		{\huge\textbf{#1}} \\
		\vspace*{10mm}
		{\large #2} \\
		\vspace*{2mm}
		#3
	\end{center}
	\vspace*{3mm}
	\hrule
	\vfill
	\begin{figure}[h]
		\centering
		\includegraphics[scale=0.05]{../03_img/logo_dhbw.png}
	\end{figure}
	\vfill
	\begin{tcolorbox}[colback=red!5,colframe=red!75!black,title=\textbf{Important}]
		Please solve the assignments in groups of 3 to 4 students. The solutions are going to be presented and discussed after the submission deadline.
		Sample solutions will not be uploaded. However, you are free to share correct solutions with your colleagues
		\textbf{after they have been graded}. Please submit your solutions via Moodle \textbf{and} in printed form.
		Only one member of the group has to submit the solutions. Therefore, make sure to specify the names of all group members.
		Please do not submit hand-written solutions, rather use proper type-setting software like \LaTeX\ or other comparable programs. \\
			
		Your homework will be corrected and given back to you. Correct solutions are rewarded with a bonus for the exam (max. 10 percent,
		if all solutions submitted are correct). \textbf{Please note:} You have to pass the exam \textbf{without the bonus points}!
		\textit{(i.e. it is not possible to turn 5.0 into 4.0)} The solutions have to be your own work. If you plagiarize, you will lose all bonus points!
	\tcblower
		\textbf{Further remarks:}
		\begin{itemize}
			\item Code assignments have to be done in \texttt{Python}
			\item The following packages are allowed: \texttt{numpy}, \texttt{pandas} \\
				(please ask, if you want to use a specific package not mentioned here)
			\item \textbf{Do not use already implemented models} (e.\,g. from \texttt{scikit-learn})
		\end{itemize}
	\end{tcolorbox}
	\newpage
}

% exercise command
\newcommand\exercise[4][]{
	\ifx\hfuzz#1\hfuzz
  		\item #2\vspace{0.2cm}\newline #3 \\[5mm] \textbf{Solution:} \\[3mm] #4%
	\else
  		\item #2\vspace{0.2cm}\newline #3\vspace{0.5cm}\newline Solution:\newline#1  \\[5mm] \textbf{Solution:} \\ #4%
	\fi
}
