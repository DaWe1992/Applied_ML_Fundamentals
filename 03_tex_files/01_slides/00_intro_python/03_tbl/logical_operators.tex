\dwTable{
	\begin{tabularx}{\textwidth}{ X | X | X}
		\textbf{Operator} & \textbf{Description} & \textbf{Example} 		\\ \hline
		< 		&	less than 				& 	\code{5 < 7}		\\ \hline
		<= 		& 	less than or equal to 		& 	\code{5 <= 7} 		\\ \hline
		== 		& 	equal to 				& 	\code{3 == (2 + 1)} 	\\ \hline
		!= 		&	not equal to 			& 	\code{4 != 42} 		\\ \hline
		>=		& 	greater than or equal to 	& 	\code{6 >= 6}		\\ \hline
		> 		& 	greater than 			& 	\code{9 > 8} 		\\ \hline\hline
		or		& 	logical or 				& 	\code{a or b} 		\\ \hline
		and 		& 	logical and 			& 	\code{a and b} 		\\ \hline
		not 		& 	logical negation 			& 	\code{not a} 		\\ \hline
		(not) in 	& 	containment 			& 	\code{2 in [3, 6, 2]} 	\\ \hline
		(not) is	& 	identity operator 		& 	\code{a is b}
	\end{tabularx}
}{Logical operators in Python}{tab:data_types}{1.4}