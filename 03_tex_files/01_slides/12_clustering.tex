\documentclass[11pt]{article}

% ====================================================
% ====================================================
% USEPACKAGES AND IMPORTS
% ====================================================
% ====================================================

\usepackage[T1]{fontenc}
\usepackage[utf8]{inputenc}
\usepackage[english]{babel}

\usepackage{fancyhdr}

% definitions
% ====================================================
\let\titleoriginal\title
\renewcommand{\title}[1]{
	\titleoriginal{#1}
	\newcommand{\thetitle}{#1}
}

\setlength{\parskip}{\baselineskip}%
\setlength{\parindent}{0pt}%

% header and footer
\pagestyle{fancy}
\fancyhf{}
\lhead{Applied Machine Learning Fundamentals}
\rhead{\thetitle}
\cfoot{\thepage}

% ====================================================
% ====================================================
% PRESENTATION DATA
% ====================================================
% ====================================================

\title[Clustering]{*** Applied Machine Learning Fundamentals *** Clustering}
\institute{SAP\,SE}
\author{M.\,Sc. Daniel Wehner}
\date{Winter term 2019/2020}
\prefix{CL}

% ====================================================
% ====================================================
% BEGIN OF DOCUMENT
% ====================================================
% ====================================================

\begin{document}

% Title frame
%______________________________________________________________________
\maketitlepage


% Lecture Overview
%______________________________________________________________________
\begin{frame}{Lecture Overview}{}
	\makeoverview{9}
\end{frame}


% Agenda
%______________________________________________________________________
\begin{frame}{Agenda for this Unit}
	\begin{multicols}{2}
		\tableofcontents
	\end{multicols}
\end{frame}


% Section: Introduction
%______________________________________________________________________
\section{Introduction}
\makedivider{Introduction}

% $k$-Means: Introduction
\begin{frame}{$k$-Means: Introduction}{}
	\begin{itemize}
		\item $k$-Means is a \highlight{clustering} algorithm and as such \textbf{unsupervised}
		\item A clustering algorithm tries to \textbf{find structure} in the data
		\item $k$ defines the number of clusters to be found
		\item Once the clusters are found, they first have to be interpreted...
		\item ...and can then be used for prediction purposes
	\end{itemize}
	
	\begin{boxBlue}
		\footnotesize
		A cluster must be \highlight{internally homogeneous}, but simultaneously \highlight{externally heterogeneous}.
		(Elements of one cluster have to be very similar, but must differ significantly from elements in other clusters.)
	\end{boxBlue}
\end{frame}


% $k$-Means: Example Use Cases
\begin{frame}{$k$-Means: Example Use Cases}{}
	\begin{itemize}
		\item \textbf{Behavioral segmentation}
		\begin{itemize}
			\item Customer segmentation (e.\,g. \highlight{sinus milieus})
			\item Creating profiles based on activity monitoring
		\end{itemize}
		\item \textbf{Sorting sensor measurements}
		\begin{itemize}
			\item Image grouping
			\item Detection of activity types in motion sensors
		\end{itemize}
		\item \textbf{Inventory categorization}
		\begin{itemize}
			\item Grouping inventory by sales activity
			\item Grouping inventory by manufacturing metrics
		\end{itemize}
		\item Many, many more, ...
	\end{itemize}
\end{frame}


\subsection{Procedure}
% $k$-Means: Procedure
\begin{frame}{$k$-Means: Procedure}{}
	\begin{itemize}
		\item \highlight{Vector quantization}
		\begin{itemize}
			\item Represent data points by a single vector (here: called \highlight{centroid}) which is close to them
			\item This is useful for \highlight{compression}!
		\end{itemize}
		\item \textbf{How to}: Create $k$ partitions ($\widehat{=}$ clusters) of the data set $\mathcal{D}$, such that
			the sum of squared deviations from the cluster centroids is \textbf{minimal}:
		\begin{equation}
			\min_{\bm{\mu}_j} \sum_{j=1}^k \sum_{\bm{x}^{(i)} \in \mathcal{D}_j} \Vert \bm{x}^{(i)} - \bm{\mu}_j \Vert^2
		\end{equation}
		\item With $\mathcal{D}_j \equiv j^{th}$ cluster, $\mu_j \equiv$ centroid of $j^{th}$ cluster
	\end{itemize}
\end{frame}


% Result: Voronoi Diagram
\begin{frame}{Result: Voronoi Diagram}{}
	\divideTwo{0.49}{
		\begin{itemize}
			\item The dots represent cluster centroids
			\item The lines visualize the \textbf{cluster boundaries}
			\item For a new point we can easily determine to which cluster it has to be assigned
		\end{itemize}
	}{0.49}{
		\begin{figure}
			\includegraphics[scale=0.2]{12_clustering/02_img/voronoi}
		\end{figure}
	}
\end{frame}


% $k$-Means Algorithm 
\begin{frame}{$k$-Means Algorithm}{}
	\begin{itemize}
		\item Input: $\mathcal{D} = \{ \bm{x}^{(1)}, \bm{x}^{(2)}, \dots, \bm{x}^{(n)} \} \in \mathbb{R}^{n \times m}$,
			Number of clusters $k$	
		\item Algorithm:
		\begin{enumerate}
			\item $t \longleftarrow 1$
			\item Randomly choose $k$ means $\bm{\mu}_1^{\langle 1 \rangle}, \bm{\mu}_2^{\langle 1 \rangle}, \dots, \bm{\mu}_k^{\langle 1 \rangle}$ 
			\item While not converged:
			\begin{itemize}
				\item[\textbf{3a}] Assign each $\bm{x}^{(i)} \in \mathcal{D}$ to the closest cluster:
				{\footnotesize
				\begin{equation*}
					\mathcal{D}_j^{\langle t \rangle}
						= \left\{
							\bm{x}^{(i)} : \Vert \bm{x}^{(i)} - \bm{\mu}_j^{\langle t \rangle} \Vert^2 \le
							\Vert \bm{x}^{(i)} - \bm{\mu}_{j^*}^{\langle t \rangle} \Vert^2;\
							\forall j^* = 1, 2, \dots, k; \bm{x}^{(i)} \in \mathcal{D}
						\right\}
				\end{equation*}}
				\item[\textbf{3b}] Update cluster centroids $\bm{\mu}_j$:
				{\footnotesize
				\begin{equation*}
					\bm{\mu}_{j}^{\langle t+1 \rangle} =
						\frac{1}{\vert \mathcal{D}_i^{\langle t \rangle} \vert} \sum_{\bm{x}^{(i)} \in \mathcal{D}_j^{\langle t \rangle}} \bm{x}^{(i)}
				\end{equation*}}
				\item[\textbf{3c}] $t \longleftarrow t + 1$
			\end{itemize}
		\end{enumerate}
	\end{itemize}
\end{frame}


% Image Compression
\begin{frame}{Image Compression}{}
	\begin{figure}
		\centering
		\includegraphics[scale=0.3]{12_clustering/02_img/image_compression}
	\end{figure}
\end{frame}


% Section: Wrap-Up
%______________________________________________________________________
\section{Wrap-Up}
\makedivider{Wrap-Up}

% Subsection: Summary
% --------------------------------------------------------------------------------------------------------
\subsection{Summary}

% Summary
\begin{frame}{Summary}{}
	\begin{itemize}
		\item 
	\end{itemize}
\end{frame}


% Subsection: Self-Test Questions
% --------------------------------------------------------------------------------------------------------
\subsection{Self-Test Questions}

% Self-Test Questions
\begin{frame}{Self-Test Questions}{}\important
	\begin{enumerate}
		\item
	\end{enumerate}
\end{frame}


% Subsection: Lecture Outlook
% --------------------------------------------------------------------------------------------------------
\subsection{Lecture Outlook}

\begin{frame}{What's next...?}{}
	\makeoverview{10}
\end{frame}


% Subsection: Recommended Literature and further Reading
% --------------------------------------------------------------------------------------------------------
\subsection{Recommended Literature and further Reading}

% Literature
%______________________________________________________________________
\begin{frame}{Recommended Literature and further Reading}{}
	\footnotesize
	\begin{thebibliography}{2}
	
	\end{thebibliography}
\end{frame}


% Thank you
%______________________________________________________________________
\makethanks

\end{document}