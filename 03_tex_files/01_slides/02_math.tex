% rubber: clean 02_math.aux
% rubber: clean 02_math.fls
% rubber: clean 02_math.out
% rubber: clean 02_math.log
\documentclass[11pt]{article}

% ====================================================
% ====================================================
% USEPACKAGES AND IMPORTS
% ====================================================
% ====================================================

\usepackage[T1]{fontenc}
\usepackage[utf8]{inputenc}
\usepackage[english]{babel}

\usepackage{fancyhdr}

% definitions
% ====================================================
\let\titleoriginal\title
\renewcommand{\title}[1]{
	\titleoriginal{#1}
	\newcommand{\thetitle}{#1}
}

\setlength{\parskip}{\baselineskip}%
\setlength{\parindent}{0pt}%

% header and footer
\pagestyle{fancy}
\fancyhf{}
\lhead{Applied Machine Learning Fundamentals}
\rhead{\thetitle}
\cfoot{\thepage}
\usepackage{hyperref}
\usepackage{movie15}

% ====================================================
% ====================================================
% PRESENTATION DATA
% ====================================================
% ====================================================

\title[Mathematical Foundations]{*** Applied Machine Learning Fundamentals *** Mathematical Foundations}
\institute{SAP\,SE}
\author{Daniel Wehner}
\date{\today}
\prefix{MATH}

% ====================================================
% ====================================================
% BEGIN OF DOCUMENT
% ====================================================
% ====================================================

\begin{document}

% Title frame
%______________________________________________________________________
\maketitlepage


% Lecture Overview
%______________________________________________________________________
\begin{frame}{Lecture Overview}{}
	\makeoverview{2}
\end{frame}


% Agenda
%______________________________________________________________________
\begin{frame}{Agenda \today}
	\begin{multicols}{2}
		\tableofcontents
	\end{multicols}
\end{frame}


% Section: Introduction
%______________________________________________________________________
\section{Introduction}
\makedivider{Introduction}

% Introduction
\begin{frame}{Introduction}{}

\end{frame}


% Section: Linear Algebra
%______________________________________________________________________
\section{Linear Algebra}
\makedivider{Linear Algebra}

% Subsection: Vectors
% --------------------------------------------------------------------------------------------------------
\subsection{Vectors}

% What is a Vector?
\begin{frame}{What is a Vector?}{}
	\divideTwo{0.49}{
		\begin{align*}
			\bm{x} 		&= \left[\begin{array}{C{1em}} $x_1$ \\ $x_2$ \end{array}\right] 	\\[4mm]
			\bm{x}^{(1)} 	&= \left[\begin{array}{C{1em}} $3$ \\ $1$ \end{array}\right] 	\\[4mm]
			\bm{x}^{(2)} 	&= \left[\begin{array}{C{1em}} $2$ \\ $4$ \end{array}\right]
		\end{align*}
	}{0.49}{
		\begin{figure}
	\centering
	\begin{tikzpicture}[
		scale=0.8,
		axis/.style={
			->,ultra thick
		}
	]
	
		% draw grid
		\draw[lightgray] (0,0) grid (5,5);
		
		\draw[very thick,myblue1,-stealth] (0,0) -- (3,1) node[right=3mm,fill=myblue1!30] {$\bm{v}$};
		\draw[very thick,gray,-stealth] (0,0) -- (2,4) node[right=3mm,fill=lightgray!50] {$\bm{w}$};
	
		% draw axes
		\draw[axis] (0,0) -- (0,5) node[above]{$x_2$}; \draw[axis] (0,0) -- (5,0) node[right]{$x_1$};
		% axis labels
		\foreach \x in {0,1,...,5} \node at (\x,-0.5){\footnotesize{\x}};
		\foreach \y in {0,1,...,5} \node at (-0.5,\y){\footnotesize{\y}};

	\end{tikzpicture}
\end{figure}
	}
\end{frame}


% Section: Statistics
%______________________________________________________________________
\section{Statistics}
\makedivider{Statistics}

% 
\begin{frame}{}{}

\end{frame}


% Section: Optimization
%______________________________________________________________________
\section{Optimization}
\makedivider{Optimization}

% 
\begin{frame}{Motivation}{Every machine learning problem is an optimization problem!}
\begin{itemize}
\item In every machine learning problem, you will have:
\begin{itemize}
\item an objective function you want to optimize
\item data you want to learn from
\item parameters which need to be learned
\item assumptions on your problem, your data and how the world works
\end{itemize}
\item Thus, we would like to have general solutions to the problem of learning
\item Machine learning provides suitable objective functions for optimization based on the data, different models embody different objective functions and assumptions
\end{itemize}
\end{frame}

\begin{frame}{Constrained Optimization}{How to formalize an optimization problem}
	\begin{align*}
		\underset{\theta}{\text{min }} J(\theta, D) &= \dots &\leftarrow\text{cost function / objective}\\
		\text{s.\,t. } f(\theta, D) &= 0 &\leftarrow\text{equality constraints}\\
		g(\theta, D) &\ge 0 &\leftarrow\text{inequality constraints}\\
	\end{align*}
	What should an ideal optimization problem, i.\,e. the cost function and constraints look like?
\end{frame}

\begin{frame}{Constrained Optimization}{How to formalize an optimization problem}
	\begin{align*}
		\underset{\theta}{\text{min }} J(\theta, D) &= \dots &\leftarrow\text{convex function}\\
		\text{s.\,t. } f(\theta, D) &= 0 &\leftarrow\text{linear function}\\
		g(\theta, D) &\ge 0 &\leftarrow\text{convex set}\\
	\end{align*}
\end{frame}

\begin{frame}{Cost Functions}{Which cost functions are there? Ideally, the cost function is convex}
	\divideTwo{0.49}{\centering \href{https://upload.wikimedia.org/wikipedia/commons/a/a3/Gradient_descent.gif}{non-convex}}{0.49}{\hspace{1cm}convex}\\
	\vspace{0.4cm}
	\includegraphics[width=0.9\textwidth]{02_math/02_img/costfunctions.png}
\end{frame}

\begin{frame}{Convexity}{Convex Sets}
	\begin{itemize}
		\item A set $C \subseteq \mathbb{R}^n$ is convex if for each $x, y \in C$ and any $\alpha \in [0,1]$, $\alpha x + (1 - \alpha)y \in C$. Examples are $\mathbb{R}^n$ and norm balls.\\
	\end{itemize}
	\begin{figure}
		\center
		\includegraphics[width=0.6\textwidth]{02_math/02_img/convexsets.png}
	\end{figure}
\end{frame}

\begin{frame}{Convexity}{Convex Functions}
	\begin{itemize}
		\item A function $f : \mathbb{R}^n \rightarrow \mathbb{R}$ is convex if for $x, y \in dom(f)$ and any $\alpha \in [0,1]$, $f(\alpha x + (1 - \alpha)y) \le \alpha f(x) + (1 - \alpha)f(y)$. Examples are linear functions $f(x) = w^Tx + b$ and quadratic functions $f(x) = x^TAx + b^Tx + c$.\\
	\end{itemize}
	\begin{figure}
		\center
		\includegraphics[width=0.6\textwidth]{02_math/02_img/convexfunction.png}
	\end{figure}
\end{frame}

\begin{frame}{Convexity}{Why are convex cost functions so nice?}
	\begin{itemize}
		\item Local solutions are global optima
		\item Efficient implementations of optimizers are available
	\end{itemize}
\end{frame}

\begin{frame}{Convexity}{How to recognize a convex function? Convexity conditions}
	\begin{itemize}
		\item First-order convexity condition:\\ Suppose $f : \mathbb{R}^n \rightarrow \mathbb{R}$ is differentiable. The function $f$ is convex iff $f(y) \ge f(x) + \nabla f(x)^T(y - x) \hspace{0.2cm}\forall x, y \in dom(f)$.
		\item Second-order convexity condition:\\ Suppose $f : \mathbb{R}^n \rightarrow \mathbb{R}$ is twice differentiable. The function $f$ is convex iff $\nabla^2 f(x) \ge 0 \vspace{0.2cm}\forall x \in dom(f)$.
	\end{itemize}
\end{frame}

\begin{frame}{Constrained Optimization}{How to solve an optimization problem with constraints}
	\begin{align*}
		\text{min } f(x,y) &= 2\cdot y + x\\
		\text{s.\,t. } 0 = g(x,y) &= y^2 +xy -1\\
	\end{align*}
	\begin{itemize}
		\item Convert the problem to an unconstrained one\\
		\item Introduce Lagrange multipliers
	\end{itemize}
\end{frame}

\begin{frame}{Constrained Optimization}{Lagrange multipliers}
	\divideTwo{0.49}{
		\begin{align*}
			\text{min } f(x,y) &= 2\cdot y + x\\
			\text{s.\,t. } 0 = g(x,y) &= y^2 +xy -1\\
		\end{align*}
	}{0.49}{
		\begin{align*}
			L(x,y,\lambda) &= f(x,y) + \lambda g(x,y)\\
			\frac{\partial L}{\partial x} &= 1 + \lambda y\\
			\frac{\partial L}{\partial y} &= 2 + 2\lambda y + \lambda x\\
			\frac{\partial L}{\partial \lambda} &= y^2 + xy - 1\\
		\end{align*}
	}
\end{frame}

\begin{frame}{Constrained Optimization}{Lagrange multipliers}
	\divideTwo{0.49}{
		\begin{align*}
			\text{I.    }0 &= 1 + \lambda y\\
			\text{II.   }0 &= 2 + 2\lambda y + \lambda x\\
			\text{III.  }0 &= y^2 + xy - 1\\
		\end{align*}
	}{0.49}{
		\begin{align*}
			\text{I.  }\lambda &= -\frac{1}{y}\\
			\text{I.} \rightarrow \text{II.  }x &= 0\\
			\text{III. }y &= \pm 1
		\end{align*}
	}
\end{frame}

\begin{frame}{Numerical Optimization}{What to do if we cannot solve it analytically?}
	\begin{itemize}
		\item Different numerical optimization algorithms exist for optimizing a function numerically on a computer if we can't solve it analytically
		\item Many approaches incrementally update an estimate $\theta_{new} := \theta_{old} + \alpha \delta\theta$ of the optimal parameters, so that after each update $J(\theta_{new}) < J(\theta_{old})$
		\item The challenge is to find the right step size $\alpha$ and direction $\delta\theta$
		\item Different algorithms differ in the number of required iterations, the computational cost per iteration, the convergence guarantees, the robustness with noisy cost functions and their memory usage
	\end{itemize}
\end{frame}

\begin{frame}{Numerical Optimization}{Optimization algorithms}
	\begin{itemize}
		\item There are various approaches to numerical optimization
		\item Gradient-based methods require differentiable functions and not too many iterations, but only guarantee to find a local optimum, examples are:
		\begin{itemize}
			\item Gradient Descent (with constant, variable or line-search-optimized step size)
			\item (L-)BFGS
			\item Conjugate Gradient Descent
		\end{itemize}
		\item Non-gradient based methods may find a global optimum, but require a large number of steps, examples are:
		\begin{itemize}
			\item Genetic Algorithms
			\item Non-Linear Simplex
			\item Nelder-Mead
		\end{itemize}
	\end{itemize}
\end{frame}

\begin{frame}{Numerical Optimization}{There are many other things you have to consider}
Initialization also matters...
\includegraphics[width=0.75\textwidth]{02_math/02_img/gradientascent.png}
\end{frame}

\begin{frame}{Want to learn more about optimization?}{Every machine learning problem is an optimization problem!}
\begin{itemize}
\item Deep Learning book chapters 4.3, 4.4 and 8 (\href{https://www.deeplearningbook.org/contents/numerical.html}{Link chapters 4.3, 4.4}, \href{https://www.deeplearningbook.org/contents/optimization.html}{Link chapter 8}) are highly recommended
\item Boyd \& Vandenberghe, Convex Optimization (\href{http://web.stanford.edu/~boyd/cvxbook/bv_cvxbook.pdf}{Link})
\item Stanford convex optimization course (\href{https://web.stanford.edu/class/ee364a/lectures.html}{Link})
\item MOOC on constrained optimization (\href{https://www.khanacademy.org/math/multivariable-calculus/applications-of-multivariable-derivatives/lagrange-multipliers-and-constrained-optimization/v/constrained-optimization-introduction}{Link})
\end{itemize}
\end{frame}


% Section: Wrap-Up
%______________________________________________________________________
\section{Wrap-Up}
\makedivider{Wrap-Up}

% Subsection: Summary
% --------------------------------------------------------------------------------------------------------
\subsection{Summary}

% Summary
\begin{frame}{Summary}{}
	\begin{itemize}
		\item
	\end{itemize}
\end{frame}


% Subsection: Self-Test Questions
% --------------------------------------------------------------------------------------------------------
\subsection{Self-Test Questions}

% Self-Test Questions
\begin{frame}{Self-Test Questions}{}\important
	\begin{enumerate}
		\item 
	\end{enumerate}
\end{frame}


% Subsection: Lecture Outlook
% --------------------------------------------------------------------------------------------------------
\subsection{Lecture Outlook}

\begin{frame}{What's next...?}{}
	\makeoverview{3}
\end{frame}


% Subsection: Recommended Literature and further Reading
% --------------------------------------------------------------------------------------------------------
\subsection{Recommended Literature and further Reading}

% Literature
%______________________________________________________________________
\begin{frame}{Recommended Literature and further Reading}{}
	\footnotesize
	\begin{thebibliography}{2}

	\end{thebibliography}
\end{frame}


% Thank you
%______________________________________________________________________
\makethanks

\end{document}