% select theme
\documentclass[t,8pt,aspectratio=169]{beamer}

% ====================================================
% ====================================================
% USEPACKAGES
% ====================================================
% ====================================================

\usepackage[T1]{fontenc}
\usepackage[utf8]{inputenc}
\usepackage[english]{babel}

% tables
\usepackage{tabularx}
\usepackage{colortbl}
\usepackage{multirow}
\usepackage{makecell}

% tikz and colors
\usepackage{tikz}
\usepackage{xcolor}
\usepackage{pgf-pie}
\usepackage{pgfplots}
\usepackage{pgfplotstable}
\usepackage{tikzsymbols}

\usetikzlibrary{calc}
\usetikzlibrary{trees}
\usetikzlibrary{patterns}
\usetikzlibrary{shadings}
\usetikzlibrary{positioning}
\usetikzlibrary{intersections}
\usetikzlibrary{decorations.pathreplacing}

\usetikzlibrary{arrows}
\usetikzlibrary{arrows.meta}

\usetikzlibrary{shapes}
\usetikzlibrary{shapes.arrows}
\usetikzlibrary{shapes.callouts}
\usetikzlibrary{shapes.symbols}
\usetikzlibrary{shapes.geometric}

\usepgfplotslibrary{patchplots}
\usepgfplotslibrary{fillbetween}

% boxes
\usepackage[many]{tcolorbox}

% math packages and fonts
\usepackage{bm}
\usepackage{ccfonts}
\usepackage{eulervm}
\usepackage{amsmath}
\usepackage{amsfonts}
\usepackage{amssymb}
\usepackage{amsthm}
\usepackage{mathtools}
\usepackage{nicefrac}
\usepackage{slashed}
\usepackage{bbold}
\usepackage{array}
\usepackage{cancel}

% algorithms and listings
\usepackage[ruled,vlined,linesnumbered]{algorithm2e}
\usepackage{listings}
\usepackage{setspace}

\tcbuselibrary{listings}
\tcbuselibrary{breakable}
\tcbuselibrary{skins}

% misc
\usepackage{soul}
\usepackage{pifont}
\usepackage{skull}
\usepackage{multicol}
\usepackage{animate}
\usepackage{cleveref}
\usepackage{hyperref}
\usepackage{wasysym}
\usepackage[absolute,overlay]{textpos}
\usepackage[hang,flushmargin]{footmisc}
\usepackage[framemethod=tikz]{mdframed} 				% provides frame for framed figure and framed table in theme_2

% ====================================================
% ====================================================
% COMMON COMMANDS AND DEFINITIONS
% ====================================================
% ====================================================

% define colors
% ====================================================
\definecolor{myblue1}{RGB}{35,119,189}
\definecolor{myblue2}{RGB}{95,179,238}
\definecolor{myblue3}{RGB}{129,168,207}
\definecolor{myblue4}{RGB}{26,89,142}

\definecolor{myred1}{RGB}{247,12,12}

% math definitions
% ====================================================
\DeclareMathOperator*{\argmax}{arg\,max}
\DeclareMathOperator*{\argmin}{arg\,min}
\newcommand*\diff{\mathop{}\!\mathrm{d}}
\newcommand{\indep}{\rotatebox[origin=c]{90}{$\models$}}

\newcommand*{\vertbar}{\rule[-1ex]{0.5pt}{2.5ex}}
\newcommand*{\horzbar}{\rule[.5ex]{2.5ex}{0.5pt}}

% math cancel sign
\newcommand\hcancel[2][black]{\setbox0=\hbox{$#2$}%
	\rlap{\raisebox{.45\ht0}{\textcolor{#1}{\rule{\wd0}{1pt}}}}#2}

% column type for matrices
\newcolumntype{C}[1]{>{\centering\arraybackslash}p{#1}}

% table definitions
% ====================================================
% centered X column in tabularx
\newcolumntype{Y}{>{\centering\arraybackslash}X}

% font commands
% ====================================================
% style of hyperlinks
\newcommand{\linkstyle}[1]{\underline{\smash{\texttt{#1}}}}

% slide architecture
% ====================================================
% divide frame into two parts
\newcommand{\divideTwo}[4]{
	\begin{minipage}{#1\textwidth}
		#2
	\end{minipage}
	\hfill
	\begin{minipage}{#3\textwidth}
		#4
	\end{minipage}
}

% divide frame into two parts (start on top)
\newcommand{\divideTwoTop}[4]{
	\begin{minipage}[t]{#1\textwidth}
		#2
	\end{minipage}
	\hfill
	\begin{minipage}[t]{#3\textwidth}
		#4
	\end{minipage}
}


% ====================================================
% ====================================================
% LAYOUT AND THEME
% ====================================================
% ====================================================

% adjust margin left and right
\setbeamersize{text margin left=25pt,text margin right=25pt}

% define colors
% ====================================================
\setbeamercolor{frametitle}{fg=black}
\setbeamercolor{itemize item}{fg=black}
\setbeamercolor{itemize subitem}{fg=black}
\setbeamercolor{caption name}{fg=black!80!white}
\setbeamercolor{section in toc}{fg=black}
\setbeamercolor{subsection in toc}{fg=gray!20!black}

% define fonts and sizes
% ====================================================
\setbeamerfont{frametitle}{series=\bf,size=\footnotesize}
\setbeamerfont{caption name}{series=\bf}
\setbeamertemplate{itemize/enumerate subbody begin}{\normalsize}
\setbeamertemplate{itemize/enumerate subsubbody begin}{\small}
\usefonttheme[onlymath]{serif}

% table of contents
% ====================================================
\makeatletter
\def\beamer@endinputifotherversion#1{}
\def\beamer@sectionintoc#1#2#3#4#5{{\large \vspace*{3mm} \textbf{#2} \hfill \textbf{#3} \par}}
\def\beamer@subsectionintoc#1#2#3#4#5#6{{\normalsize \hspace*{3mm} \textbf{#3} \hfill \textbf{#4} \par}}
\def\beamer@subsubsectionintoc#1#2#3#4#5#6#7{{\normalsize \hspace*{6mm} \textbf{#4} \hfill #5 \par}}
\makeatother

% define bullet points
% ====================================================
\setbeamertemplate{itemize item}[circle]
\setbeamertemplate{itemize subitem}{--}
\setbeamertemplate{itemize subsubitem}{\textcolor{black}{$\triangleright$}}
\setlength{\leftmargini}{3.5mm}
\setlength{\leftmarginii}{3.5mm}
\setlength{\leftmarginiii}{3.5mm}

% frametitle
% ====================================================
\setbeamertemplate{frametitle}{%
	\vspace*{1mm}
	\ifthenelse{\boolean{deeptoc}}{
		\ifnum\insertframenumber=\insertsectionstartpage%
			\vspace*{0.1mm}
    			\begin{tcolorbox}[
				skin=enhanced,
      				boxrule=0.6mm, boxsep=0mm,
     	   			lowerbox=ignored,
        			colback=orange!60!red, colframe=black,
        			borderline={0.5pt}{3pt}{black}, borderline={1pt}{2pt}{red}
    			]
        			\centering
        			\Huge\textbf\insertsectionhead\par
    			\end{tcolorbox}
		\fi%
	}{}
	\ifnum\insertframenumber=\insertsubsectionstartpage%
    		\vspace*{0.1mm}
    		\begin{tcolorbox}[
     		   	boxrule=0.4mm, boxsep=-0.5mm,
     	   		lowerbox=ignored,
     	   		colback=yellow!60!orange, colframe=black
    		]
        		\centering
        		\huge\textbf\insertsubsectionhead\par
   	 	\end{tcolorbox}
	\fi%
	\begin{tcolorbox}[
    		boxrule=0.2mm,
    		boxsep=0mm,
    		lowerbox=ignored,
    		colback=yellow, colframe=black
	]
    		\centering
    		\large\insertframetitle
	\end{tcolorbox}
	\vspace*{-2mm}
}

% header and Footer
% ====================================================

% header
\setbeamertemplate{headline}{
	% skip header line on first frame of section
	\ifnum\insertsectionstartframe=\insertframenumber%
     		\vskip-\headheight%
	\else
		\begin{beamercolorbox}[wd=\textwidth,ht=4mm,dp=1mm]{}
			\hspace*{25pt} \Fheadline \hspace{25pt}
		\end{beamercolorbox}
		\centerline{\rule{\linewidth}{.2pt}}
	\fi
}

\setbeamertemplate{footline}{%
	\centerline{\rule{\linewidth}{.2pt}}
	\begin{beamercolorbox}[wd=\textwidth,ht=2mm,dp=3mm]{}
		\hspace*{25pt} \Ffootline \hspace{25pt}
	\end{beamercolorbox}
}

% definition of header
\newcommand{\Fheadline}{
	\ifx\insertshorttitle\undefined
		\inserttitle
	\else
		\insertshorttitle
	\fi
	\hfill
	\insertsection
}

% definition of footer
\newcommand{\Ffootline}{
	\textbf{\insertauthor~(\insertshortinstitute),~\insertdate}
	\hfill
	\insertframenumber
}

% customize captions
% ====================================================
\setbeamertemplate{caption}{%
	\begin{tcolorbox}[
		colback=lightgray, colframe=black,
		width=.17\linewidth,
		height=17pt,
		boxrule=0.5pt, boxsep=-1pt
	]
		\textbf{\strut\insertcaptionname~\insertcaptionnumber%
		\usebeamertemplate{caption label separator}}%
	\end{tcolorbox}%
	\space%
	\begin{tcolorbox}[
		colback=lightgray, colframe=black,
		width=.82\linewidth,
		height=17pt,
		boxrule=0.5pt, boxsep=-1pt
	]
		\strut\insertcaption%
	\end{tcolorbox}%
}

% remove navigation symbols
\setbeamertemplate{navigation symbols}{}

% remove header line on first frame of section
% ====================================================
\makeatletter
\newcount\beamer@sectionstartframe
\beamer@sectionstartframe=1
\apptocmd{\beamer@section}{\addtocontents{nav}{\protect\headcommand{%
            \protect\beamer@sectionframes{\the\beamer@sectionstartframe}{\the\c@framenumber}}}
}{}{}
\apptocmd{\beamer@section}%
{\beamer@sectionstartframe=\c@framenumber\advance\beamer@sectionstartframe by1\relax}{}{}
\AtEndDocument{
	\immediate\write\@auxout{\string\@writefile{nav}%
        	{\noexpand\headcommand{\noexpand\beamer@sectionframes{\the\beamer@sectionstartframe}%
	{\the\c@framenumber}}}}
}{}{}

\def\beamer@startframeofsection{1}
\def\beamer@endframeofsection{1}
\def\beamer@sectionframes#1#2{%
    		\ifnum\c@framenumber<#1%
    		\else%
    			\ifnum\c@framenumber>#2%
    			\else%
    				\gdef\beamer@startframeofsection{#1}%
    				\gdef\beamer@endframeofsection{#2}%
    			\fi%
    		\fi%
}

\newcommand\insertsectionstartframe{\beamer@startframeofsection}
\newcommand\insertsectionendframe{\beamer@endframeofsection}
\makeatother

% ====================================================
% ====================================================
% COMMANDS AND GENERAL DEFINITIONS
% ====================================================
% ====================================================

% variable definitions
% ====================================================
% variable deeptoc (deep or flat table of contents)
\newcommand{\dwDeepToc}[1]{
	\newboolean{deeptoc}
	\setboolean{deeptoc}{#1}
}

% adjust font
% ====================================================
\renewcommand*{\familydefault}{\sfdefault}

% framed floatings
% ====================================================
\newmdenv[
	innerlinewidth=0.05pt,
	roundcorner=4pt,
	linecolor=black,
	innerleftmargin=6pt,
	innerrightmargin=6pt,
	innertopmargin=6pt,
	innerbottommargin=6pt
]{mybox}

\newmdenv[
	innerlinewidth=0.05pt,
	roundcorner=4pt,
	linecolor=black,
	innerleftmargin=0pt,
	innerrightmargin=0pt,
	innertopmargin=0pt,
	innerbottommargin=-1pt
]{mytablebox}

% framed figure
\newcommand{\dwFigure}[3]{
	\begin{figure}
		\begin{mybox}
			\centering #1
  		\end{mybox}
  		\vspace{-4mm}
  		\caption{#2}
  		\label{#3}
	\end{figure}
	\vspace{-3mm}
}

% framed table
\newcommand{\dwTable}[4]{
	\begin{table}
		\begin{mytablebox}
			\renewcommand{\arraystretch}{#4} #1
		\end{mytablebox}
		\vspace{-2.5mm}
		\caption{#2}
		\label{#3}
	\end{table}
}

% sections and subsections
% ====================================================
% section command
\newcommand{\dwSection}[1]{
	\ifthenelse{\boolean{deeptoc}}{
		\section{#1}
	}{
		\section{#1}
		\subsection*{#1}
	}
}

% subsection command
\newcommand{\dwSubsection}[1]{\subsection{#1}}

% header
% ====================================================
% green header
\newcommand{\dwHeader}[1]{%
	\begin{tcolorbox}[
		boxrule=0.2mm, boxsep=-1mm,
		lowerbox=ignored,
		colback=green, colframe=black,
		hbox
	]
		\textbf{#1}
	\end{tcolorbox}
}

% alert box / info box
% ====================================================
\newcommand{\dwAlertBox}[1]{
	\vspace*{2mm}\hspace*{0.25mm}
	\begin{minipage}[c]{0.05\textwidth}
		\begin{tikzpicture}[rotate=180, transform shape]\thicki\end{tikzpicture}
	\end{minipage}
	\hfill
	\begin{minipage}[c]{0.92\textwidth}
		\begin{mybox}
			\textcolor{red}{\textbf{#1}}
		\end{mybox}
	\end{minipage}
}

\newcommand{\dwInfoBox}[1]{
	\vspace*{2mm}\hspace*{0.25mm}
	\begin{minipage}[c]{0.05\textwidth}
		\begin{tikzpicture}\thicki\end{tikzpicture}
	\end{minipage}
	\hfill
	\begin{minipage}[c]{0.92\textwidth}
		\begin{mybox}
			\textcolor{blue}{\textbf{#1}}
		\end{mybox}
	\end{minipage}
}

\newcommand{\thicki}{
	\draw[thick,fill=lightgray] (4.2,0) -- (4.5,0.5) -- (4.8,0) -- (4.2,0) -- cycle;
	\fill (4.45,0.05) rectangle (4.55,0.25);
	\fill (4.50,0.34) circle (1.5pt);
}

% custom itemize environment
% ====================================================
\let\tempone\itemize
\let\temptwo\enditemize
\renewenvironment{itemize}{\vspace*{1.5mm}\tempone\addtolength{\itemsep}{0.5\baselineskip}}{\temptwo}
\let\tempthree\enumerate
\let\tempfour\endenumerate
\renewenvironment{enumerate}{\vspace*{1.5mm}\tempthree\addtolength{\itemsep}{0.5\baselineskip}}{\tempfour}

% frames
% ====================================================
\newenvironment{dwHeaderFrame}[1]{
	\subsubsection{#1}
	\begin{frame}{#1}
}{
	\end{frame}
}

% special pages
% ====================================================
% title page
\newcommand{\dwPrintTitle}{
	{\usebackgroundtemplate{%
		\tikz[overlay,remember picture] \node[opacity=0.9, at=(current page.center)] {
  			\includegraphics[height=\paperheight,width=\paperwidth]{../03_img/processor_red.jpg}
		};
	}
	\begin{frame}[plain]
		\begin{center}
			\begin{tcolorbox}[
				skin=enhanced,
	      			boxrule=0.6mm, boxsep=0mm,
				lowerbox=ignored,
				colback=orange!60!red, colframe=black,
				borderline={0.5pt}{3pt}{black}, borderline={1pt}{2pt}{red},
				width=\textwidth
			]
				\centering
				\Huge\textbf{\inserttitle}
			\end{tcolorbox}
			\vspace*{1.4cm}
			\begin{tcolorbox}[width=0.5\textwidth]
				\centering
				\textbf{\insertauthor} \\[2mm]
				\insertinstitute \\[2mm]
				\insertdate
			\end{tcolorbox}
			
			\begin{textblock}{1}(1,13.5)
				\includegraphics[scale=0.04]{../03_img/logo_dhbw}
			\end{textblock}
		\end{center}
	\end{frame}}
}

% table of contents
\newcommand{\dwPrintToc}[1]{
	{\makeatletter
   		\setbeamertemplate{headline}[default]
   		\def\beamer@entrycode{\vspace*{-\headheight}}
	\makeatother

	\begin{frame}[allowframebreaks]
		\begin{tcolorbox}[
			skin=enhanced,
      			boxrule=0.6mm, boxsep=0mm,
			lowerbox=ignored,
			colback=orange!60!red, colframe=black,
			borderline={0.5pt}{3pt}{black}, borderline={1pt}{2pt}{red},
			width=\textwidth
		]
			\centering
			\huge\textbf{Agenda for this Unit}
		\end{tcolorbox}
		\vspace{2mm}
		
		{\renewcommand{\baselinestretch}{1.4}
		\tableofcontents}
	\end{frame}}
}

% thank you page
\newcommand{\makethanks}{
	{\beamertemplatenavigationsymbolsempty
	\begin{frame}[plain]
		\vfill
		\begin{tcolorbox}[
			skin=enhanced,
      			boxrule=0.6mm, boxsep=0mm,
			lowerbox=ignored,
			colback=orange!60!red, colframe=black,
			borderline={0.5pt}{3pt}{black}, borderline={1pt}{2pt}{red},
			width=\textwidth
		]
			\centering
			\Huge \textbf{Thank you very much for the attention!}
		\end{tcolorbox}
		
		\vfill
		\begin{tabbing}
			\hspace*{1.5cm}\= \kill
			\textbf{Topic:} 	\> \inserttitle \\
			\textbf{Term:} 	\> \insertdate
		\end{tabbing}
		
		\vfill
		\textbf{Contact:} \\
		\insertauthor \\
		\insertinstitute \\
		\href{mailto:daniel.wehner@sap.com}{\linkstyle{daniel.wehner@sap.com}}
		
		\vfill
		\begin{center}
			\Large\textbf{Do you have any questions?}
		\end{center}
		\vfill
	\end{frame}}
}

% hyperlinks
% ====================================================
% redefine cref (add a hyperlink)
\let\chyperref\cref % save original command under a new name
\renewcommand{\cref}[1]{\hyperlink{#1}{\textcolor{blue}{$\Rightarrow$ \chyperref{#1}}}}

\newcommand{\externalurl}[2]{\href{#1}{\textcolor{blue}{$\Rightarrow$ #2}}}


% ====================================================
% ====================================================
% OPTIONS
% ====================================================
% ====================================================

% number of levels in toc
\dwDeepToc{false}

% ====================================================
% ====================================================
% PRESENTATION DATA
% ====================================================
% ====================================================

\title[Data Preprocessing]{***** Advanced Machine Learning ***** Data Preprocessing}
\author{M.\,Sc. Daniel Wehner}
\date{Summer term 2020}
\institute{SAP\,SE / DHBW Mannheim}

% ====================================================
% ====================================================
% BEGIN OF DOCUMENT
% ====================================================
% ====================================================

\begin{document}

% Title frame
%______________________________________________________________________
\dwPrintTitle

% Agenda
%______________________________________________________________________
\dwPrintToc

% Section: Introduction
%______________________________________________________________________
\dwSection{Introduction}

% Why Data Preprocessing
\begin{dwHeaderFrame}{Why Data Preprocessing?}
	\begin{itemize}
		\item Data preprocessing is an important step in data mining and machine learning.
		\item \textbf{`Garbage in, garbage out'} holds true for all data mining and machine learning algorithms (you always get back a result, but is it sensible or useful?).
		\item There are lots of problems which impede effective learning:
		\begin{itemize}
			\item Out-of-range values (e.\,g.: \texttt{income} = -100)
			\item impossible data combinations (e.\,g.: \texttt{sex} = male $\wedge$ \texttt{pregnant} = yes)
			\item Missing values
			\item Anomalies and outliers (values which deviate drastically from the other ones)
		\end{itemize}
	\end{itemize}
\end{dwHeaderFrame}


% Data Mining Processes
\begin{frame}
	\begin{itemize}
		\item Several data mining processes were introduced in order to ensure high-quality data:
		\begin{itemize}
			\item KDD (Knowledge Discovery in Databases) process \cref{fig:kdd}
			\item CRISP-DM (Cross Industry Standard Process for Data Mining) \cref{fig:crisp-dm}
		\end{itemize}
		\item Key steps in any data mining process:
		\begin{itemize}
			\item Data cleaning
			\item Data transformation
			\item Data integration
			\item Data reduction
		\end{itemize}
	\end{itemize}

	\dwAlertBox{Proper data preprocessing is necessary to learn effectively from the data!}
\end{frame}


% Section: Data Mining Processes
%______________________________________________________________________
\dwSection{Data Mining Processes}

% Knowledge Discovery in Databases (KDD)
\begin{dwHeaderFrame}{Knowledge Discovery in Databases (KDD)}
	\dwFigure{\begin{figure}
	\centering
	\begin{tikzpicture}[
		scale=0.4,
		every node/.append style={scale=0.7},
		box/.style={draw=black,very thick,minimum width=2cm,minimum height=1cm,align=center,fill=lightgray},
		arr/.style={thick,-stealth,shorten >= 1mm,shorten <= 1mm}
	]
		
		\node[box] (A) at (0,0) {\textbf{Data}};
		\node[box] (B) at (6,0) {\textbf{Target data}};
		\node[box] (C) at (12,0) {\textbf{Cleaned} \\ \textbf{data}};
		\node[box] (D) at (18,0) {\textbf{Transformed} \\ \textbf{data}};
		\node[box] (E) at (24,0) {\textbf{Patterns}};
		\node[box] (F) at (30,0) {\textbf{Knowledge}};

		\draw[arr] (A) -- (B);
		\draw[arr] (B) -- (C);
		\draw[arr] (C) -- (D);
		\draw[arr] (D) -- (E);
		\draw[arr] (E) -- (F);

		\node (a) at (3,-3) {\textit{Selection}};
		\node (b) at (9,-3) {\textit{Preprocessing}};
		\node (c) at (15,-3) {\textit{Transformation}};
		\node (d) at (21,-3) {\textit{Data Mining}};
		\node (e) at (27,-3) {\textit{Evaluation}};

		\draw[dashed] (a) -- ++(0,3);
		\draw[dashed] (b) -- ++(0,3);
		\draw[dashed] (c) -- ++(0,3);
		\draw[dashed] (d) -- ++(0,3);
		\draw[dashed] (e) -- ++(0,3);
		
	\end{tikzpicture}
\end{figure}}{KDD process}{fig:kdd}
	\dwAlertBox{The terms are not used consistently throughout the literature.}
\end{dwHeaderFrame}


% Cross Industry Standard Process for Data Mining (CRISP-DM)
\begin{dwHeaderFrame}{Cross Industry Standard Process for Data Mining (CRISP-DM)}
	\dwFigure{\begin{figure}
	\centering
	\begin{tikzpicture}[
		scale=0.3,
		every node/.append style={scale=0.7},
		box/.style={draw=black,very thick,minimum width=2cm,minimum height=1cm,align=center,fill=lightgray},
		arr/.style={thick,-stealth,shorten >= 0.5mm,shorten <= 0.5mm},
		triangle/.style={regular polygon, regular polygon sides=3,fill=black}
	]
		
		\node[box] (A) at (-4,6) {\textbf{Business} \\ \textbf{Understanding}};
		\node[box] (B) at (4,6) {\textbf{Data} \\ \textbf{Understanding}};
		\node[box] (C) at (8,2) {\textbf{Data} \\ \textbf{Preparation}};
		\node[box] (D) at (8,-2) {\textbf{Modeling}};
		\node[box] (E) at (0,-6) {\textbf{Evaluation}};
		\node[box] (F) at (-8,0) {\textbf{Deployment}};

		\node[draw=black,thick,star, star points=6,fill=yellow] (G) at (0,0) {\textbf{Data}};
		\draw[thick,dotted,rounded corners=50pt] (-12,-8) rectangle (12,8);
		\node[triangle] at (-12,0) {};
		\node[triangle,rotate=180] at (12,0) {};

		\draw[arr,stealth-stealth] (A) to (B);
		\draw[arr] (B) to[out=0,in=90] (C);
		\draw[arr,stealth-stealth] (C) to (D);
		\draw[arr] (D) to[out=270,in=0] (E);
		\draw[arr] (E) to[out=180,in=270] (F);
		\draw[arr] (E) to[out=10,in=350] (A);

		
		
	\end{tikzpicture}
\end{figure}}{CRISP-DM process}{fig:crisp-dm}
\end{dwHeaderFrame}


\begin{frame}
	\dwHeader{Phases of CRISP-DM}
	\begin{itemize}
		\item \textbf{Business understanding}
		\begin{itemize}
			\item Determine what you want to accomplish from a business perspective. \\
				(\textit{What goals do we want to achieve?}, \textit{Why is the project necessary?})
			\item Assess the current business situation w.\,r.\,t. risks, resources, constraints, assumptions, etc.
			\item Results: Project plan, business success criteria (\textit{how to measure success?})
		\end{itemize} 
		\item \textbf{Data understanding:}
		\begin{itemize}
			\item Acquire the data needed to achieve the goals specified in the project plan.
			\item Use tools for data exploration (\textit{e.\,g. compute distributions of key attributes, perform simple aggregations and statistical analyses}).
			\item Results: Data description report and data quality report.
		\end{itemize}
		\item \textbf{Data preparation:}
		\begin{itemize}
			\item Integrate, select and clean the data based on the data description report / data quality report.
			\item Construct new features if needed (\textbf{feature engineering}).
		\end{itemize}
	\end{itemize}
\end{frame}


\begin{frame}
	\begin{itemize}
		\item \textbf{Modeling}
		\begin{itemize}
			\item Choose a machine learning / data mining technique and find good hyper-parameters.
			\item Train the model and test it on a separate test set.
		\end{itemize}
		\item \textbf{Evaluation}
		\begin{itemize}
			\item Evaluate the model(s) w.\,r.\,t. the business objectives. (\textit{In how far does it meet the business goals?})
			\item Review the entire process. (\textit{e.\,g. highlight activities that have been missed or should be repeated})
		\end{itemize}
		\item \textbf{Deployment}
		\begin{itemize}
			\item Deploy the model into a productive environment.
			\item Determine the maintenance strategy and monitor the model.
		\end{itemize}
	\end{itemize}
\end{frame}


% Section: Data Preparation
%______________________________________________________________________
\dwSection{Data Preparation}

% Data Cleaning
\begin{dwHeaderFrame}{Data Cleaning}
	\begin{itemize}
		\item Bad data quality can (and will most probably) lead to impoverished downstream task results.
		\item Therefore, it is necessary to remove erroneous data, inconsistencies and outliers.
		\item The detection of such anomalies often requires a great extent of domain knowledge and is therefore not easy.
	\end{itemize}
	\dwFigure{\begin{figure}
	\centering
	\begin{tikzpicture}[
		scale=0.7,
		every node/.append style={scale=0.7},
		db/.style={cylinder,draw=black,thick,aspect=0.7,minimum height=1.2cm,minimum width=1.5cm,shape border rotate=90,cylinder uses custom fill},
		bubble/.style={ball color=green!20,draw=black,thick},
		arr/.style={thick,-stealth}
	]
		
		\node[db,cylinder body fill=green!30,cylinder end fill=green!10] (A) at (0,0){};

		%  Bubbles
		\shade[bubble] (0.0,1.4) circle[radius=0.2cm];	\shade[bubble] (-0.3,1.2) circle[radius=0.09cm];
		\shade[bubble] (0.2,1.34) circle[radius=0.15cm];	\shade[bubble] (0.11,0.95) circle[radius=0.2cm];
		\shade[bubble] (-0.5,0.8) circle[radius=0.18cm];	\shade[bubble] (0.3,1.6) circle[radius=0.1cm];
		\shade[bubble] (0.5,0.9) circle[radius=0.1cm];	\shade[bubble] (-0.6,1.5) circle[radius=0.06cm];
		\shade[bubble] (-0.1,0.55) circle[radius=0.09cm];	\shade[bubble] (0.7,0.8) circle[radius=0.15cm];
		
		\fill[draw=white,inner color=yellow,outer color=white] (5,0.1) circle (1.2);
		\node[db] (B) at (5,0){};
		\draw[arr] (1,0) -- (4,0);

	\end{tikzpicture}
\end{figure}}{Data cleaning}{fig:data-cleaning}
\end{dwHeaderFrame}


\begin{frame}
	\begin{itemize}
		\item Another problem to be handled is given by missing data (e.\,g. some feature values are not known for some of the data examples).
		\item Possible strategies include:
		\begin{itemize}
			\item Deletion of the affected data example
			\item Imputation of the missing value(s), e.\,g.:
			\begin{itemize}
				\item Further data collection.
				\item Use the mean / median / mode as a substitute (\textbf{What is the difference between these three?})
				\item Fill in the most probable value (learn a model, e.\,g. decision trees to impute the missing value)
			\end{itemize}
			\item Replace unknown values by a global placeholder, e.\,g. \textit{`unknown'} or \textit{`?'}
		\end{itemize}
	\end{itemize}
	\dwAlertBox{Which technique is used depends on the number of missing values.}
\end{frame}


% Data Transformation
\begin{dwHeaderFrame}{Data Transformation}
	\begin{itemize}
		\item Most algorithms require the data to be in a certain form.
		\item If the form of the data is not as required, it has to be transformed accordingly:
		\begin{itemize}
			\item Data smoothing (\textit{removal of noise and peaks in the data})
			\item Aggregation (\textit{e.\,g. computation of sum or average values})
			\item Normalization (\textit{force the data to be in a certain range})
			\item Discretization (\textit{numeric data} $\rightarrow$ \textit{discrete data})
			\item Numerization (\textit{discrete data} $\rightarrow$ \textit{numeric data})
		\end{itemize}
		\item We will have a closer look at normalization and discretization.
	\end{itemize}
\end{dwHeaderFrame}


\begin{frame}
	\dwHeader{Normalization}
	\begin{itemize}
		\item Observation: Features which can take large values dominate features with a small range of values.
		\item Possible transformations:
		\begin{itemize}
			\item \textbf{Min-max normalization} (left: resulting range [0, 1], right: resulting range [a, b]):
			\begin{equation}
				z  = \frac{x - x_{min}}{x_{max} - x_{min}} \qquad\qquad z = a + \frac{(x - x_{min}) \cdot (b - a)}{x_{max} - x_{min}}
			\end{equation}
			\item \textbf{Mean normalization}:
			\begin{equation}
				z = \frac{x - \overline{x}}{x_{max} - x_{min}}
			\end{equation}
			\item \textbf{Standardization} ($\overline{x} = 0$ and $\sigma = 1$):
			\begin{equation}
				z = \frac{x - \overline{x}}{\sigma}
			\end{equation}
		\end{itemize}
		\item Scaling can be harmful, if the data contains many outliers. Libraries like \texttt{scikit-learn} also offer robust scaling
			which can be used in such cases.
	\end{itemize}
\end{frame}


\begin{frame}
	\dwHeader{Unsupervised Discretization}
	\begin{itemize}
		\item \textbf{Domain dependent}
		\begin{itemize}
			\item Suitable discretizations are often known.
			\item E.\,g. \texttt{age} [0--18] $\longrightarrow$ baby [0--3], child (3--6], school child (6--10], teenager (10--18] 
		\end{itemize}
		\item \textbf{Equal-width}
		\begin{itemize}
			\item Divide the range into a number of intervals with equal width.
			\item E.\,g. \texttt{age} [0--18] $\longrightarrow$ [0--3], [4--7], [8--11], [12--15], [16--18] 
		\end{itemize}
		\item \textbf{Equal-frequency}
		\begin{itemize}
			\item Create the intervals such that they comprise roughly the same number of data points.
			\item E.\,g. if the number of bins is set to 5, each will comprise 20\,\% of the data.
		\end{itemize}
	\end{itemize}
\end{frame}


\begin{frame}
	\dwHeader{Supervised Discretization -- $\chi$-Merge}
	\begin{itemize}
		\item \textbf{Initialization:}
		\begin{enumerate}
			\item Sort the examples by feature value.
			\item Construct one interval for each value.
		\end{enumerate}
		\item \textbf{Interval merging:}
		\begin{enumerate}
			\item Compute the $\chi^2$ value for each pair of adjacent intervals:
			\begin{align}
				\chi^2
					&= \sum_{j=1}^2 \sum_{k=1}^{\kappa} \frac{(a_{jk} - e_{jk})^2}{e_{jk}} \qquad\text{where}\qquad e_{jk} = n_j \cdot \frac{a_{1k} + a_{2k}}{n_1 + n_2} \\
				\shortintertext{Legend:}
				\nonumber
				a_{jk}
					&\equiv \text{number of examples in $j$-th interval which have class $k$} \\
				\nonumber
				e_{jk}
					&\equiv \text{expected number of examples in $j$-th interval which have class $k$} \\
				\nonumber
				n_j
					&\equiv \text{total number of examples in $j$-th interval}
			\end{align}
			\item Merge the intervals with the lowest $\chi^2$ value
		\end{enumerate}
		\item \textbf{Stopping criterion:} $\chi^2$ values of all pairs exceed a significance threshold.
	\end{itemize}
\end{frame}


% Data Reduction
\begin{dwHeaderFrame}{Data Reduction}
	\begin{itemize}
		\item Databases are typically not collected with data mining / machine learning in mind.
		\item Many features may be:
		\begin{itemize}
			\item irrelevant
			\item uninteresting
			\item redundant
		\end{itemize}
		\item Removing such features might increase efficiency, improve accuracy and prevent overfitting.
		\item \textbf{Feature subset selection (FSS)} techniques try to determine appropriate features automatically.
	\end{itemize}

	\dwInfoBox{Principal component analysis (PCA) can also be used to reduce the data. Since the algorithm was already covered, it is not presented here.}
\end{dwHeaderFrame}


\begin{frame}
	\dwHeader{Unsupervised FSS}
	\begin{itemize}
		\item Use \textbf{domain knowledge}: An expert may know in advance that some features are irrelevant uninteresting or redundant.
		\item \textbf{Random sampling}
		\begin{itemize}
			\item Select a random subset of the features.
			\item Such an approach may be appropriate in the case of many weakly correlated features or in conjunction with ensemble methods (\textit{remember random forests?}).
		\end{itemize}
	\end{itemize}
\end{frame}


\begin{frame}
	\dwHeader{Supervised FSS}
	\begin{itemize}
		\item \textbf{Filter approaches:}
		\begin{itemize}
			\item Such techniques attempt to estimate the features' capabilities to discriminate between the classes.
			\item The most discriminatory features are ultimately selected.
			\item \textbf{Problems:}
			\begin{itemize}
				\item Redundant features will receive similar weights.
				\item Some features may only be important in combination with other features (e.\,g. \texttt{XOR}-problem).
			\end{itemize}
		\end{itemize}
		\item \textbf{Wrapper approaches:}
		\begin{itemize}
			\item Search through the space of all possible feature subsets.
			\item Each feature subset is tried in combination with the learning algorithm.
			\item The subset which performs best is kept.
		\end{itemize}
	\end{itemize}
\end{frame}


\begin{frame}
	\dwHeader{RELIEF algorithm (filter approach)}
	\begin{itemize}
		\item 
	\end{itemize}
\end{frame}


\begin{frame}
	\dwHeader{Wrapper approaches}
	\dwFigure{\begin{figure}
	\centering
	\begin{tikzpicture}[
		scale=0.3,
		every node/.append style={scale=0.7},
		box/.style={draw=black,very thick,minimum width=5cm,minimum height=1cm,align=center,fill=lightgray!60,anchor=east},
		arr/.style={thick,-stealth,shorten >= 0.5mm,shorten <= 0.5mm},
	]

		% wrapper
		\draw[draw=black,very thick,fill=lightgray] (-12.75,-10) rectangle (1,2);
		\draw[arr] (-17.75,0) -- node[above] {training data} ++(5,0);
		
		\draw[arr] (-9,0) -- node[left] {feature set} ++(0,-6.75);
		\draw[arr] (-3,-7) -- node[right] {hypothesis} ++(0,1.9);
		\draw[arr] (-3,-3) -- node[right] {perf. est.} ++(0,1.9);
		
		\node[box] (A) at (0,0) {\textbf{Feature Selection Search}};
		\node[box,minimum width=2.5cm] (B) at (0,-4) {\textbf{Feature} \\ \textbf{Evaluation}};
		\node[box] (C) at (0,-8) {\textbf{Induction Algorithm}};


		% induction algorithm
		\node[box] at (18,0) {\textbf{Induction Algorithm}};
		\draw[arr] (1,0.75) -- node[above] {training data} ++(5,0);
		\draw[arr] (1,-0.75) -- node[below] {feature set} ++(5,0);

		% final evaluation
		\node[box] at (18,-14) {\textbf{Final Evaluation}};
		\draw[arr] (12,-1) -- ++(0,-11.75);
		\draw[arr] (-17.75,-14) -- node[above] {test data} ++(23.75,0);

		\draw[arr] (17.8,-14) -- node[above] {accuracy} ++(5,0);
		
	\end{tikzpicture}
\end{figure}}{Wrapper approach for feature subset selection}{fig:wrapper}
\end{frame}


\begin{frame}
	\begin{itemize}
		\item \textbf{Forward selection:}
		\begin{enumerate}
			\item Start with an empty feature set $\mathcal{F}$
			\item For each attribute $A$ estimate accuracy of learning algorithm on $\mathcal{F} \cup \{ A \}$
			\item $\mathcal{F} \longleftarrow \mathcal{F} \cup \{ \text{attribute with highest accuracy} \}$
			\item go to 2 (until $m$ features have been found
		\end{enumerate}
		\item \textbf{Backward selection:}
		\begin{itemize}
			\item Start with a full feature set $\mathcal{F}$
			\item Subsequently remove attributes from $\mathcal{F}$
		\end{itemize}
	\end{itemize}
\end{frame}


% Data Integration
\begin{dwHeaderFrame}{Data Integration}
	\begin{itemize}
		\item
	\end{itemize}
\end{dwHeaderFrame}


% Thank you
%______________________________________________________________________
\makethanks

\end{document}
