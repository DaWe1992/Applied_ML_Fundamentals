\documentclass[12pt,aspectratio=169]{beamer}

% ====================================================
% ====================================================
% USEPACKAGES AND IMPORTS
% ====================================================
% ====================================================

\usepackage[T1]{fontenc}
\usepackage[utf8]{inputenc}
\usepackage[english]{babel}

% tables
\usepackage{tabularx}
\usepackage{colortbl}
\usepackage{multirow}
\usepackage{makecell}

% tikz and colors
\usepackage{tikz}
\usepackage{xcolor}
\usepackage{pgf-pie}
\usepackage{pgfplots}
\usepackage{pgfplotstable}
\usepackage{tikzsymbols}

\usetikzlibrary{calc}
\usetikzlibrary{trees}
\usetikzlibrary{patterns}
\usetikzlibrary{shadings}
\usetikzlibrary{positioning}
\usetikzlibrary{intersections}
\usepgfplotslibrary{patchplots}
\usepgfplotslibrary{fillbetween}
\usetikzlibrary{decorations.pathreplacing}

\usetikzlibrary{arrows}
\usetikzlibrary{arrows.meta}

\usetikzlibrary{shapes}
\usetikzlibrary{shapes.arrows}
\usetikzlibrary{shapes.callouts}
\usetikzlibrary{shapes.symbols}
\usetikzlibrary{shapes.geometric}

\usetikzlibrary{intersections}

% boxes
\usepackage[many]{tcolorbox}

% math packages and fonts
\usepackage{bm}
\usepackage{ccfonts}
\usepackage{eulervm}
\usepackage{amsmath}
\usepackage{amsfonts}
\usepackage{amssymb}
\usepackage{amsthm}
\usepackage{mathtools}
\usepackage{nicefrac}
\usepackage{slashed}
\usepackage{bbold}
\usepackage{array}
\usepackage{cancel}

% algorithms and listings
\usepackage[ruled,vlined,linesnumbered]{algorithm2e}
\usepackage{listings}
\usepackage{setspace}

\tcbuselibrary{listings}
\tcbuselibrary{breakable}
\tcbuselibrary{skins}

% misc
\usepackage{soul}
\usepackage{pifont}
\usepackage{skull}
\usepackage{multicol}
\usepackage{animate}
\usepackage{hyperref}
\usepackage{wasysym}
\usepackage[absolute,overlay]{textpos}
\usepackage[hang,flushmargin]{footmisc}

% ====================================================
% ====================================================
% LAYOUT AND THEME
% ====================================================
% ====================================================

\usetheme{Copenhagen}

% color definitions
\definecolor{myblue1}{RGB}{35,119,189}
\definecolor{myblue2}{RGB}{95,179,238}
\definecolor{myblue3}{RGB}{129,168,207}
\definecolor{myblue4}{RGB}{26,89,142}

\definecolor{myred1}{RGB}{247,12,12}

% set theme colors
\setbeamercolor*{structure}{fg=myblue1,bg=blue}
\setbeamercolor*{palette primary}{use=structure,fg=white,bg=structure.fg}
\setbeamercolor*{palette secondary}{use=structure,fg=white,bg=structure.fg!75!black}
\setbeamercolor*{palette tertiary}{use=structure,fg=white,bg=structure.fg!50!black}
\setbeamercolor*{palette quaternary}{fg=black,bg=white}

\setbeamertemplate{itemize item}[circle]
\setbeamertemplate{itemize subitem}[circle]
\setbeamertemplate{itemize subsubitem}[circle]

\setbeamertemplate{enumerate item}[circle]
\setbeamertemplate{enumerate subitem}[circle]
\setbeamertemplate{enumerate subsubitem}[circle]

\setbeamercolor{itemize item}{fg=myblue1}
\setbeamercolor{itemize subitem}{fg=myblue1}
\setbeamercolor{itemize subsubitem}{fg=myblue1}

\setbeamertemplate{section in toc}[circle]
\setbeamertemplate{subsection in toc}[circle]
\setbeamerfont{subsection in toc}{size=\scriptsize}

\setbeamercolor{frametitle continuation}{fg=black}

% title graphic -- sap logo and dhbw logo
\titlegraphic{\includegraphics[scale=0.1]{../03_img/logo_sap}\hspace*{4.75cm}~%
   	\includegraphics[scale=0.05]{../03_img/logo_dhbw}
}

\makeatletter
% frame title
\defbeamertemplate*{frametitle}{mydefault}[1][left]
{
  	\ifbeamercolorempty[bg]{frametitle}{}{\nointerlineskip}%
  	\nointerlineskip%
 	\@tempdima=\textwidth%
  	\advance\@tempdima by\beamer@leftmargin%
  	\advance\@tempdima by\beamer@rightmargin%
  	\begin{tcolorbox}[
  		enhanced,
  		outer arc=0pt,
  		arc=0pt,
  		boxrule=0pt,
  		top=0pt,
  		bottom=0pt,
  		enlarge left by=-\beamer@leftmargin,
  		enlarge right by=-\beamer@rightmargin,
  		width=\paperwidth,
  		nobeforeafter,
  		interior style={
    			left color=myblue2,
    			right color=white
    		},
  		shadow={0mm}{-0.4mm}{0mm}{black!60,opacity=0.6},    
  		shadow={0mm}{-0.8mm}{0mm}{black!40,opacity=0.4},    
  	]
    	\usebeamerfont{frametitle}%
    	\vbox{}\vskip-1ex%
    	\if@tempswa\else\csname beamer@fte#1\endcsname\fi%
    	\insertframetitle\par%
    	{%
      		\ifx\insertframesubtitle\@empty%
      		\else%
      		{\usebeamerfont{framesubtitle}\usebeamercolor[fg]{black}\insertframesubtitle\strut\par}%
      		\fi
    	}%
    	\vskip-1ex%
    	\if@tempswa\else\vskip-.3cm\fi
  	\end{tcolorbox}%
}

% footline of a frame
\defbeamertemplate*{footline}{mysplit theme}
{%
  	\leavevmode%
  	\hbox{
		\begin{beamercolorbox}[
			wd=.5\paperwidth,ht=2.5ex,dp=1.125ex,leftskip=.3cm plus1fill,rightskip=.3cm
		]{author in head/foot}%
    			\usebeamerfont{author in head/foot}\insertshortauthor\ (\insertinstitute), \insertdate
  		\end{beamercolorbox}%
  		\begin{beamercolorbox}[
			wd=.5\paperwidth,ht=2.5ex,dp=1.125ex,leftskip=.3cm,rightskip=.3cm plus1fil
		]{title in head/foot}%
    			\usebeamerfont{title in head/foot}\insertshorttitle\hfill
    			\insertprefix-\insertframenumber/\inserttotalframenumber\hspace*{0.5em}
  		\end{beamercolorbox}}%
  	\vskip0pt%
}
\makeatother

% ====================================================
% ====================================================
% COMMANDS AND GENERAL DEFINITIONS
% ====================================================
% ====================================================

% page number prefix
\newcommand\insertprefix{}  % empty by default
\newcommand\prefix[1]{\renewcommand\insertprefix{#1}}

% math definitions
% ====================================================
\DeclareMathOperator*{\argmax}{arg\,max}
\DeclareMathOperator*{\argmin}{arg\,min}
\newcommand*\diff{\mathop{}\!\mathrm{d}}
\newcommand{\indep}{\rotatebox[origin=c]{90}{$\models$}}

\newcommand*{\vertbar}{\rule[-1ex]{0.5pt}{2.5ex}}
\newcommand*{\horzbar}{\rule[.5ex]{2.5ex}{0.5pt}}

% math cancel sign
\newcommand\hcancel[2][black]{\setbox0=\hbox{$#2$}%
	\rlap{\raisebox{.45\ht0}{\textcolor{#1}{\rule{\wd0}{1pt}}}}#2}

% column type for matrices
\newcolumntype{C}[1]{>{\centering\arraybackslash}p{#1}}

% commands
% ====================================================

% highlight commands
% --------------------------------------------------------------------------------------------------------
% highlight command
\newcommand{\highlight}[1]{\textcolor{myblue1}{\textbf{#1}}}
\newcommand{\highlighttt}[1]{\textcolor{myblue1}{\texttt{#1}}}
\newcommand{\Highlight}[1]{\textcolor{myred1}{\textbf{#1}}}

% blue color boxes (with frame/without frame/without fill)
\newtcolorbox{boxBlue}{colback=myblue1!10!white,colframe=myblue4}
\newtcolorbox{boxBlueNoFrame}{colback=myblue1!10!white,colframe=myblue1!10!white}
\newtcolorbox{boxBlueNoFill}{colback=white,colframe=myblue4}

% font commands
% --------------------------------------------------------------------------------------------------------
\newcommand{\linkstyle}[1]{\underline{\smash{\texttt{#1}}}} 		% style of hyperlinks

% tikz commands
% --------------------------------------------------------------------------------------------------------

% yellow sticky note
\newcommand{\bubble}[3]{
\begin{textblock}{100}(#1, #2)
      	\begin{tikzpicture}
		\node[rectangle,draw=yellow,very thick,fill=yellow!60,align=center] at (0,0) {#3};
	\end{tikzpicture}
\end{textblock}
}

\newcommand{\floattext}[3]{
\begin{textblock}{100}(#1, #2)
      	#3
\end{textblock}
}

\newcommand{\doublecircle}[2]{
	\draw[fill=white,draw=myblue1] (#1,#2) circle (2mm);
	\draw[fill=myblue1,draw=myblue1] (#1,#2) circle (1.5mm);
}

% intersection of lines
\newcommand{\intersect}[5]{
	\draw[#5] let \p{c} = (intersection of #1 -- #2 and #3 -- #4) in (#1) -- ($(\p{c})!1.75mm!(#1)$) 
  		to[bend right=90] ($(\p{c})!1.75mm!(#2)$) -- (#2)
}

% circled numbers
\newcommand*\circled[1]{\tikz[baseline=(char.base)]{\node[shape=circle,draw,inner sep=2pt] (char) {#1};}}


% slide modifiers
% --------------------------------------------------------------------------------------------------------
% mark slide as optional
\newcommand{\optional}{
	\begin{textblock}{100}(0.15,0.30)
      		\includegraphics[scale=0.2]{../03_img/scream}
    	\end{textblock}
}

% mark slide as important
\newcommand{\important}{
	\begin{textblock}{100}(0.10,0.15)
      		\includegraphics[scale=0.1]{../03_img/important}
    	\end{textblock}
}

% citation
% --------------------------------------------------------------------------------------------------------
% first argument in {book, online, article}
\newcommand{\literature}[5]{
	\setbeamertemplate{bibliography item}[#1]
	\bibitem{#2}
	\highlight{#3} \\
	\textcolor{darkgray}{\textit{#4}} \\
	\textcolor{black}{#5}
}
% cite content
\newcommand{\citeAuthor}[3]{\vfill\scriptsize\textcolor{gray}{#1 \cite{#2}, #3}}

% slide architecture
% --------------------------------------------------------------------------------------------------------
% divide frame into two parts
\newcommand{\divideTwo}[4]{
	\begin{minipage}{#1\textwidth}
		#2
	\end{minipage}
	\hfill
	\begin{minipage}{#3\textwidth}
		#4
	\end{minipage}
}

% divide frame into two parts (start on top)
\newcommand{\divideTwoTop}[4]{
	\begin{minipage}[t]{#1\textwidth}
		#2
	\end{minipage}
	\hfill
	\begin{minipage}[t]{#3\textwidth}
		#4
	\end{minipage}
}

% special pages
% --------------------------------------------------------------------------------------------------------
% title page
\newcommand{\maketitlepage}{
	{
		\beamertemplatenavigationsymbolsempty
		\usebackgroundtemplate{%
			\tikz[overlay,remember picture] \node[opacity=0.2, at=(current page.center)] {
  				\includegraphics[height=\paperheight,width=\paperwidth]{../03_img/processor.jpg}
			};
		}
		\begin{frame}[plain]
			\vspace*{0.75cm}
			\maketitle
			\vfill
			\begin{center}
				\footnotesize Find all slides on \href{https://github.com/DaWe1992/Applied_ML_Fundamentals}{\linkstyle{GitHub}}
			\end{center}
		\end{frame}
	}
}

% divider page
\newcommand{\makedivider}[1]{
	{
		\beamertemplatenavigationsymbolsempty
		\usebackgroundtemplate{%
			\tikz[overlay,remember picture] \node[opacity=0.2, at=(current page.center)] {
  				\includegraphics[height=\paperheight,width=\paperwidth]{../03_img/processor.jpg}
			};
		}
		\begin{frame}[plain]
			\vfill
			\begin{boxBlue}
				\centering
				\textbf{Section:} \\
				\large \highlight{#1}
			\end{boxBlue}
			\vfill
			\centering
			\includegraphics[scale=0.05]{../03_img/logo_dhbw.png}
			\vfill
		\end{frame}
	}
}

% overview page
\newcommand{\makeoverview}[1]{
	\begin{tabbing}
		\hspace*{3.5cm}\= \kill
		\ifnum #1=1 \highlight{\textbf{Unit I}} \else \textbf{Unit I} \fi
		\> \ifnum #1=1 \highlight{Machine Learning Introduction} \else Machine Learning Introduction \fi \\
		\ifnum #1=2 \highlight{\textbf{Unit II}} \else \textbf{Unit II} \fi
		\> \ifnum #1=2 \highlight{Mathematical Foundations} \else Mathematical Foundations \fi \\
		\ifnum #1=3 \highlight{\textbf{Unit III}} \else \textbf{Unit III} \fi
		\> \ifnum #1=3 \highlight{Bayesian Decision Theory} \else Bayesian Decision Theory \fi \\
		\ifnum #1=4 \highlight{\textbf{Unit IV}} \else \textbf{Unit IV} \fi
		\> \ifnum #1=4 \highlight{Probability Density Estimation} \else Probability Density Estimation \fi \\
		\ifnum #1=5 \highlight{\textbf{Unit V}} \else \textbf{Unit V} \fi
		\> \ifnum #1=5 \highlight{Regression} \else Regression \fi \\
		\ifnum #1=6 \highlight{\textbf{Unit VI}} \else \textbf{Unit VI} \fi
		\> \ifnum #1=6 \highlight{Classification I} \else Classification I \fi \\
		\ifnum #1=7 \highlight{\textbf{Unit VII}} \else \textbf{Unit VII} \fi
		\> \ifnum #1=7 \highlight{Evaluation} \else Evaluation \fi \\
		\ifnum #1=8 \highlight{\textbf{Unit VIII}} \else \textbf{Unit VIII} \fi
		\> \ifnum #1=8 \highlight{Classification II} \else Classification II \fi \\
		\ifnum #1=9 \highlight{\textbf{Unit IX}} \else \textbf{Unit IX} \fi
		\> \ifnum #1=9 \highlight{Clustering} \else Clustering \fi \\
		\ifnum #1=10 \highlight{\textbf{Unit X}} \else \textbf{Unit X} \fi
		\> \ifnum #1=10 \highlight{Dimensionality Reduction} \else Dimensionality Reduction \fi \\
	\end{tabbing}
}

% thank you page
\newcommand{\makethanks}{
	{\beamertemplatenavigationsymbolsempty
	\begin{frame}[plain]
		\vfill
		\begin{boxBlue}
			\centering
			\Large \highlight{Thank you very much for the attention!}
		\end{boxBlue}
		
		\vfill\footnotesize
		\begin{tabbing}
			\hspace*{1.5cm}\= \kill
			\highlight{Topic:} 	\> \inserttitle \\
			\highlight{Term:} 	\> \insertdate
		\end{tabbing}
		
		\vfill
		\highlight{Contact:} \\
		\insertauthor \\
		\insertinstitute \\
		\href{mailto:daniel.wehner@sap.com}{\linkstyle{daniel.wehner@sap.com}}
		
		\vfill\normalsize
		\begin{center}
			\large\highlight{Do you have any questions?}
		\end{center}
		\vfill
	\end{frame}}
}

% global pfgplots settings
% --------------------------------------------------------------------------------------------------------
\pgfplotsset{
	% allow filtering of data for pgfplots
	discard if/.style 2 args={
        		x filter/.code={
            		\edef\tempa{\thisrow{#1}}
            		\edef\tempb{#2}
            		\ifx\tempa\tempb
                		\def\pgfmathresult{inf}
            		\fi
        		}
    	},
    	discard if not/.style 2 args={
        		x filter/.code={
            		\edef\tempa{\thisrow{#1}}
            		\edef\tempb{#2}
            		\ifx\tempa\tempb
            		\else
                		\def\pgfmathresult{inf}
            		\fi
        		}
    	}
}
