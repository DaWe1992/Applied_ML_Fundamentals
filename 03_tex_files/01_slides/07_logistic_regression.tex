\documentclass[11pt]{article}

% ====================================================
% ====================================================
% USEPACKAGES AND IMPORTS
% ====================================================
% ====================================================

\usepackage[T1]{fontenc}
\usepackage[utf8]{inputenc}
\usepackage[english]{babel}

\usepackage{fancyhdr}

% definitions
% ====================================================
\let\titleoriginal\title
\renewcommand{\title}[1]{
	\titleoriginal{#1}
	\newcommand{\thetitle}{#1}
}

\setlength{\parskip}{\baselineskip}%
\setlength{\parindent}{0pt}%

% header and footer
\pagestyle{fancy}
\fancyhf{}
\lhead{Applied Machine Learning Fundamentals}
\rhead{\thetitle}
\cfoot{\thepage}

% ====================================================
% ====================================================
% PRESENTATION DATA
% ====================================================
% ====================================================

\title[Principal Component Analysis]{*** Applied Machine Learning Fundamentals *** Logistic Regression}
\institute{SAP\,SE}
\author{Daniel Wehner}
\date{\today}
\prefix{LR}

% ====================================================
% ====================================================
% BEGIN OF DOCUMENT
% ====================================================
% ====================================================

\begin{document}

% Title frame
%______________________________________________________________________
\maketitlepage


% Agenda
%______________________________________________________________________
\begin{frame}{Agenda \today}
	\begin{multicols}{2}
		\tableofcontents
	\end{multicols}
\end{frame}


% Section: Introduction
%______________________________________________________________________
\section{Introduction}
\makedivider{Introduction}

% Subsection: What is logistic Regression?
% --------------------------------------------------------------------------------------------------------
\subsection{What is logistic Regression?}

% What is logistic Regression?
\begin{frame}{What is logistic Regression?}{}
	\begin{itemize}
		\item Learning algorithm for \textbf{classification} {\footnotesize \textit{(despite the name...)}}
		\item In its standard form it's applicable to \textbf{binary classification problems only}, but you can use techniques like:
		\begin{itemize}
			\item \highlight{One-vs-One (OVO)}
			\item \highlight{One-vs-Rest (OVR)}
		\end{itemize}
		\item \textbf{Class labels:}
		\begin{itemize}
			\item The `positive class' is usually encoded as \textbf{1} / $\oplus$
			\item The `negative class' as \textbf{0} / $\ominus$
		\end{itemize}
		\item \textbf{Probabilistic interpretation:} The output of the algorithm is between 0 and 1
			{\footnotesize \textit{(probability of the instance belonging to the positive class)}}
	\end{itemize}
\end{frame}


% Subsection: Why you should not use linear Regression
% --------------------------------------------------------------------------------------------------------
\subsection{Why you should not use linear Regression}

% Why you should not use linear Regression...
\begin{frame}{Why you should not use linear Regression...}{}\important
	\begin{figure}
	\centering
	\begin{tikzpicture}[
		scale=0.75
	]

		\draw[->,thick] (0,0) -- node[below] {Tumor Size} (15,0);
		\draw[->,thick] (0,0) -- (0,4) node[above] {Malignant?};

		\draw[dashed,lightgray] (0,1) -- (15,1) node[right] {0 (No)};
		\draw[dashed,lightgray] (0,3) -- (15,3) node[right] {1 (Yes)};
		\draw[dashed,thick,gray] (0,2) -- (15,2) node[right] {0.5};

		\draw[fill=yellow] (1,1) circle (4pt);
		\draw[fill=yellow] (1.2,1) circle (4pt);
		\draw[fill=yellow] (1.6,1) circle (4pt);
		\draw[fill=yellow] (1.8,1) circle (4pt);

		\draw[fill=violet] (2.6,3) circle (4pt);
		\draw[fill=violet] (2.8,3) circle (4pt);
		\draw[fill=violet] (3.3,3) circle (4pt);
		\draw[fill=violet] (3.6,3) circle (4pt);

		\only<1>\draw[thick,myblue1] (0.2558,0) -- (5,4.7869);
		\only<1>\draw[myblue1] (2.24,0) -- (2.24,4);

		% another violet data point
		\only<2>\draw[fill=violet] (14.5,3) circle (4pt);

		\only<2>\draw[thick,myblue1] (0,1.660) -- (14.5,3.4783);
		\only<2>\draw[myblue1] (2.71,0) -- (2.71,4);

	\end{tikzpicture}
\end{figure}
\end{frame}


% Why you should not use linear Regression... (Ctd.)
\begin{frame}{Why you should not use linear Regression... (Ctd.)}{}
	\begin{itemize}
		\item Linear regression: $h_{\bm{\theta}}(x) = \bm{\theta}^{\intercal} \bm{x}$
		\item By putting a \textbf{threshold} at 0.5, we can turn linear regression into a classifier
		\begin{itemize}
			\item If $h_{\bm{\theta}}(x) \ge 0.5$, predict $y = 1$
			\item If $h_{\bm{\theta}}(x) < 0.5$, predict $y = 0$
		\end{itemize}
		\item Outliers \textbf{affect the decision boundary}
		\item Furthermore, we only want $0 \le h_{\bm{\theta}}(x) \le 1$
		\item Linear regression can output $h_{\bm{\theta}}(x) \ll 0$ or $h_{\bm{\theta}}(x) \gg 1$
		\item \highlight{We need a better strategy!}
	\end{itemize}
\end{frame}


% Section: Model Architecture
%______________________________________________________________________
\section{Model Architecture}
\makedivider{Model Architecture}

% Subsection: Sigmoid Function
% --------------------------------------------------------------------------------------------------------
\subsection{Sigmoid Function}

% Logistic Regression Model
\begin{frame}{Logistic Regression Model}{}\important
	\begin{itemize}
		\item Remember that we want: $0 \le h_{\bm{\theta}}(x) \le 1$
		\item \textbf{Solution:} \highlight{Logistic/Sigmoid function:}
		\begin{equation}
			g(z) = \frac{1}{1 + e^{-z}}
		\end{equation}
		\item We plug $\bm{\theta}^{\intercal} \bm{x}$ into the sigmoid function:
		\begin{equation}
			h_{\bm{\theta}}(x) = g(\bm{\theta}^{\intercal} \bm{x}) = \frac{1}{1 + e^{-(\bm{\theta}^{\intercal} \bm{x})}}
		\end{equation}
	\end{itemize}
\end{frame}


% Logistic/Sigmoid Function
\begin{frame}{Logistic/Sigmoid Function}{}
	\divideTwo{0.49}{
		\vspace*{2mm}
		\begin{figure}
	\centering
	\begin{tikzpicture}[
		scale=0.75
	]

		\begin{axis}[
			grid=both,
    			grid style={line width=0.1pt, draw=lightgray!70},
			domain=-10:10,
			xlabel={$z$},
			ylabel={$g(z)$}
		]
    			\addplot[no marks,ultra thick,smooth,myblue1]{1 / (1 + exp(-x))};
    		\end{axis}

	\end{tikzpicture}
\end{figure}
	}{0.49}{
		\begin{itemize}
			\item $g(z)$ is symmetric around $z = 0$
			\item $0 \le g(z) \le 1$ holds true
		\end{itemize}
	}
\end{frame}


% Where does the Sigmoid come from?
\begin{frame}{Where does the Sigmoid come from?}{}\optional
	{\footnotesize
	\begin{alignat}{2}
		p(\mathcal{C}_1 \vert \bm{x})
			&=	\frac{p(\bm{x} \vert \mathcal{C}_1) p(\mathcal{C}_1)}{p(\bm{x})}
				= \frac{p(\bm{x} \vert \mathcal{C}_1) p(\mathcal{C}_1)}{\sum_j p(\bm{x}, \mathcal{C}_j)}
				= \frac{p(\bm{x} \vert \mathcal{C}_1) p(\mathcal{C}_1)}{\sum_j p(\bm{x} \vert \mathcal{C}_j) p(\mathcal{C}_j)}
			\\[1mm]
			&= \frac{p(\bm{x} \vert \mathcal{C}_1) p(\mathcal{C}_1)}{
				p(\bm{x} \vert \mathcal{C}_1) p(\mathcal{C}_1) + p(\bm{x} \vert \mathcal{C}_2) p(\mathcal{C}_2)}
			\\[1mm]
			&= \frac{1}{1 + p(\bm{x} \vert \mathcal{C}_2) p(\mathcal{C}_2) / (p(\bm{x} \vert \mathcal{C}_1) p(\mathcal{C}_1))}
			\\[1mm]
			&= \frac{1}{1 + \exp\{ -z \}} = g(z) && \longrightarrow \text{\highlight{logistic sigmoid}}
			\\[1mm]
		z 	&= \log \frac{p(\bm{x} \vert \mathcal{C}_1) p(\mathcal{C}_1)}{p(\bm{x} \vert \mathcal{C}_2) p(\mathcal{C}_2)}
			&& \longrightarrow \text{\highlight{log odds}}
	\end{alignat}}
\end{frame}


% Subsection: Probabilistic Interpretation
% --------------------------------------------------------------------------------------------------------
\subsection{Probabilistic Interpretation}

% Interpretation of Hypothesis Output
\begin{frame}{Interpretation of Hypothesis Output}{}
	\begin{itemize}
		\item $h_{\bm{\theta}}(x)$ is interpreted as the probability of instance $x$ belonging to class $y = 1$
		\item \textbf{Example:}
		\begin{equation}
			\bm{x} = \begin{bmatrix} x_0 \\ x_1 \end{bmatrix} = \begin{bmatrix} 1 \\ tumorSize \end{bmatrix}
		\end{equation}
		\vspace*{1mm}
		\item If $h_{\bm{\theta}}(x) = 0.7$, we have to tell the patient that there
			is a \textbf{70\,\% chance} of the tumor being malignant $\Rightarrow p(y = 1 \vert \bm{x}, \bm{\theta})$
		\item \textbf{Binary case:} $p(y = 0 \vert \bm{x}, \bm{\theta}) = 1 - p(y = 1 \vert \bm{x}, \bm{\theta})$
	\end{itemize}
\end{frame}


% Subsection: Model Training
% --------------------------------------------------------------------------------------------------------
\subsection{Model Training}

% Training Setup
\begin{frame}{Training Setup}{}
	\begin{itemize}
		\item We have a labeled training set ($\Rightarrow$ \textbf{supervised learning}):
		\begin{equation}
			\mathcal{D} = 
				\left\{ (\bm{x}^{(1)}, y^{(1)}), (\bm{x}^{(2)}, y^{(2)}), \dots, (\bm{x}^{(n)}, y^{(n)}) \right\}
				= \left\{ (\bm{x}^{(i)}, y^{(i)}) \right\}_{i=1}^n
		\end{equation}
		\item Each $\bm{x}$ is a vector of features:
		\begin{equation}
			\bm{x} = \begin{bmatrix} x_0 \\ \vdots \\ x_{m} \end{bmatrix} \in \mathbb{R}^{m+1}
			\quad \text{and} \quad x_0 = 1 \quad \text{and} \quad y \in \{0, 1\}
		\end{equation}
		\item \highlight{How to choose the parameters $\bm{\theta}$?}
	\end{itemize}
\end{frame}


% Logistic Regression Cost Function
\begin{frame}{Logistic Regression Cost Function}{}
	\begin{itemize}
		\item Gradient descent is performed in order to find the parameters $\bm{\theta}$
		\item To this end, a cost function is needed:
		\begin{equation}
			\mathcal{J}(\bm{\theta}) = \frac{1}{n} \sum_{i=1}^n \ell(h_{\bm{\theta}}(\bm{x}^{(i)}), y^{(i)})
		\end{equation}
		\item The cost function $\ell(h_{\bm{\theta}}(\bm{x}), y)$ is defined as follows: \\
		{\footnotesize \textit{(square loss would be \textbf{non-convex...})}}
		\begin{equation}
			\ell(h_{\bm{\theta}}(\bm{x}), y) =
			\begin{cases}
				-\log(h_{\bm{\theta}}(\bm{x})) 		& \text{if}\ y = 1	\\
				-\log(1 - h_{\bm{\theta}}(\bm{x}))	& \text{if}\ y = 0
			\end{cases}
		\end{equation}
	\end{itemize}
\end{frame}


% Logistic Regression Cost Function (Ctd.)
\begin{frame}{Logistic Regression Cost Function (Ctd.)}{}
	\divideTwo{0.49}{
		\textbf{\textit{y} = 1:}
		\begin{figure}
	\centering
	\begin{tikzpicture}[
		scale=0.6
	]

		\begin{axis}[
			domain=0:1,
			xlabel={$h_{\bm{\theta}}(\bm{x})$},
			ylabel={$cost\ \ell$}
		]
    			\addplot[no marks,ultra thick,smooth,myblue1]{-ln(x)};	
    		\end{axis}

	\end{tikzpicture}
\end{figure}
	}{0.49}{
		\textbf{\textit{y} = 0:}
		\begin{figure}
	\centering
	\begin{tikzpicture}[
		scale=0.6
	]

		\begin{axis}[
			domain=0:1,
			xlabel={$h_{\bm{\theta}}(\bm{x})$},
			ylabel={$cost\ \ell$}
		]
    			\addplot[no marks,ultra thick,smooth,myblue1,samples=500]{-ln(1 - x)};
    		\end{axis}

	\end{tikzpicture}
\end{figure}
	}
\end{frame}


% Logistic Regression Cost Function (Ctd.)
\begin{frame}{Logistic Regression Cost Function (Ctd.)}{}\important
	\begin{itemize}
		\item $\ell(h_{\bm{\theta}}(\bm{x}), y)$ can be written in a more compact form:
		\begin{equation}
			\ell(h_{\bm{\theta}}(\bm{x}), y) = -y \log(h_{\bm{\theta}}(\bm{x})) -
				(1 - y) \log(1 - h_{\bm{\theta}}(\bm{x}))
		\end{equation}
		\vspace*{-6mm}
		\begin{itemize}
			\item If $y = 1$, we get: $-\log(h_{\bm{\theta}}(\bm{x}))$
			\item If $y = 0$, we get: $-\log(1 - h_{\bm{\theta}}(\bm{x}))$
		\end{itemize}
		\item This gives the \highlight{cross entropy} cost function $\mathcal{J}(\bm{\theta})$:
		\begin{equation}
			\mathcal{J}(\bm{\theta}) = \frac{1}{n} \sum_{i=1}^n \left[
				-y^{(i)} \log(h_{\bm{\theta}}(\bm{x}^{(i)})) - (1 - y^{(i)}) \log(1 - h_{\bm{\theta}}(\bm{x}^{(i)}))
			\right]
		\end{equation}
	\end{itemize}
\end{frame}


% Derivation of Cross Entropy
\begin{frame}{Derivation of Cross Entropy}{}\optional
	\begin{itemize}
		\item The likelihood function can be written in the form:
		\begin{equation}
			\mathcal{L}(\bm{\theta}) = \prod_{i=1}^n h_{\bm{\theta}}(\bm{x}^{(i)})^{y^{(i)}} \cdot (1 - h_{\bm{\theta}}(\bm{x}^{(i)}))^{1 - y^{(i)}}
		\end{equation}
		\item The cost function is then given by the \textbf{negative log-likelihood}:
		\begin{equation}
			\mathcal{J}(\bm{\theta}) = -\log \mathcal{L}(\bm{\theta})
		\end{equation}
	\end{itemize}
\end{frame}


% Gradient Descent
\begin{frame}{Gradient Descent}{}
	\begin{itemize}
		\item The goal is to minimize $\mathcal{J}(\bm{\theta})$: $\bm{\theta}^*
			= \argmin_{\bm{\theta}} \mathcal{J}(\bm{\theta})$
		\item \texttt{Repeat until convergence \{} \\
			$\qquad \bm{\theta}^{(t+1)} \longleftarrow \bm{\theta}^{(t)} - \alpha \nabla_{\bm{\theta}}
				\mathcal{J}(\bm{\theta}^{(t)}) \quad$
			\textcolor{myblue1}{\textit{// simultaneously update all} $\theta_j$} \\
		\texttt{\}}
		\item The partial derivative $\nabla_{\bm{\theta}} \mathcal{J}(\bm{\theta})$ is given by:
		{\footnotesize
		\begin{equation}
			\nabla_{\bm{\theta}} \mathcal{J}(\bm{\theta}) = \frac{1}{n} \sum_{i=1}^n \left(
				h_{\bm{\theta}}(\bm{x}^{(i)}) - y^{(i)}
			\right) x_j^{(i)}
		\end{equation}}
	\end{itemize}
	\vspace*{-3mm}
	\begin{boxBlueNoFrame}
		\highlight{Algorithm looks identical to linear regression, but $h_{\bm{\theta}}(\bm{x})$ is different!}
	\end{boxBlueNoFrame}
\end{frame}


% Subsection: Decision Boundary
% --------------------------------------------------------------------------------------------------------
\subsection{Decision Boundary}

% Decision Boundary
\begin{frame}{Decision Boundary}{}
	\divideTwo{0.62}{
		\begin{itemize}
			\item \highlight{For classification we have to set a threshold}
			\item Suppose we predict $y = 1$ if $h_{\bm{\theta}}(x) \ge 0.5$
			\begin{itemize}
				\item This means $g(z) \ge 0.5$
				\item This is equivalent to $z \ge 0$ and $\bm{\theta}^{\intercal} \bm{x} \ge 0$
			\end{itemize}
			\item Suppose we predict $y = 0$ if $h_{\bm{\theta}}(x) < 0.5 \Rightarrow \bm{\theta}^{\intercal} \bm{x} < 0$
		\end{itemize}
	}{0.37}{
		\vspace*{2mm}
		\begin{figure}
	\centering
	\begin{tikzpicture}[
		scale=0.6
	]

		\begin{axis}[
			domain=-10:10,
			xlabel={$\bm{\theta}^{\intercal} \bm{x}$},
			ylabel={$h_{\bm{\theta}}(\bm{x}) = g(\bm{\theta}^{\intercal} \bm{x})$}
		]
		
			\draw[fill=gray,fill opacity=0.40] (axis cs:-12,-0.2) rectangle (axis cs:0,1.2);
			\draw[fill=lightgray,fill opacity=0.10] (axis cs:0,-0.2) rectangle (axis cs:12,1.2);
			\addplot[no marks,ultra thick,smooth,myblue1]{1 / (1 + exp(-x))};

			\node at (axis cs:-5,0.5) {\huge $\bm{\ominus}$};
			\node at (axis cs:5,0.5) {\huge $\bm{\oplus}$};
			
    		\end{axis}

	\end{tikzpicture}
\end{figure}
	}
\end{frame}


% Decision Boundary (Ctd.)
\begin{frame}{Decision Boundary (Ctd.)}{}
	\begin{itemize}
		\item Suppose we have the following hypothesis:
		\begin{equation*}
			h_{\bm{\theta}}(\bm{x}) = g(\theta_0 + \theta_1 x_1 + \theta_2 x_2)
		\end{equation*}
		\item Using gradient descent we obtained the following coefficients:
		\begin{equation*}
			\theta_0 = -3 \qquad \theta_1 = 1 \qquad \theta_2 = 1
		\end{equation*}
		\item Predict $y = 1$ if $-3 + x_1 + x_2 \ge 0$
	\end{itemize}
\end{frame}


% Decision Boundary (Ctd.)
\begin{frame}{Decision Boundary (Ctd.)}{}
	\divideTwo{0.33}{
		\begin{figure}
	\centering
	\begin{tikzpicture}[
		scale=0.5
	]

		\draw[->,thick] (0,0) -- (7,0) node[right] {$x_1$};
		\draw[->,thick] (0,0) -- (0,7) node[above] {$x_2$};

		\draw[very thick,myblue1] (0,6) -- (6,0);
		\node[myblue1,rotate=-45] at (3.375,3.375) {\footnotesize Decision boundary};

		\foreach \x/\i in {2/1,4/2,6/3} \draw (\x,-0.25) -- (\x,0.25) node[below=2mm] {\i};
		\foreach \y/\i in {2/1,4/2,6/3} \draw (-0.25,\y) -- (0.25,\y) node[left=2mm] {\i};

		\draw[fill=yellow] (2,2) circle (5pt);
		\draw[fill=yellow] (3.5,2) circle (5pt);
		\draw[fill=yellow] (1,1) circle (5pt);

		\draw[fill=violet] (4,4) circle (5pt);
		\draw[fill=violet] (4.4,6) circle (5pt);
		\draw[fill=violet] (3,5) circle (5pt);
		\draw[fill=violet] (4.5,5) circle (5pt);

	\end{tikzpicture}
\end{figure}
	}{0.66}{
		\begin{itemize}
			\item Predict $y = 1$, if $-3 + x_1 + x_2 \ge 0$
			\item The decision boundary satisfies $-3 + x_1 + x_2 = 0$
			\item If $x_2 = 0$, then $x_1 = 3$ and vice versa
			
			\vspace*{5mm}
			\begin{boxBlueNoFrame}
				\footnotesize
				\highlight{Logistic regression is not a maximum-margin classifier
				(although the cost function can be adjusted to get that $\Rightarrow$ Hinge loss)}
			\end{boxBlueNoFrame}
		\end{itemize}
	}
\end{frame}


% Example: Decision Boundary
\begin{frame}{Example: Decision Boundary}{}
	\divideTwo{0.49}{
		\begin{figure}
			\includegraphics[scale=0.45]{07_logistic_regression/02_img/logreg_example_linear}
		\end{figure}
	}{0.49}{
		\begin{figure}
			\includegraphics[scale=0.45]{07_logistic_regression/02_img/logreg_example_linear_boundary}
		\end{figure}
	}
	
	\begin{center}
		\highlight{Where is the sigmoid function?}
	\end{center}
\end{frame}


% Example: Logistic Function
\begin{frame}{Example: Logistic Function}{}
	\vspace*{-7mm}
	\begin{figure}
		\centering
		\includegraphics[scale=0.6]{07_logistic_regression/02_img/logreg_example_linear_logistic_function}
	\end{figure}
\end{frame}


% Section: Non-linear Data
%______________________________________________________________________
\section{Non-linear Data}
\makedivider{Non-linear Data}

% Subsection: Feature Mapping
% --------------------------------------------------------------------------------------------------------
\subsection{Feature Mapping}

% Non-Linear Decision Boundaries
\begin{frame}{Non-Linear Decision Boundaries}{}
	\begin{itemize}
		\item \highlight{Feature mapping} can be used to obtain non-linear decision boundaries
		\item \textbf{Example:}
		\begin{itemize}
			\item Imagine a circular data set
			\item Using the features...
			\begin{equation*}
				h_{\bm{\theta}}(\bm{x}) = g(\theta_0 + \theta_1 x_1 + \theta_2 x_2 +
				\textcolor{myblue1}{\theta_3 x_1^2} + \textcolor{myblue1}{\theta_4 x_2^2})
			\end{equation*}
			\item ...the algorithm could e.\,g. choose: $\bm{\theta} =
				\begin{bmatrix} -1, 0, 0, 1, 1 \end{bmatrix}^{\intercal}$
			\item So we would get: $x_1^2 + x_2^2 = 1 \Rightarrow$ \textbf{equation of a unit circle}
		\end{itemize}
	\end{itemize}
\end{frame}


% Example: Non-Linear Decision Boundary
\begin{frame}{Example: Non-Linear Decision Boundary}{}
	\divideTwo{0.49}{
		\begin{figure}
			\includegraphics[scale=0.29]{07_logistic_regression/02_img/logreg_example_non_linear}
		\end{figure}
	}{0.49}{
		\begin{figure}
			\includegraphics[scale=0.3]{07_logistic_regression/02_img/logreg_example_non_linear_boundary}
		\end{figure}
	}
\end{frame}


% Subsection: Regularization
% --------------------------------------------------------------------------------------------------------
\subsection{Regularization}

% Logistic Regression Cost Function (Ctd.)
\begin{frame}{Logistic Regression Cost Function (Ctd.)}{}
	\begin{itemize}
		\item We should apply regularization for non-linear decision boundaries:
		\footnotesize
		\begin{equation}
			\frac{1}{n} \sum_{i=1}^n \left[
				-y^{(i)} \log(h_{\bm{\theta}}(\bm{x}^{(i)})) - (1 - y^{(i)}) \log(1 - h_{\bm{\theta}}(\bm{x}^{(i)}))
			\right] + \textcolor{myblue1}{\frac{\lambda}{2m} \sum_{j=1}^m \theta_j^2}
		\end{equation}
		\normalsize
		\item The last term prevents the parameters $\theta_j$ from becoming too large
		\item $\lambda \ge 0$ controls the degree of regularization
		\item This leads to smoother decision boundaries
	\end{itemize}
\end{frame}


% Section: Multi-Class Classification
%______________________________________________________________________
\section{Multi-Class Classification}
\makedivider{Multi-Class Classification}

% Subsection: Multiple Classes
% --------------------------------------------------------------------------------------------------------
\subsection{Multiple Classes}

% Multi-Class Classification
\begin{frame}{Multi-Class Classification}{}
	\begin{itemize}
		\item Logistic regression can handle two classes only, namely \textbf{0} and \textbf{1}
		\item \textbf{What if there are more than two classes?}
		\item Two common techniques:
		\begin{itemize}
			\item \highlight{One-vs-Rest (OVR)} 	$\Rightarrow$ One-against-All
			\item \highlight{One-vs-One (OVO)} 	$\Rightarrow$ Pairwise classification
		\end{itemize}
		\item Several classifiers are trained
		\item During prediction the classifiers \textbf{vote for the correct class}
		\item Such techniques can be used for all binary classifiers
	\end{itemize}
\end{frame}


% Subsection: One-vs-Rest (OVR)
% --------------------------------------------------------------------------------------------------------
\subsection{One-vs-Rest (OVR)}

% Multi-Class Classification: One-vs-Rest (OVR)
\begin{frame}{Multi-Class Classification: One-vs-Rest (OVR)}{}
	\divideTwo{0.49}{
		\begin{itemize}
			\item \textbf{Train one classifier per class} (expert for that class)
			\item We get $\vert \mathcal{C} \vert$ classifiers
			\item The $k$-th classifier learns to distinguish the $k$-th class from all the others
			\item Set the labels of examples from class $k$ to \textbf{1}, all the others to \textbf{0}
		\end{itemize}
	}{0.49}{
		\vspace*{2mm}
		\begin{figure}
	\centering
	\begin{tikzpicture}[
		scale=0.8
	]

		\begin{axis}[
			xlabel={$x_1$},
			ylabel={$x_2$}
		]
		
			\addplot[
				only marks,mark=*,mark size=2.0,fill=red,discard if not={c}{0}
			] table{07_logistic_regression/05_data/data_multi_class.txt};

			\addplot[
				only marks,mark=*,mark size=2.0,fill=blue,discard if not={c}{1}
			] table{07_logistic_regression/05_data/data_multi_class.txt};

			\addplot[
				only marks,mark=*,mark size=2.0,fill=green,discard if not={c}{2}
			] table{07_logistic_regression/05_data/data_multi_class.txt};

			\draw[thick,dashed,red] (axis cs: -7,-12) -- (axis cs: -2,12);
			\node[red] at (axis cs: -10,-6) {\textbf{1}};
			\node at (axis cs: 0,0) {\textbf{0}};
			
    		\end{axis}
		
	\end{tikzpicture}
\end{figure}
	}
\end{frame}


% Subsection: One-vs-One (OVO)
% --------------------------------------------------------------------------------------------------------
\subsection{One-vs-One (OVO)}

% Multi-Class Classification: One-vs-One (OVO)
\begin{frame}{Multi-Class Classification: One-vs-One (OVO)}{}
	\divideTwo{0.49}{
		\vspace*{2mm}
		\begin{figure}
	\centering
	\begin{tikzpicture}[
		scale=0.8
	]

		\begin{axis}[
			xlabel={$x_1$},
			ylabel={$x_2$}
		]
		
			\addplot[
				only marks,mark=*,mark size=2.0,fill=red,discard if not={c}{0}
			] table{11_svm/05_data/data_multi_class.txt};

			\addplot[
				only marks,mark=*,mark size=2.0,fill=blue,discard if not={c}{1}
			] table{11_svm/05_data/data_multi_class.txt};

			\addplot[
				only marks,mark=*,mark size=2.0,fill=white,draw=gray,discard if not={c}{2}
			] table{11_svm/05_data/data_multi_class.txt};

			\draw[thick,dashed,red] (axis cs: -10,-10) -- (axis cs: 5,7.5);
			\node[red] at (axis cs: -10,-6) {\textbf{+1}};
			\node[blue] at (axis cs: -5,-7.5) {\textbf{-1}};
			\node[gray] at (axis cs: -2.5,7.5) {ignored};
			
    		\end{axis}
		
	\end{tikzpicture}
\end{figure}
	}{0.49}{
		\begin{itemize}
			\item \textbf{Train one classifier for each pair of classes}
			\item We get $\binom{\mathcal{C}}{2}$ classifiers
			\item Ignore all other examples that do not belong to either of the two classes
			\item \textbf{Voting}: Count how often each class wins; the class with highest count is predicted
		\end{itemize}
	}
\end{frame}


% Section: Wrap-Up
%______________________________________________________________________
\section{Wrap-Up}
\makedivider{Wrap-Up}

% Subsection: Summary
% --------------------------------------------------------------------------------------------------------
\subsection{Summary}

% Summary
\begin{frame}{Summary}{}

\end{frame}


% Subsection: Lecture Overview
% --------------------------------------------------------------------------------------------------------
\subsection{Lecture Overview}

\makeoverview{3}


% Subsection: Self-Test Questions
% --------------------------------------------------------------------------------------------------------
\subsection{Self-Test Questions}

% Self-Test Questions
\begin{frame}{Self-Test Questions}{}

\end{frame}


% Subsection: Recommended Literature and further Reading
% --------------------------------------------------------------------------------------------------------
\subsection{Recommended Literature and further Reading}

% Literature
%______________________________________________________________________
\begin{frame}{Recommended Literature and further Reading}{}
	\footnotesize
	\begin{thebibliography}{2}

	\end{thebibliography}
\end{frame}


% Thank you
%______________________________________________________________________
\makethanks

\end{document}