\documentclass[11pt]{article}

% ====================================================
% ====================================================
% USEPACKAGES AND IMPORTS
% ====================================================
% ====================================================

\usepackage[T1]{fontenc}
\usepackage[utf8]{inputenc}
\usepackage[english]{babel}

\usepackage[
  left=2.5cm,
  right=2cm,
  top=2.5cm,
  bottom=4cm,
]{geometry}

\usepackage{tikz}
\usepackage{eso-pic}
\usepackage{fancyhdr}
\usepackage{etoolbox}
\usepackage{enumitem}
\usepackage{amssymb}
\usepackage{xparse}
\usepackage{roboto}
\renewcommand{\familydefault}{\sfdefault}

% definitions
% ====================================================
\let\titleoriginal\title
\renewcommand{\title}[1]{
	\titleoriginal{#1}
	\newcommand{\thetitle}{#1}
}

\setlength{\parskip}{\baselineskip}%
\setlength{\parindent}{0pt}%

\newcommand\exercise[4][]{
	\ifx\hfuzz#1\hfuzz
  		\item #2\vspace{0.2cm}\newline #3 \vspace{#4}%
	\else
  		\item #2\vspace{0.2cm}\newline #3\vspace{0.5cm}\newline Solution:\newline#1 \vspace{#4}%
	\fi
}

% header and footer
\pagestyle{fancy}
\fancyhf{}

\lhead{\thetitle}
\rhead{\includegraphics[width=2cm]{../../img/logo_dhbw.png}}
\cfoot{\thepage\\\vspace{0.6cm} \small Applied Machine Learning Fundamentals}

\setlength{\headsep}{1.5cm}

% title page theme
\newcommand\BackgroundPic{%
\put(0,0){%
\parbox[b][\paperheight]{\paperwidth}{%
\vfill
\centering
\tikz[overlay,remember picture] \node[opacity=0.2, at=(current page.center)] {
	\includegraphics[height=\paperheight,width=\paperwidth]{../../img/processor.jpg}
};
\includegraphics[width=4cm, trim=-5.5cm 0 0 -1cm]{../../img/logo_dhbw.png}
%\includegraphics[width=\paperwidth,height=\paperheight,%
%keepaspectratio]{../../img/processor.jpg}%
\vfill
}}}


% title and author
\title{Exercise 1 - Introduction}

% begin of document
\begin{document}
\AddToShipoutPicture*{\BackgroundPic}

\maketitle
\medskip
\newpage

\section{Linear Algebra Refresher}
\begin{enumerate}[label=\alph*)]
\exercise{Matrix Operations (1 point)}{A fellow student suggests that matrix addition and multiplication are very similar to scalar addition and multiplication, i.\,e. commutative, associative and distributive. Is this a correct statement? Prove it mathematically or disprove it by providing at least one counter example per property (commutativity, associativity, distributivity).}{14.5cm}

\exercise{Matrix Inverse (1 point)}{What is a matrix inverse? How can you build the inverse of a non-square matrix? You would like to invert a matrix $M \in \mathbb{R}^{2 \times 3}$ - write down the equation for computing it and specify the dimensionality of the matrices after each single operation (e.g. multiplication, inverse).}{7cm}

\exercise{Eigenvectors and Eigenvalues (1 point)}{Explain what eigenvectors and eigenvalues of a matrix $M$ are. Why are they relevant in machine learning?}{7cm}

\end{enumerate}

\section{Statistics Refresher}
\begin{enumerate}[label=\alph*)]
\exercise{Terminology (1 point)}{What is a random variable? What is a probability density function (PDF)? What is a probability mass function (PMF)? What do a PDF and a PMF tell us about a random variable?}{7cm}

\exercise{Expectation and Variance (1 point)}{State the general definition of expectation and variance for the probability density $f : \Omega \rightarrow \mathbb{R}$ of a continuous random variable. What do expectation and variance express?}{7cm}

\end{enumerate}

\section{Optimization}
\begin{enumerate}[label=\alph*)]
\exercise{Numerical Optimization - Gradient Descent (5 points)}{Implement a simple gradient descent algorithm for finding a minimum of the Rosenbrock function with $n = 2$ using Python and NumPy: $$f(x) = \sum_{i = 0}^{n - 1} \left[100 (x_{i+1} - x_i^2)^2 + (x_i - 1)^2\right]$$ Submit your code and a plot of the learning curve for the best run of your gradient descent implementation. Which learning rate worked best? (Hint: You need to find the first derivative(s) of $f(x)$ for $n = 2$ and iteratively evaluate them during gradient descent. Automatic differentiation tools are not allowed for this exercise.)}{7cm}

\end{enumerate}

\end{document}