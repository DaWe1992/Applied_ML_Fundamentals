\documentclass[11pt]{article}

% ====================================================
% ====================================================
% USEPACKAGES AND IMPORTS
% ====================================================
% ====================================================

\usepackage[T1]{fontenc}
\usepackage[utf8]{inputenc}
\usepackage[english]{babel}

\usepackage{fancyhdr}

% definitions
% ====================================================
\let\titleoriginal\title
\renewcommand{\title}[1]{
	\titleoriginal{#1}
	\newcommand{\thetitle}{#1}
}

\setlength{\parskip}{\baselineskip}%
\setlength{\parindent}{0pt}%

% header and footer
\pagestyle{fancy}
\fancyhf{}
\lhead{Applied Machine Learning Fundamentals}
\rhead{\thetitle}
\cfoot{\thepage}

% ====================================================
% ====================================================
% PRESENTATION DATA
% ====================================================
% ====================================================

\title[Classification I]{*** Applied Machine Learning Fundamentals *** Decision Trees and Ensembles}
\institute{SAP\,SE}
\author{Daniel Wehner}
\date{\today}
\prefix{DT}

% ====================================================
% ====================================================
% BEGIN OF DOCUMENT
% ====================================================
% ====================================================

\begin{document}

% Title frame
%______________________________________________________________________
\maketitlepage


% Agenda
%______________________________________________________________________
\begin{frame}{Agenda \today}
	\begin{multicols}{2}
		\tableofcontents
	\end{multicols}
\end{frame}


% Section: Introduction
%______________________________________________________________________
\section{Introduction}
\makedivider{Introduction}

% What we want...
\begin{frame}{What we want...}{}
	\divideTwo{0.49}{
		\vspace*{4mm}
		\begin{table}
	\scalebox{0.6}{
	\begin{tabular}{| c | c | c | c | c |}
		\hline
		\highlight{A}	&
		\highlight{F} 	&
		\highlight{S} 	&
		\highlight{N} 	&
		\highlight{H}		\\ \hline\hline
		0	&	1	&	0	&	1	&	1	\\ \hline
		1	&	0	&	0	&	0	&	0	\\ \hline
		1	&	0	&	1	&	0	&	1	\\ \hline
		1	&	1	&	1	&	1	&	0	\\ \hline
		0	&	0	&	1	&	1	&	0	\\ \hline
		0	&	0	&	0	&	1	&	1	\\ \hline
		1	&	0	&	0	&	0	&	0	\\ \hline
		0	&	1	&	0	&	1	&	1	\\ \hline
		1	&	1	&	0	&	0	&	0	\\ \hline
		1	&	0	&	1	&	0	&	1	\\ \hline
		1	&	1	&	1	&	1	&	1	\\ \hline
		1	&	1	&	0	&	1	&	0	\\ \hline
		1	&	0	&	1	&	0	&	0	\\ \hline
		0	&	1	&	0	&	0	&	1	\\ \hline
		1	&	0	&	0	&	1	&	1	\\ \hline
		1	&	1	&	1	&	0	&	0	\\ \hline
	\end{tabular}}
\end{table}
	}{0.49}{
		\input{08_decision_trees/01_tikz/tree}
	}
\end{frame}


% What are Decision Trees?
\begin{frame}{What are Decision Trees?}{}
	\begin{itemize}
		\item Decision trees are induced in a \highlight{supervised} fashion
		\item Originally invented by \textit{Ross Quinlan} (1986)
		\item Decision trees are grown \textbf{recursively} $\rightarrow$ \textit{'divide-and-conquer'}
		\item A decision tree consists of:

		\begin{tabbing}
			\hspace*{2.5cm}\= \kill
			\textbf{Nodes}	\>	Each node corresponds to an attribute test 	\\
			\textbf{Edges}	\>	One edge per possible test outcome			\\
			\textbf{Leaves}	\>	Class label to predict
		\end{tabbing}
	\end{itemize}
\end{frame}


% Classifying new Instances
\begin{frame}{Classifying new Instances}{}
	\divideTwo{0.49}{
		\begin{itemize}
			\item Suppose we get a new instance:
			
			\footnotesize
			\begin{tabbing}
				\hspace*{2.5cm}\= \kill
				\texttt{Outlook}			\>	rainy		\\
				\texttt{Temperature} 		\>	mild	 	\\
				\texttt{Humidity}			\>	normal	\\
				\texttt{Wind}			\>	strong
			\end{tabbing}
			\normalsize

			\item \textbf{What is its class?}
			\item Answer: \textbf{No}
		\end{itemize}
	}{0.49}{
		\input{08_decision_trees/01_tikz/tree}
	}
\end{frame}


% Another Decision Tree...
\begin{frame}{Another Decision Tree...}{}
	\bubble{1}{11}{\footnotesize \textbf{Is this one better?}}
	\vspace*{-2mm}
	% Set the overall layout of the tree
\tikzstyle{level 1}=[level distance=2cm,sibling distance=6cm]
\tikzstyle{level 2}=[level distance=2.5cm,sibling distance=2cm]
\tikzstyle{level 3}=[level distance=2.5cm,sibling distance=1.5cm]

% Define styles for bags and leafs
\tikzstyle{bag}=[rectangle,draw=black,text width=4em,text centered]
\tikzstyle{end}=[circle,draw=black,minimum width=3pt,fill,inner sep=0pt]

\begin{figure}
	\centering
	\begin{tikzpicture}[
		scale=0.6,
		every node/.style={scale=0.5},
		sloped
	]
		\node[bag]{\highlight{Temp.}}
    		child{
     	   		node[bag]{\highlight{Outlook}}        
            			child{
                			node[end, label=below:{\textbf{\underline{No}}}]{}
                			edge from parent
                			node[above]{\textit{sunny}}
            			}
            			child{
                			node[end, label=below:{\textbf{\textcolor{red}{\underline{?}}}}]{}
               			edge from parent
                			node[above]{\textit{rain}}
            			}
				child{
                			node[end, label=below:{\textbf{\underline{Yes}}}]{}
               			edge from parent
                			node[above]{\textit{overcast}}
            			}
            			edge from parent 
            			node[above]{\textit{hot}}
    		}
		child{
        		node[bag]{\highlight{Outlook}}        
        		child {
                		node[bag]{\highlight{Humid.}}
                		child{
                			node[end, label=below:{\textbf{\underline{No}}}]{}
                			edge from parent
                			node[above]{\textit{high}}
                		}
				child{
                			node[end, label=below:{\textbf{\underline{Yes}}}]{}
                			edge from parent
                			node[above]{\textit{normal}}
                		}
				child[missing]{}
                		edge from parent
                		node[above]{\textit{sunny}}
            		}
            		child{
                		node[bag]{\highlight{Humid.}}
				child[missing]{}
				child{
					node[bag]{\highlight{Wind}}
					child{
						node[end, label=below:{\textbf{\underline{No}}}]{}
                				edge from parent
                				node[above]{\textit{strong}}
					}
					child{
						node[end, label=below:{\textbf{\underline{Yes}}}]{}
                				edge from parent
                				node[above]{\textit{weak}}
					}
					edge from parent
                			node[above]{\textit{high}}
				}
				child{
					node[end, label=below:{\textbf{\underline{Yes}}}]{}
                			edge from parent
                			node[above]{\textit{normal}}
				}
                		edge from parent
                		node[above]{\textit{rainy}}
            		}
			child{
                		node[end, label=below:{\textbf{\underline{Yes}}}]{}
                		edge from parent
                		node[above]{\textit{overcast}}
            		}
        		edge from parent         
            		node[above]{\textit{mild}}
		}
    		child{
        		node[bag]{\highlight{Outlook}}        
        		child {
                		node[end, label=below:{\textbf{\underline{Yes}}}]{}
                		edge from parent
                		node[above]{\textit{sunny}}
            		}
            		child{
                		node[bag]{\highlight{Humid.}}
                		child{
                			node[end, label=below:{\textbf{\textcolor{red}{\underline{?}}}}]{}
                			edge from parent
                			node[above]{\textit{high}}
                		}
				child{
                			node[bag]{\highlight{Wind}}
					child{
                				node[end, label=below:{\textbf{\underline{No}}}]{}
                				edge from parent
                				node[above]{\textit{strong}}
                			}
					child{
                				node[end, label=below:{\textbf{\underline{Yes}}}]{}
                				edge from parent
                				node[above]{\textit{weak}}
                			}
                			edge from parent
                			node[above]{\textit{normal}}
                		}
                		edge from parent
                		node[above]{\textit{rainy}}
            		}
			child{
                		node[end, label=below:{\textbf{\underline{Yes}}}]{}
                		edge from parent
                		node[above]{\textit{overcast}}
            		}
        		edge from parent         
            		node[above]{\textit{cool}}
		};
	\end{tikzpicture}
\end{figure}
\end{frame}


% Inductive Bias of Decision Trees
\begin{frame}{Inductive Bias of Decision Trees}{}
	\divideTwo{0.75}{
		\begin{itemize}
			\item Complex models tend to \textbf{overfit} the data and do not generalize well
			\item Small decision trees are preferred
			\vspace*{4mm}
			\begin{boxBlueNoFrame}
				\textbf{Occam's razor}: \\
				\footnotesize \textbf{`More things should not be used than are necessary.'}
			\end{boxBlueNoFrame}
			\vspace*{2mm}
			\item \highlight{Prefer the simplest hypothesis that fits the data!}
		\end{itemize}
	}{0.20}{
		\begin{figure}
			\centering
			\includegraphics[scale=0.25]{08_decision_trees/02_img/william_of_ockham}
		\end{figure}
	}
\end{frame}


% The Root of all Evil... Which Attribute to choose?
\begin{frame}{The Root of all Evil... Which Attribute to choose?}{}
	\divideTwo{0.49}{
		\input{08_decision_trees/01_tikz/attribute_split_outlook}
		\vspace*{0.25mm}	
	}{0.49}{
		% Set the overall layout of the tree
\tikzstyle{level 1}=[level distance=3.5cm, sibling distance=2cm]
\tikzstyle{level 2}=[level distance=3.5cm, sibling distance=2cm]

% Define styles for bags and leafs
\tikzstyle{bag} = [rectangle, draw=black, text width=4em, text centered]
\tikzstyle{end} = [circle, draw=black, minimum width=3pt, fill, inner sep=0pt]

\begin{figure}
	\centering
	\begin{tikzpicture}[
		scale=0.6,
		every node/.style={scale=0.5},
		sloped
	]
		\node[bag]{\highlight{Temp.}}
		child{
			node[bag,align=center]{Yes Yes\\No No}
			edge from parent
			node[above]{\textit{hot}}
		}
    		child{
			node[bag,align=center]{Yes Yes\\Yes Yes\\No No}
			edge from parent
			node[above]{\textit{mild}}
		}
		child{
			node[bag,align=center]{Yes Yes\\Yes No}
			edge from parent
			node[above]{\textit{cool}}
		};
	\end{tikzpicture}
\end{figure}
		\vspace*{0.25mm}	
	}

	\divideTwo{0.49}{
		\input{08_decision_trees/01_tikz/attribute_split_wind}
	}{0.49}{
		\input{08_decision_trees/01_tikz/attribute_split_humidity}
	}
\end{frame}


% Finding a proper Attribute
\begin{frame}{Finding a proper Attribute}{}
	\divideTwo{0.79}{
		\begin{itemize}
			\item Simple and small trees are preferred
			\begin{itemize}
				\item Data in successor node should be \textbf{as pure as possible}
				\item I.\,e. nodes containing one class only are preferable
			\end{itemize}
			\item \textbf{Question:} How can we express this thought as a mathematical formula?
			\item \textbf{Answer:}
			\begin{itemize}
				\item \highlight{Entropy} (\textit{Claude E. Shannon})
				\item Originates in the field of \textbf{information theory}
			\end{itemize}
		\end{itemize}
	}{0.19}{
		\includegraphics[scale=0.3]{08_decision_trees/02_img/claude_shannon}
	}
\end{frame}


% Measure of Impurity: Entropy
\begin{frame}{Measure of Impurity: Entropy}{}
	\begin{itemize}
		\item Entropy is a measure of chaos in the data (measured in bits)
		\item \textbf{Example:} Consider two classes $A$ and $B$ ($\mathcal{C} = \{ A, B \}$)
	
		\footnotesize
		\begin{tabbing}
			\hspace*{5cm}\=\hspace*{1.5cm}\= \kill
			$E(\{ \bm{A}, \bm{A}, \bm{A}, \bm{A}, \bm{A}, \bm{A}, \bm{A}, \bm{A} \})$
				\>	$\rightarrow$ 0		\>	$Bits$	\\
			$E(\{ \bm{A}, \bm{A}, \bm{A}, \bm{A}, \bm{A}, \bm{A}, B, B \})$
				\>	$\rightarrow$ 0.81 	\>	$Bits$	\\
			$E(\{ \bm{A}, \bm{A}, \bm{A}, \bm{A}, B, B, B, B \})$
				\>	$\rightarrow$ 1		\>	$Bit$		\\
			$E(\{ \bm{A}, \bm{A}, B, B, B, B, B, B \})$
				\>	$\rightarrow$ 0.81 	\>	$Bits$	\\
			$E(\{ B, B, B, B, B, B, B, B \})$	
				\>	$\rightarrow$ 0		\>	$Bits$
		\end{tabbing}
		\normalsize
	\end{itemize}
	
	\begin{boxBlueNoFrame}
		\footnotesize
		\highlight{If both classes are equally distributed, the entropy function $E$ reaches its maximum.
		Pure data sets have minimal entropy}.
	\end{boxBlueNoFrame}
\end{frame}


% Measure of Impurity: Entropy (Ctd.)
\begin{frame}{Measure of Impurity: Entropy (Ctd.)}{}
	\begin{figure}
	\centering
	\begin{tikzpicture}
    
    		\begin{axis}[
			scale=0.7,
			xlabel={Relative frequency of class $A$ ($p_A$)},
			ylabel={Entropy $E(\mathcal{D})$},
			grid=both,
    			grid style={line width=.1pt, draw=gray!10},
			minor tick num=2,
    			major grid style={line width=.2pt,draw=gray!80},
			minor grid style={line width=.2pt,draw=gray!30},
  			x=10cm,
  			y=5.5cm
		]

			\pgfplotstableread{08_decision_trees/05_data/data_entropy.txt} \datatable
			\addplot[myblue1,no marks,ultra thick,smooth] table[x=x,y=y] from \datatable;

			\node[fill=myblue1] at (axis cs:0.5,0.45) {\footnotesize $\textcolor{white}{E(\mathcal{D}) =
				-\sum_{c \in \mathcal{C}} p_c \cdot \log_2 p_c}$};
		
		\end{axis}
	\end{tikzpicture}
\end{figure}
\end{frame}


% Measure of Impurity: Entropy (Ctd.)
\begin{frame}{Measure of Impurity: Entropy (Ctd.)}{}
	\begin{boxBlueNoFrame}
		\highlight{Entropy formula:}
		\begin{equation}
			E(\mathcal{D}) = -\sum_{c \in \mathcal{C}} p_c \cdot \log_2 p_c
		\end{equation}
	\end{boxBlueNoFrame}

	\begin{itemize}
		\item Where $p_c$ denotes the relative frequency of class $c \in \mathcal{C}$	
		\item \textbf{Weather data:}
		\begin{align*}
			\mathcal{C} &= \{ yes, no \} \qquad \text{i.\,e.} \qquad
				p_{yes} = \nicefrac{9}{14} \quad \text{and} \quad p_{no} = \nicefrac{5}{14} \\[3mm]
			E(\mathcal{D})
				&= -\sum_{c \in \mathcal{C}} p_c \cdot \log_2 (p_c)
	       		   	= - (\nicefrac{9}{14} \cdot \log_2 (\nicefrac{9}{14}) + \nicefrac{5}{14} \cdot \log_2 (\nicefrac{5}{14}))
				= \boldsymbol{0.9403}
		\end{align*}
	\end{itemize}
\end{frame}


% Section: Wrap-Up
%______________________________________________________________________
\section{Wrap-Up}
\makedivider{Wrap-Up}

% Subsection: Summary
% --------------------------------------------------------------------------------------------------------
\subsection{Summary}

% Summary
\begin{frame}{Summary}{}

\end{frame}


% Subsection: Lecture Overview
% --------------------------------------------------------------------------------------------------------
\subsection{Lecture Overview}

\makeoverview{3}


% Subsection: Self-Test Questions
% --------------------------------------------------------------------------------------------------------
\subsection{Self-Test Questions}

% Self-Test Questions
\begin{frame}{Self-Test Questions}{}

\end{frame}


% Subsection: Recommended Literature and further Reading
% --------------------------------------------------------------------------------------------------------
\subsection{Recommended Literature and further Reading}

% Literature
%______________________________________________________________________
\begin{frame}{Recommended Literature and further Reading}{}
	\footnotesize
	\begin{thebibliography}{2}

	\end{thebibliography}
\end{frame}


% Thank you
%______________________________________________________________________
\makethanks

\end{document}