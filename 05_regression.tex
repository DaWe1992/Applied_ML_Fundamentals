\documentclass[11pt]{article}

% ====================================================
% ====================================================
% USEPACKAGES AND IMPORTS
% ====================================================
% ====================================================

\usepackage[T1]{fontenc}
\usepackage[utf8]{inputenc}
\usepackage[english]{babel}

\usepackage{fancyhdr}

% definitions
% ====================================================
\let\titleoriginal\title
\renewcommand{\title}[1]{
	\titleoriginal{#1}
	\newcommand{\thetitle}{#1}
}

\setlength{\parskip}{\baselineskip}%
\setlength{\parindent}{0pt}%

% header and footer
\pagestyle{fancy}
\fancyhf{}
\lhead{Applied Machine Learning Fundamentals}
\rhead{\thetitle}
\cfoot{\thepage}

% ====================================================
% ====================================================
% PRESENTATION DATA
% ====================================================
% ====================================================

\title[Machine Learning Introduction]{*** Applied Machine Learning Fundamentals *** Regression}
\institute{SAP\,SE}
\author{Daniel Wehner}
\date{\today}
\prefix{REG}

% ====================================================
% ====================================================
% BEGIN OF DOCUMENT
% ====================================================
% ====================================================

\begin{document}

% Title frame
%______________________________________________________________________
\maketitlepage


% Agenda
%______________________________________________________________________
\begin{frame}{Agenda \today}
	\begin{multicols}{2}
		\tableofcontents
	\end{multicols}
\end{frame}


% Section: Introduction to Regression
%______________________________________________________________________
\section{Introduction to Regression}
\makedivider{Introduction to Regression}


% Regression
\begin{frame}{Regression}{}
	\vspace*{-6mm}
	\begin{tabbing}
		\hspace*{7cm}\= \kill
		\highlight{Type of target variable}			\> Continuous 		\\[1mm]
		\highlight{Type of training information} 	\> Supervised 		\\[1mm]
		\highlight{Example Availability}			\> Batch learning		\\[1mm]
	\end{tabbing}
	
	\vspace*{-7mm}
	\footnotesize
	\begin{boxBlue}
		\textbf{Algorithm sketch:} Given the training data $\mathcal{D}$ the algorithm derives
		a function of the type
	
		\vspace*{-6mm}
		\begin{equation}
			h_{\bm{\theta}}(\bm{x}) = \theta_0 + \theta_1 x_1 + \dots + \theta_{m+1} x_{m}
				\qquad \bm{x} \in \mathbb{R}^{m}, \bm{\theta} \in \mathbb{R}^{m+1}
		\end{equation}

		from the data. $\bm{\theta}$ is the parameter vector containing the coefficients to be estimated by the regression
 		algorithm. Once $\bm{\theta}$ is learned it can be used for prediction. 
	\end{boxBlue}
\end{frame}


% Example Data Set: Revenues
\begin{frame}{Example Data Set: Revenues}{}
	\divideTwo{0.44}{
		\begin{figure}
	\centering
	\begin{tikzpicture}
    
  	  \begin{axis}[
			scale=0.6,
			axis lines=middle,
			xlabel={\textbf{Marketing Expenses} $x_1$},
			ylabel={\textbf{Revenue} $y$},
    			xtick={\empty},ytick={\empty},
			x label style={at={(axis description cs:0.5,-0.3)},anchor=south},
    			y label style={at={(axis description cs:0,1)},anchor=south}
		]

			\pgfplotstableread{05_regression/05_data/data_regression_marketing_expenses.txt} \datatable
			\addplot[only marks,mark=*,mark size=2,draw=myblue1,fill=lightgray] table[x=x,y=y] from \datatable;
		\end{axis}
	\end{tikzpicture}
\end{figure}
	}{0.55}{
		\begin{itemize}
			\item Find a linear function:
			\begin{equation*}
				h_{\bm{\theta}}(\bm{x}) = \theta_0 + \theta_1 x_1 + \dots + \theta_{m+1} x_{m}
			\end{equation*}
			\item Usually: $x_0 = 1$:
		\end{itemize}
		\begin{align*}
			\bm{\widehat{x}} \in \mathbb{R}^{m+1}
				&= [1\ \bm{x}]^{\intercal} \\
			h_{\bm{\theta}}(\bm{\widehat{x}}) = \sum_{j=0}^{m+1} \theta_j x_j
				&= \bm{\theta}^{\intercal} \bm{\widehat{x}}
		\end{align*}
	}
\end{frame}


% Error Function for Regression
\begin{frame}{Error Function for Regression}{}
	\begin{itemize}
		\item In order to know how good the function fits we need an error function $\mathcal{J}(\bm{\theta})$:
		\begin{equation}
			\mathcal{J}(\bm{\theta}) = \frac{1}{2n} \sum_{i=1}^n (h_{\bm{\theta}}(\bm{\widehat{x}}^{(i)}) - y^{(i)})^2
		\end{equation}
		\item We want to minimize $\mathcal{J}(\bm{\theta})$:
		\begin{equation*}
			\min_{\bm{\theta}} \frac{1}{2n} \sum_{i=1}^n (h_{\bm{\theta}}(\bm{\widehat{x}}^{(i)}) - y^{(i)})^2
		\end{equation*}
		\item This is \highlight{ordinary least squares (OLS)}
	\end{itemize}
\end{frame}


% Error Function Intuition
\begin{frame}{Error Function Intuition}{}
	% tikz image showing least squares
\end{frame}


% Closed-Form Solutions
\begin{frame}{Closed-Form Solutions}{}
	\bubble{0.5}{5}{
		\scriptsize sample mean $\overline x$
	}
	\begin{itemize}
		\item Usual approach (for two unknowns): Calculate $\theta_0$ and $\theta_1$ according to
		\begin{equation}
			\theta_0 = \overline{y} - \theta_2 \overline{x} \qquad\qquad
			\theta_1 = \frac{\sum_{i=1}^n (x^{(i)} - \overline{x}) \cdot
				(y^{(i)} - \overline{y})}{\sum_{i=1}^n (x^{(i)} - \overline{x})^2}
		\end{equation}
		\item \highlight{`Normal equation'} (scales to arbitrary dimensions):
		\begin{equation}
			\bm{\theta} = \underbracket{
				(\bm{\widehat{X}}^{\intercal} \bm{\widehat{X}})^{-1} \bm{\widehat{X}}^{\intercal}
			}_{\substack{\text{Moore-Penrose} \\ \text{pseudo-inverse}}} \bm{y}
		\end{equation}
		$\bm{\widehat{X}}$ is called `design matrix' or `regressor matrix'
	\end{itemize}
\end{frame}


% Design Matrix / Regressor Matrix
\begin{frame}{Design Matrix / Regressor Matrix}{}
	\bubble{12}{4}{
		\scriptsize In the following \\
		\scriptsize $\bm{\widehat{X}} \equiv \bm{X}$
	}
	\begin{itemize}
		\item The design matrix $\bm{\widehat{X}} \in \mathbb{R}^{n \times (m + 1)}$ looks as follows:
		\footnotesize
		\begin{equation}
			\bm{\widehat{X}} =
			\begin{pmatrix}
  				1 & x_1^{(1)} & x_2^{(1)} & \cdots & x_m^{(1)} \\
				1 & x_1^{(2)} & x_2^{(2)} & \cdots & x_m^{(2)} \\
				1 & x_1^{(3)} & x_2^{(3)} & \cdots & x_m^{(3)} \\
				\vdots & \vdots & \vdots & \ddots & \vdots \\
				1 & x_1^{(n)} & x_2^{(n)} & \cdots & x_m^{(n)} \\
 			\end{pmatrix}
		\end{equation}
		\normalsize
		\item And the $n \times 1$ label vector:
		\footnotesize
		\begin{equation*}
			 \bm{y} =
			 \begin{pmatrix}
				y^{(1)}, y^{(2)}, y^{(3)}, \dots, y^{(n)}
			\end{pmatrix}^{\intercal}
		\end{equation*}
	\end{itemize}
\end{frame}


% Derivation of the Normal Equation
\begin{frame}{Derivation of the Normal Equation}{}\optional
	\begin{itemize}
		\item The derivation involves a bit of linear algebra
		\item Step \ding{182}: Rewrite $\mathcal{J}(\bm{\theta})$ in matrix-vector notation:
		\begin{align*}
			\mathcal{J}(\bm{\theta})
				&= \frac{1}{2}(\bm{X} \bm{\theta} - \bm{y})^{\intercal} (\bm{X} \bm{\theta} - \bm{y}) \\
				&= ((\bm{X} \bm{\theta})^{\intercal} - \bm{y}^{\intercal}) (\bm{X} \bm{\theta} - \bm{y}) \\
				&= (\bm{X} \bm{\theta})^{\intercal} \bm{X} \bm{\theta} - (\bm{X} \bm{\theta})^{\intercal} \bm{y}
					- \bm{y}^{\intercal} (\bm{X} \bm{\theta}) + \bm{y}^{\intercal} \bm{y} \\
				&= \bm{\theta}^{\intercal} \bm{X}^{\intercal} \bm{X} \bm{\theta}
					- 2 (\bm{X} \bm{\theta})^{\intercal} \bm{y} + \bm{y}^{\intercal} \bm{y}
		\end{align*}
		\item To be continued...
	\end{itemize}
\end{frame}


% Derivation of the Normal Equation (Ctd.)
\begin{frame}{Derivation of the Normal Equation (Ctd.)}{}\optional
	\begin{itemize}
		\item Step \ding{183}: Calculate the derivative of $\mathcal{J}(\bm{\theta})$ and set it to zero:
		\begin{align*}
			\nabla_{\bm{\theta}}\mathcal{J}(\bm{\theta})
				&= 2 \bm{X}^{\intercal} \bm{X} \bm{\theta} - 2 \bm{X}^{\intercal} \bm{y} \overset{!}{=} \bm{0} \\	
				&\Leftrightarrow \bm{X}^{\intercal} \bm{X} \bm{\theta} = \bm{X}^{\intercal} \bm{y}
		\end{align*}
		\item If $\bm{X}^{\intercal} \bm{X}$ is invertible, we can multiply both sides by
			$(\bm{X}^{\intercal} \bm{X})^{-1}$:
		\vspace*{3mm}
		\begin{boxBlueNoFrame}
			\highlight{Normal equation:}
			\begin{equation*}
				\bm{\theta} = (\bm{X}^{\intercal} \bm{X})^{-1} \bm{X}^{\intercal} \bm{y}
			\end{equation*}
		\end{boxBlueNoFrame}
	\end{itemize}
\end{frame}


% Section: Wrap-Up
%______________________________________________________________________
\section{Wrap-Up}

% Subsection: Summary
% --------------------------------------------------------------------------------------------------------
\subsection{Summary}

% Summary
\begin{frame}{Summary}{}

\end{frame}


% Subsection: Lecture Overview
% --------------------------------------------------------------------------------------------------------
\subsection{Lecture Overview}

\makeoverview{3}


% Subsection: Self-Test Questions
% --------------------------------------------------------------------------------------------------------
\subsection{Self-Test Questions}

% Self-Test Questions
\begin{frame}{Self-Test Questions}{}

\end{frame}

% Subsection: Recommended Literature and further Reading
% --------------------------------------------------------------------------------------------------------
\subsection{Recommended Literature and further Reading}

% Literature
%______________________________________________________________________
\begin{frame}{Recommended Literature and further Reading}{}
	\footnotesize
	\begin{thebibliography}{2}

	\end{thebibliography}
\end{frame}


% Thank you
%______________________________________________________________________
\makethanks

\end{document}